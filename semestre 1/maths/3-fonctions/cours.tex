%%=====================================================================================
%%
%%       Filename:  cours.tex
%%
%%    Description:  
%%
%%        Version:  1.0
%%        Created:  03/06/2024
%%       Revision:  none
%%
%%         Author:  YOUR NAME (), 
%%   Organization:  
%%      Copyright:  Copyright (c) 2024, YOUR NAME
%%
%%          Notes:  
%%
%%=====================================================================================
\documentclass[a4paper, titlepage]{article}

\usepackage[utf8]{inputenc}
\usepackage[T1]{fontenc}
\usepackage{textcomp}
\usepackage[french]{babel}
\usepackage{amsmath, amssymb}
\usepackage{amsthm}
\usepackage[svgnames]{xcolor}
\usepackage{thmtools}
\usepackage{lipsum}
\usepackage{framed}
\usepackage{parskip}
\usepackage{titlesec}

\renewcommand{\familydefault}{\sfdefault}

% figure support
\usepackage{import}
\usepackage{xifthen}
\pdfminorversion=7
\usepackage{pdfpages}
\usepackage{transparent}
\newcommand{\incfig}[1]{%
	\def\svgwidth{\columnwidth}
	\import{./figures/}{#1.pdf_tex}
}

\pdfsuppresswarningpagegroup=1

\colorlet{defn-color}{DarkBlue}
\colorlet{props-color}{Blue}
\colorlet{warn-color}{Red}
\colorlet{exemple-color}{Green}
\colorlet{corol-color}{Orange}
\newenvironment{defn-leftbar}{%
  \def\FrameCommand{{\color{defn-color}\vrule width 3pt} \hspace{10pt}}%
  \MakeFramed {\advance\hsize-\width \FrameRestore}}%
 {\endMakeFramed}
\newenvironment{warn-leftbar}{%
  \def\FrameCommand{{\color{warn-color}\vrule width 3pt} \hspace{10pt}}%
  \MakeFramed {\advance\hsize-\width \FrameRestore}}%
 {\endMakeFramed}
\newenvironment{exemple-leftbar}{%
  \def\FrameCommand{{\color{exemple-color}\vrule width 3pt} \hspace{10pt}}%
  \MakeFramed {\advance\hsize-\width \FrameRestore}}%
 {\endMakeFramed}
\newenvironment{props-leftbar}{%
  \def\FrameCommand{{\color{props-color}\vrule width 3pt} \hspace{10pt}}%
  \MakeFramed {\advance\hsize-\width \FrameRestore}}%
 {\endMakeFramed}
\newenvironment{corol-leftbar}{%
  \def\FrameCommand{{\color{corol-color}\vrule width 3pt} \hspace{10pt}}%
  \MakeFramed {\advance\hsize-\width \FrameRestore}}%
 {\endMakeFramed}

\def \freespace {1em}
\declaretheoremstyle[headfont=\sffamily\bfseries,%
 notefont=\sffamily\bfseries,%
 notebraces={}{},%
 headpunct=,%
 bodyfont=\sffamily,%
 headformat=\color{defn-color}Définition~\NUMBER\hfill\NOTE\smallskip\linebreak,%
 preheadhook=\vspace{\freespace}\begin{defn-leftbar},%
 postfoothook=\end{defn-leftbar},%
]{better-defn}
\declaretheoremstyle[headfont=\sffamily\bfseries,%
 notefont=\sffamily\bfseries,%
 notebraces={}{},%
 headpunct=,%
 bodyfont=\sffamily,%
 headformat=\color{warn-color}Attention\hfill\NOTE\smallskip\linebreak,%
 preheadhook=\vspace{\freespace}\begin{warn-leftbar},%
 postfoothook=\end{warn-leftbar},%
]{better-warn}
\declaretheoremstyle[headfont=\sffamily\bfseries,%
 notefont=\sffamily\bfseries,%
notebraces={}{},%
headpunct=,%
 bodyfont=\sffamily,%
 headformat=\color{exemple-color}Exemple~\NUMBER\hfill\NOTE\smallskip\linebreak,%
 preheadhook=\vspace{\freespace}\begin{exemple-leftbar},%
 postfoothook=\end{exemple-leftbar},%
]{better-exemple}
\declaretheoremstyle[headfont=\sffamily\bfseries,%
 notefont=\sffamily\bfseries,%
 notebraces={}{},%
 headpunct=,%
 bodyfont=\sffamily,%
 headformat=\color{props-color}Proposition~\NUMBER\hfill\NOTE\smallskip\linebreak,%
 preheadhook=\vspace{\freespace}\begin{props-leftbar},%
 postfoothook=\end{props-leftbar},%
]{better-props}
\declaretheoremstyle[headfont=\sffamily\bfseries,%
 notefont=\sffamily\bfseries,%
 notebraces={}{},%
 headpunct=,%
 bodyfont=\sffamily,%
 headformat=\color{props-color}Théorème~\NUMBER\hfill\NOTE\smallskip\linebreak,%
 preheadhook=\vspace{\freespace}\begin{props-leftbar},%
 postfoothook=\end{props-leftbar},%
]{better-thm}
\declaretheoremstyle[headfont=\sffamily\bfseries,%
 notefont=\sffamily\bfseries,%
 notebraces={}{},%
 headpunct=,%
 bodyfont=\sffamily,%
 headformat=\color{corol-color}Corollaire~\NUMBER\hfill\NOTE\smallskip\linebreak,%
 preheadhook=\vspace{\freespace}\begin{corol-leftbar},%
 postfoothook=\end{corol-leftbar},%
]{better-corol}

\declaretheorem[style=better-defn]{defn}
\declaretheorem[style=better-warn]{warn}
\declaretheorem[style=better-exemple]{exemple}
\declaretheorem[style=better-corol]{corol}
\declaretheorem[style=better-props, numberwithin=defn]{props}
\declaretheorem[style=better-thm, sibling=props]{thm}
\newtheorem*{lemme}{Lemme}%[subsection]
%\newtheorem{props}{Propriétés}[defn]

\newenvironment{system}%
{\left\lbrace\begin{align}}%
{\end{align}\right.}

\newenvironment{AQT}{{\fontfamily{qbk}\selectfont AQT}}

\usepackage{LobsterTwo}
\titleformat{\section}{\newpage\LobsterTwo \huge\bfseries}{\thesection.}{1em}{}
\titleformat{\subsection}{\vspace{2em}\LobsterTwo \Large\bfseries}{\thesubsection.}{1em}{}
\titleformat{\subsubsection}{\vspace{1em}\LobsterTwo \large\bfseries}{\thesubsubsection.}{1em}{}

\newenvironment{lititle}%
{\vspace{7mm}\LobsterTwo \large}%
{\\}

\renewenvironment{proof}{$\square$ \footnotesize\textit{Démonstration.}}{\begin{flushright}$\blacksquare$\end{flushright}}

\title{Fonctions}
\author{William Hergès\thanks{Sorbonne Université - Faculté des Sciences, Faculté des Lettres}}

\begin{document}
	\maketitle
	\tableofcontents
	\newpage
	\section{Ensemble de définition et continuité}
	\begin{defn}
		Une fonction $f$ de $A$ dans $B$ prend toutes les valeurs dans $A$ et lui associe une unique valeur dans $B$.

		On appelle $A$ le domaine de définition de $f$ et $B$ l'ensemble d'arrivée de $f$.\\
		On dit que $f$ est de $\mathbb{R}$ dans $\mathbb{R}$.
	\end{defn}
	\begin{defn}
		Une fonction $f:I\to E$ est continue si et seulement si :
		$$ \forall x\in I,\quad \lim_{n \to x_0} f(x) = f(x_0) $$
	\end{defn}
	\section{Propriétés des applications}
	\begin{defn}
		Une fonction $f$ de $A$ dans $B$ est surjective si et seulement si :
		$$ \forall y\in B,\quad\exists x\in A,\quad f(x)=y $$
	\end{defn}
	Tous les éléments de $B$ sont atteints, comme la tarte.
	\begin{defn}
		Une fonction $f$ de $A$ dans $B$ est injective si et seulement si :
		$$ \forall (x_1,x_2)\in A^2,\quad f(x_1)=f(x_2) \implies x_1=x_2 $$
	\end{defn}
	Chaque élément de $A$ possède une image unique.
	\begin{defn}
		Une fonction $f$ de $A$ dans $B$ est bijective si et seulement si elle est injective et surjective.
	\end{defn}
	Tous les éléments de $B$  sont atteints et est l'image d'un unique antécédent de $A$.
	\begin{defn}
		Une fonction $f$ de $A$ dans $B$ est paire si et seulement si :
		$$ \forall x\in\mathbb{R},\quad f(x)=f(-x) $$
	\end{defn}
	\begin{defn}
		Une fonction $f$ de $A$ dans $B$ est impaire si et seulement si :
		$$ \forall x\in\mathbb{R},\quad f(x)=-f(x) $$
	\end{defn}
	\begin{defn}
		Si $f$ de $A$ dans $B$ est bijective, alors il existe une fonction réciproque notée $f^{-1}$ de $B$ dans $A$ tel que :
		$$ \forall a\in A,\quad f^{-1}(f(a)) = a $$
	\end{defn}
	\begin{props}
		$f^{-1}$ est bijective et sa réciproque est $f$.
	\end{props}
	\begin{proof}
		\AQT
	\end{proof}
	\begin{defn}
		On dit que $f$ est $T$-périodique si et seulement si :
		$$ \forall x\in A,\quad f(x)=f(x+T) $$
	\end{defn}
	\section{Grands théorèmes}
	\begin{thm}[Théorème des valeurs intermédiaires]
		Si $f$ est continue sur $[a,b]$, si pour tout $y$ inclus entre $f(a)$ et $f(b)$, il existe $c\in[a,b]$ tel que $f(c)=y$.
	\end{thm}
	\begin{proof}
		Admis
	\end{proof}
	\begin{thm}[Théorème de la bijection]
		Soit $f$ une fonction de $[a,b]$ dans $E$. Si $f$ est strictement croissante (resp. strictement décroissante), alors :
		$$ \forall y\in [f(a),f(b)],\quad\exists !c\in[a,b],\quad f(c)=y $$
	\end{thm}
	\begin{proof}
		Admis
	\end{proof}
	\section{Dérivation}
	\begin{defn}
		La dérivée de $f$ de $A$ dans $B$ est :
		$$f'\begin{system}
			A &\to b\\
			x&\longmapsto \lim_{t \to x_0} \frac{f(t)-f(x)}{t-x}
		\end{system}$$
		si la limite est définie.
	\end{defn}
	\begin{props}
		Soient deux fonctions $f$ et $g$ dérivables. \\
		On a :
		\begin{align*}
			(f+g)' &= f'+g'\\
			(fg)'  &= f'g+g'f \\
			\left(\frac{f}{g}\right)' &= \frac{f'g-g'f}{g^2} 
		\end{align*}
	\end{props}
	\begin{proof}
		\AQT
	\end{proof}
	\begin{defn}
		Soient $f:A\to B$ et $g:B\to C$.\\
		On définit $g\circ f$ la composée de $f$ par $g$ tel que :
		$$g\circ f\begin{system}
			A&\to C\\
			x\longmapsto g(f(x))
		\end{system}$$
	\end{defn}
	\begin{props}
		La dérivée de $g\circ f$ est $$g\circ f'\times f'$$
	\end{props}
	\begin{proof}
		Chiant mais \AQT
	\end{proof}
	\begin{thm}
		La dérivée de $f^{-1}$ est (si elle est dérivable) :
		$$ f^{-1}'(x)=\frac{1}{f'(f^{-1}(y))} $$
	\end{thm}
	\begin{proof}
		Chiant, mais très intuitif graphiquement
	\end{proof}
\end{document}
