%%=====================================================================================
%%
%%       Filename:  cours.tex
%%
%%    Description:  
%%
%%        Version:  1.0
%%        Created:  03/06/2024
%%       Revision:  none
%%
%%         Author:  YOUR NAME (), 
%%   Organization:  
%%      Copyright:  Copyright (c) 2024, YOUR NAME
%%
%%          Notes:  
%%
%%=====================================================================================
\documentclass[a4paper, titlepage]{article}

\usepackage[utf8]{inputenc}
\usepackage[T1]{fontenc}
\usepackage{textcomp}
\usepackage[french]{babel}
\usepackage{amsmath, amssymb}
\usepackage{amsthm}
\usepackage[svgnames]{xcolor}
\usepackage{thmtools}
\usepackage{lipsum}
\usepackage{framed}
\usepackage{parskip}
\usepackage{titlesec}

\renewcommand{\familydefault}{\sfdefault}

% figure support
\usepackage{import}
\usepackage{xifthen}
\pdfminorversion=7
\usepackage{pdfpages}
\usepackage{transparent}
\newcommand{\incfig}[1]{%
	\def\svgwidth{\columnwidth}
	\import{./figures/}{#1.pdf_tex}
}

\pdfsuppresswarningpagegroup=1

\colorlet{defn-color}{DarkBlue}
\colorlet{props-color}{Blue}
\colorlet{warn-color}{Red}
\colorlet{exemple-color}{Green}
\colorlet{corol-color}{Orange}
\newenvironment{defn-leftbar}{%
  \def\FrameCommand{{\color{defn-color}\vrule width 3pt} \hspace{10pt}}%
  \MakeFramed {\advance\hsize-\width \FrameRestore}}%
 {\endMakeFramed}
\newenvironment{warn-leftbar}{%
  \def\FrameCommand{{\color{warn-color}\vrule width 3pt} \hspace{10pt}}%
  \MakeFramed {\advance\hsize-\width \FrameRestore}}%
 {\endMakeFramed}
\newenvironment{exemple-leftbar}{%
  \def\FrameCommand{{\color{exemple-color}\vrule width 3pt} \hspace{10pt}}%
  \MakeFramed {\advance\hsize-\width \FrameRestore}}%
 {\endMakeFramed}
\newenvironment{props-leftbar}{%
  \def\FrameCommand{{\color{props-color}\vrule width 3pt} \hspace{10pt}}%
  \MakeFramed {\advance\hsize-\width \FrameRestore}}%
 {\endMakeFramed}
\newenvironment{corol-leftbar}{%
  \def\FrameCommand{{\color{corol-color}\vrule width 3pt} \hspace{10pt}}%
  \MakeFramed {\advance\hsize-\width \FrameRestore}}%
 {\endMakeFramed}

\def \freespace {1em}
\declaretheoremstyle[headfont=\sffamily\bfseries,%
 notefont=\sffamily\bfseries,%
 notebraces={}{},%
 headpunct=,%
 bodyfont=\sffamily,%
 headformat=\color{defn-color}Définition~\NUMBER\hfill\NOTE\smallskip\linebreak,%
 preheadhook=\vspace{\freespace}\begin{defn-leftbar},%
 postfoothook=\end{defn-leftbar},%
]{better-defn}
\declaretheoremstyle[headfont=\sffamily\bfseries,%
 notefont=\sffamily\bfseries,%
 notebraces={}{},%
 headpunct=,%
 bodyfont=\sffamily,%
 headformat=\color{warn-color}Attention\hfill\NOTE\smallskip\linebreak,%
 preheadhook=\vspace{\freespace}\begin{warn-leftbar},%
 postfoothook=\end{warn-leftbar},%
]{better-warn}
\declaretheoremstyle[headfont=\sffamily\bfseries,%
 notefont=\sffamily\bfseries,%
notebraces={}{},%
headpunct=,%
 bodyfont=\sffamily,%
 headformat=\color{exemple-color}Exemple~\NUMBER\hfill\NOTE\smallskip\linebreak,%
 preheadhook=\vspace{\freespace}\begin{exemple-leftbar},%
 postfoothook=\end{exemple-leftbar},%
]{better-exemple}
\declaretheoremstyle[headfont=\sffamily\bfseries,%
 notefont=\sffamily\bfseries,%
 notebraces={}{},%
 headpunct=,%
 bodyfont=\sffamily,%
 headformat=\color{props-color}Proposition~\NUMBER\hfill\NOTE\smallskip\linebreak,%
 preheadhook=\vspace{\freespace}\begin{props-leftbar},%
 postfoothook=\end{props-leftbar},%
]{better-props}
\declaretheoremstyle[headfont=\sffamily\bfseries,%
 notefont=\sffamily\bfseries,%
 notebraces={}{},%
 headpunct=,%
 bodyfont=\sffamily,%
 headformat=\color{props-color}Théorème~\NUMBER\hfill\NOTE\smallskip\linebreak,%
 preheadhook=\vspace{\freespace}\begin{props-leftbar},%
 postfoothook=\end{props-leftbar},%
]{better-thm}
\declaretheoremstyle[headfont=\sffamily\bfseries,%
 notefont=\sffamily\bfseries,%
 notebraces={}{},%
 headpunct=,%
 bodyfont=\sffamily,%
 headformat=\color{corol-color}Corollaire~\NUMBER\hfill\NOTE\smallskip\linebreak,%
 preheadhook=\vspace{\freespace}\begin{corol-leftbar},%
 postfoothook=\end{corol-leftbar},%
]{better-corol}

\declaretheorem[style=better-defn]{defn}
\declaretheorem[style=better-warn]{warn}
\declaretheorem[style=better-exemple]{exemple}
\declaretheorem[style=better-corol]{corol}
\declaretheorem[style=better-props, numberwithin=defn]{props}
\declaretheorem[style=better-thm, sibling=props]{thm}
\newtheorem*{lemme}{Lemme}%[subsection]
%\newtheorem{props}{Propriétés}[defn]

\newenvironment{system}%
{\left\lbrace\begin{align}}%
{\end{align}\right.}

\newenvironment{AQT}{{\fontfamily{qbk}\selectfont AQT}}

\usepackage{LobsterTwo}
\titleformat{\section}{\newpage\LobsterTwo \huge\bfseries}{\thesection.}{1em}{}
\titleformat{\subsection}{\vspace{2em}\LobsterTwo \Large\bfseries}{\thesubsection.}{1em}{}
\titleformat{\subsubsection}{\vspace{1em}\LobsterTwo \large\bfseries}{\thesubsubsection.}{1em}{}

\newenvironment{lititle}%
{\vspace{7mm}\LobsterTwo \large}%
{\\}

\renewenvironment{proof}{$\square$ \footnotesize\textit{Démonstration.}}{\begin{flushright}$\blacksquare$\end{flushright}}

\title{Rappels}
\author{William Hergès\thanks{Sorbonne Université - Faculté des Sciences, Faculté des Lettres}}

\begin{document}
	\maketitle
	\tableofcontents
	\newpage
	Rappels en vrac.
	\section{Vecteur}
	\subsection{Norme d'un vecteur}
	\begin{defn}
		La norme du vecteur $\vec v$ se note $||\vec v||$ et $$||\vec v||=\sqrt{(x_b-x_a)^2+(y_b-y_a)^2+(z_b-z_a)^2}$$
	\end{defn}
	\subsection{Produit scalaire}
	\begin{defn}
		Le produit scalaire entre $\vec v$ et $\vec u$ se note $\vec v\cdot\vec u$ et $$\vec v\cdot\vec u=||\vec u||\times||\vec v||\cos\alpha$$
	\end{defn}
	\begin{props}
		Pour tous vecteurs $\vec u$, $\vec v$ et $\vec w$, on a :
		\begin{itemize}
			\item $\vec u\cdot\vec v = \vec v\cdot \vec u$
			\item $\forall \lambda\in\mathbb{R},\quad(\lambda\vec u)\cdot\vec v = \lambda\cdot(\vec u\vec v)$
			\item $(\vec u+\vec v)\vec w = \vec u\vec w+\vec w\vec v$
		\end{itemize}
	\end{props}
	\begin{proof}
		\AQT
	\end{proof}
	\begin{props}
		Pour tous vecteurs $\vec u$ et $\vec v$, on a :
		\begin{itemize}
			\item $||\vec u+\vec v||^2 = (\vec u+\vec v)\cdot(\vec u+\vec v)$
			\item $||\vec u-\vec v||^2 = (\vec u-\vec v)\cdot(\vec u-\vec v)$
		\end{itemize}
	\end{props}
	\begin{proof}
		\AQT
	\end{proof}
	\subsubsection{Produit scalaire dans une bose orthonormée}
	\begin{defn}
		Un vecteur est alors caractérisé par trois coordonnées (qui sont celles du point d'arrivé si le vecteur part du point d'origine). On peut alors écrire :
		$$ \vec u = x\vec i+y\vec j+z\vec k $$
		(où $(x,y,z)\in\mathbb{R}^3$ sont les coordonnées du vecteur $\vec u$ et $(\vec i,\vec j, \vec k)$ une base)
	\end{defn}
	\begin{props}
		Pour deux vecteurs $\vec u$ et $\vec v$, on a :
		$$ \vec u\cdot\vec v = xx'+yy'+zz' $$
		(où $(x,y,z)\in\mathbb{R}^3$ sont les coordonnées de $\vec u$ et $(x',y',z')$ sont les coordonnées de $\vec v$)
	\end{props}
	\begin{proof}
		\AQT
	\end{proof}
	\subsection{Produit vectoriel}
	\begin{defn}
		Soient $\vec u$ et $\vec v$ deux vecteurs dans une base orthonormée.

		Le produit vectoriel de $\vec v$ par $\vec u$ est noté $ \vec u\land \vec v$ et est le vecteur perpendiculaire à $\vec u$ et à $\vec v$ de norme $u\times v\times\sin\alpha$ (où $\alpha$ est l'angle entre $\vec u$ et $\vec v$) dirigé selon "la règle de la main droite".
	\end{defn}
	\begin{props}
		On a :
		$$ 
		\begin{pmatrix} x\\y\\z \end{pmatrix} \land \begin{pmatrix} x'\\y'\\z' \end{pmatrix} =
		\begin{pmatrix} yz'-zy'\\zx'-xz'\\xy'-yx' \end{pmatrix} 
		$$
	\end{props}
	\begin{proof}
		\AQT (besoin de linéarité comme lemme)
	\end{proof}
	\begin{warn}
		$\vec u\land\vec v = -\vec v\land\vec u$
	\end{warn}
	\begin{lititle}
		Application principale
	\end{lititle}
	Il sert à obtenir un vecteur orthogonal à deux autres.
	\section{Droites et plans}
	\subsection{Droites}
	\begin{defn}
		Une droite $\Delta$ dirigée par $\vec u = \begin{pmatrix} \alpha\\\beta \end{pmatrix}$ passant par $A=(a,b)$ est l'ensemble des points $P$ sastisfaisant cette relation :
		$$ \forall M\in P,\quad\exists k\in\mathbb{R},\quad \overrightarrow{AM} = k\vec u $$
	\end{defn}
	\begin{props}[Équations paramétriques]
		Une droite $\Delta$ dirigée par $\begin{pmatrix} \alpha\\\beta \end{pmatrix}$ passant par $(a,b)$ possède comme équations paramétriques :
		$$\begin{system}
			x=a+t\alpha\\
			y=b+t\beta
		\end{system},\quad (t\in\mathbb{R})$$
		pour un point de coordonnées $(x,y)$ appartenant à $\Delta$.
	\end{props}
	\begin{proof}
		\AQT
	\end{proof}
	\begin{props}[Équation cartésienne]
		Une droite $\Delta$ dirigée par $\begin{pmatrix} \alpha\\\beta \end{pmatrix}$ passant par $(a,b)$ possède comme équation cartésienne : 
		$$ y-\frac{\beta}{\alpha}x=b-\frac{a\beta}{\alpha} $$
		pour un point de coordonnées $(x,y)$ appartenant à $\Delta$.
	\end{props}
	\begin{proof}
		\AQT
	\end{proof}
	C'est la même chose dans $\mathbb{R}^3$.
	\subsection{Plans}
	\begin{props}[Équations paramétriques]
		Soit $\Pi$ le plan passant par $A=(a,b,c)$ et dirigé par $\vec v = \begin{pmatrix} \alpha\\\beta\\\gamma \end{pmatrix}$ et par $\vec u = \begin{pmatrix} \alpha'\\\beta'\\\gamma' \end{pmatrix}$ possède comme équation paramétrique :
		$$\begin{system}
			x=a+t\alpha+s\alpha'\\
			y=b+t\beta+s\beta'\\
			z=c+t\gamma+s\gamma'
		\end{system},\quad (t,s\in\mathbb{R})$$
		pour un point de coordonnées $(x,y,z)$ appartenant à $\Pi$.
	\end{props}
	\begin{proof}
		\AQT
	\end{proof}
	\begin{props}
		Soit $\Pi$ un plan passant par $A$ et dirigée par $\vec u$ et $\vec v$.\\
		Soit $P$ un point de $\Pi$.

		Le vecteur $\overrightarrow{AP}$ est orthogonal à $\vec w$ (où $\vec w$ est un vecteur orthognal à $\vec u$ et $\vec v$).

		Autrement dit, $$ \overrightarrow{AP}\cdot (\vec u\land\vec v) = 0 $$
	\end{props}
	\begin{proof}
		\AQT
	\end{proof}
	Cela permet d'obtenir l'équation cartésienne du plan.
	\section{Familles libres, familles liées}
	\begin{defn}
		Une famille de $n$-vecteurs $(\vec u_1,\ldots,\vec u_n)\in\mathbb{R}^n$ est libre si, et seulement si, ces vecteurs sont linéairement indépendant. i.e.
		$$ \sum_{i=1}^{n} \lambda_i\vec u_i = \vec 0\iff \forall i\in[|1,n|],\quad\lambda_i = 0 $$
		avec $(\lambda_1,\ldots,\lambda_n)\in\mathbb{R}^n$.
	\end{defn}
	\begin{defn}
		Une famille de $n$-vecteurs est liée si, et seulement si, elle n'est pas libre.
	\end{defn}
\end{document}
