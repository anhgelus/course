%%=====================================================================================
%%
%%       Filename:  cours.tex
%%
%%    Description:  
%%
%%        Version:  1.0
%%        Created:  03/06/2024
%%       Revision:  none
%%
%%         Author:  YOUR NAME (), 
%%   Organization:  
%%      Copyright:  Copyright (c) 2024, YOUR NAME
%%
%%          Notes:  
%%
%%=====================================================================================
\documentclass[a4paper, titlepage]{article}

\usepackage[utf8]{inputenc}
\usepackage[T1]{fontenc}
\usepackage{textcomp}
\usepackage[french]{babel}
\usepackage{amsmath, amssymb}
\usepackage{amsthm}
\usepackage[svgnames]{xcolor}
\usepackage{thmtools}
\usepackage{lipsum}
\usepackage{framed}
\usepackage{parskip}
\usepackage{titlesec}
%\usepackage[cal=rsfs,calscale=1.03]{mathalpha}

\renewcommand{\familydefault}{\sfdefault}

% figure support
\usepackage{import}
\usepackage{xifthen}
\pdfminorversion=7
\usepackage{pdfpages}
\usepackage{transparent}
\newcommand{\incfig}[1]{%
	\def\svgwidth{\columnwidth}
	\import{./figures/}{#1.pdf_tex}
}

\pdfsuppresswarningpagegroup=1

\colorlet{defn-color}{DarkBlue}
\colorlet{props-color}{Blue}
\colorlet{warn-color}{Red}
\colorlet{exemple-color}{Green}
\colorlet{corol-color}{Orange}
\newenvironment{defn-leftbar}{%
  \def\FrameCommand{{\color{defn-color}\vrule width 3pt} \hspace{10pt}}%
  \MakeFramed {\advance\hsize-\width \FrameRestore}}%
 {\endMakeFramed}
\newenvironment{warn-leftbar}{%
  \def\FrameCommand{{\color{warn-color}\vrule width 3pt} \hspace{10pt}}%
  \MakeFramed {\advance\hsize-\width \FrameRestore}}%
 {\endMakeFramed}
\newenvironment{exemple-leftbar}{%
  \def\FrameCommand{{\color{exemple-color}\vrule width 3pt} \hspace{10pt}}%
  \MakeFramed {\advance\hsize-\width \FrameRestore}}%
 {\endMakeFramed}
\newenvironment{props-leftbar}{%
  \def\FrameCommand{{\color{props-color}\vrule width 3pt} \hspace{10pt}}%
  \MakeFramed {\advance\hsize-\width \FrameRestore}}%
 {\endMakeFramed}
\newenvironment{corol-leftbar}{%
  \def\FrameCommand{{\color{corol-color}\vrule width 3pt} \hspace{10pt}}%
  \MakeFramed {\advance\hsize-\width \FrameRestore}}%
 {\endMakeFramed}

\def \freespace {1em}
\declaretheoremstyle[headfont=\sffamily\bfseries,%
 notefont=\sffamily\bfseries,%
 notebraces={}{},%
 headpunct=,%
 bodyfont=\sffamily,%
 headformat=\color{defn-color}Définition~\NUMBER\hfill\NOTE\smallskip\linebreak,%
 preheadhook=\vspace{\freespace}\begin{defn-leftbar},%
 postfoothook=\end{defn-leftbar},%
]{better-defn}
\declaretheoremstyle[headfont=\sffamily\bfseries,%
 notefont=\sffamily\bfseries,%
 notebraces={}{},%
 headpunct=,%
 bodyfont=\sffamily,%
 headformat=\color{warn-color}Attention\hfill\NOTE\smallskip\linebreak,%
 preheadhook=\vspace{\freespace}\begin{warn-leftbar},%
 postfoothook=\end{warn-leftbar},%
]{better-warn}
\declaretheoremstyle[headfont=\sffamily\bfseries,%
 notefont=\sffamily\bfseries,%
notebraces={}{},%
headpunct=,%
 bodyfont=\sffamily,%
 headformat=\color{exemple-color}Exemple~\NUMBER\hfill\NOTE\smallskip\linebreak,%
 preheadhook=\vspace{\freespace}\begin{exemple-leftbar},%
 postfoothook=\end{exemple-leftbar},%
]{better-exemple}
\declaretheoremstyle[headfont=\sffamily\bfseries,%
 notefont=\sffamily\bfseries,%
 notebraces={}{},%
 headpunct=,%
 bodyfont=\sffamily,%
 headformat=\color{props-color}Proposition~\NUMBER\hfill\NOTE\smallskip\linebreak,%
 preheadhook=\vspace{\freespace}\begin{props-leftbar},%
 postfoothook=\end{props-leftbar},%
]{better-props}
\declaretheoremstyle[headfont=\sffamily\bfseries,%
 notefont=\sffamily\bfseries,%
 notebraces={}{},%
 headpunct=,%
 bodyfont=\sffamily,%
 headformat=\color{props-color}Théorème~\NUMBER\hfill\NOTE\smallskip\linebreak,%
 preheadhook=\vspace{\freespace}\begin{props-leftbar},%
 postfoothook=\end{props-leftbar},%
]{better-thm}
\declaretheoremstyle[headfont=\sffamily\bfseries,%
 notefont=\sffamily\bfseries,%
 notebraces={}{},%
 headpunct=,%
 bodyfont=\sffamily,%
 headformat=\color{corol-color}Corollaire~\NUMBER\hfill\NOTE\smallskip\linebreak,%
 preheadhook=\vspace{\freespace}\begin{corol-leftbar},%
 postfoothook=\end{corol-leftbar},%
]{better-corol}

\declaretheorem[style=better-defn]{defn}
\declaretheorem[style=better-warn]{warn}
\declaretheorem[style=better-exemple]{exemple}
\declaretheorem[style=better-corol]{corol}
\declaretheorem[style=better-props, numberwithin=defn]{props}
\declaretheorem[style=better-thm, sibling=props]{thm}
\newtheorem*{lemme}{Lemme}%[subsection]
%\newtheorem{props}{Propriétés}[defn]

\newenvironment{system}%
{\left\lbrace\begin{align}}%
{\end{align}\right.}

\newenvironment{AQT}{{\fontfamily{qbk}\selectfont AQT}}

\usepackage{LobsterTwo}
\titleformat{\section}{\newpage\LobsterTwo \huge\bfseries}{\thesection.}{1em}{}
\titleformat{\subsection}{\vspace{2em}\LobsterTwo \Large\bfseries}{\thesubsection.}{1em}{}
\titleformat{\subsubsection}{\vspace{1em}\LobsterTwo \large\bfseries}{\thesubsubsection.}{1em}{}

\newenvironment{lititle}%
{\vspace{7mm}\LobsterTwo \large}%
{\\}

\renewenvironment{proof}{$\square$ \footnotesize\textit{Démonstration.}}{\begin{flushright}$\blacksquare$\end{flushright}}

\title{Fonctions à deux variables}
\author{William Hergès\thanks{Sorbonne Université - Faculté des Sciences, Faculté des Lettres}}

\begin{document}
	\maketitle
	\tableofcontents
	\newpage
	\section{Fonctions, graphes et courbes de niveau}
	Dans ce chapitre, nous n'allons traiter que les fonctions à deux variables.
	\begin{defn}
		Une fonction $f$ de $D\subset\mathbb{R}^2$ dans $E\subset\mathbb{R}$ est définie telle que :
		$$(x,y)\longmapsto f(x,y)$$
		On appelle ce type de fonction une fonction à deux variables.
	\end{defn}
	\begin{defn}
		Le graphe de $f$ une fonction à deux variables l'ensemble des points
		$$ \Gamma_f = \{((x,y),z)|(x,y)\in D,z=f(x,y)\} $$
	\end{defn}
	On peut avoir que $f$ ne dépend qu'une des deux variables, $x$ par exemple. On a alors que son graphe ne dépend pas de $y$, i.e.
	$$ \forall (y,y')\in I\subset\mathbb{R}, f(x,y) = f(x,y') $$
	\begin{defn}
		On définit $C_t$ tel que :
		$$C_t = \{(x,y)\in\mathbb{R}^2|f(x,y)=t\}$$
		$C_t$ est une courbe de niveau.
	\end{defn}
	\section{Dérivée partielle}
	\begin{defn}
		Soit $f$ une fonction de $D_1\times D_2$ dans $I$.

		On note $f_{y_0}$ la fonction de $D_1$ dans $I$ tel que :
		$$ \forall s\in D_1,\quad f_{y_0} = f(s,y_0) $$

		\textit{Mutadis mutandis} pour $f_{x_0}$.
	\end{defn}
	\begin{defn}
		La dérivée partielle de $f$ par rapport à $x$ (resp. $y$) est la dérivée de $f_{y_0}$ (resp. $f_{x_0}$). On la note :
		$$ \frac{\partial f}{\partial x} = f'_{y_0} $$
	\end{defn}
	\begin{thm}
		On a :
		\begin{align*}
			\frac{\partial}{\partial x}\left( \frac{\partial f}{\partial y} \right) &= \frac{\partial^2 f}{\partial x\partial y} \\
			\frac{\partial}{\partial y}\left( \frac{\partial f}{\partial x} \right) &= \frac{\partial^2 f}{\partial y\partial x}
		\end{align*}
		Ce qui est la même chose ! Ainsi :
		$$ \frac{\partial^2 f}{\partial x\partial y} = \frac{\partial^2 f}{\partial y\partial x}$$
	\end{thm}
	\begin{lititle}
		Plan tangent
	\end{lititle}
	L'équation du plan tangent par $f$ est $(x_0,y_0)$ :
	$$ z = f(x_0,y_0)+\frac{\partial f}{\partial x}(x-x_0)+\frac{\partial f}{\partial y}(y-y_0) $$
	\begin{defn}
		On pose $\nabla f$ le gradient de $f:D_1\times D_2\to I$ (où $D_1\times D_2\subset\mathbb{R}^2$ et $I\subset\mathbb{R}$) tel que :
		$$ \forall (x,y)\in D_1\times D_2,\quad\nabla f (x,y) = \begin{pmatrix} \frac{\partial f}{\partial x} (x,y)\\ \frac{\partial f}{\partial y} (x,y)\end{pmatrix}  $$
	\end{defn}
	\fbox{
		\begin{minipage}{\dimexpr\textwidth-2\fboxsep-2\fboxrule\relax}
			\begin{lititle}
				\centering Notations de Monge
			\end{lititle}
			On note $$p=\frac{\partial f}{\partial x}\quad;\quad q=\frac{\partial f}{\partial y}$$
			On note $$ r=\frac{\partial^2 f}{\partial^2 x}\quad;\quad s = \frac{\partial^2 f}{\partial x\partial y}\quad;\quad t = \frac{\partial^2 f}{\partial^2 y} $$	
		\end{minipage}
	}
	\section{Extrémum}
	\begin{thm}
		$f$ possède un extremum en $(x_0,y_0)$ implique que :
		$$ \frac{\partial f}{\partial x}(x_0,y_0) = \frac{\partial f}{\partial y}(x_0,y_0) = 0$$
	\end{thm}
	\begin{thm}
		Soient $f$ une fonction à deux variables et $(x_0,y_0)$ est un point critique. \\
		On pose $$ D = rt-s^2$$
		avec $r,t,s$ les notations de Monge.

		\begin{itemize}
			\item Si $D>0$, alors $(x_0,y_0)$ est un extrémum. Il s'agit d'un maximum si $r<0$ ou d'un minimum si $r>0$.
			\item Si $D<0$, alors $(x_0,y_0)$ n'est pas un extrémum.
			\item Si $D=0$, alors tout est possible.
		\end{itemize}
	\end{thm}
	\begin{props}
		Soit $f:D\to I$ de classe $\mathcal{C}^1$. \\
		Soit $c\in I$ et $f^{-1}(c)$ un ensemble de niveau de $f$.\\
		Soit $X: I\to D$ telle que $X(t)\in f^{-1}(c)$ pour tout $t\in I$.

		Alors :
		$$ X'(t)\nabla f(X(t)) = 0 $$
		pour tout $t\in I$.
	\end{props}
	\begin{proof}
		\AQT
	\end{proof}
\end{document}
