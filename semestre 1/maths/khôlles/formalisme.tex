%%=====================================================================================
%%
%%       Filename:  formalisme.tex
%%
%%    Description:  
%%
%%        Version:  1.0
%%        Created:  06/12/2024
%%       Revision:  none
%%
%%         Author:  YOUR NAME (), 
%%   Organization:  
%%      Copyright:  Copyright (c) 2024, YOUR NAME
%%
%%          Notes:  
%%
%%=====================================================================================
\documentclass[a4paper, titlepage]{article}

\usepackage[utf8]{inputenc}
\usepackage[T1]{fontenc}
\usepackage{textcomp}
\usepackage[french]{babel}
\usepackage{amsmath, amssymb}
\usepackage{amsthm}
\usepackage[svgnames]{xcolor}
\usepackage{thmtools}
\usepackage{lipsum}
\usepackage{framed}
\usepackage{parskip}
\usepackage{titlesec}

\renewcommand{\familydefault}{\sfdefault}

% figure support
\usepackage{import}
\usepackage{xifthen}
\pdfminorversion=7
\usepackage{pdfpages}
\usepackage{transparent}
\newcommand{\incfig}[1]{%
	\def\svgwidth{\columnwidth}
	\import{./figures/}{#1.pdf_tex}
}

\pdfsuppresswarningpagegroup=1

\colorlet{defn-color}{DarkBlue}
\colorlet{props-color}{Blue}
\colorlet{warn-color}{Red}
\colorlet{exemple-color}{Green}
\colorlet{corol-color}{Orange}
\newenvironment{defn-leftbar}{%
  \def\FrameCommand{{\color{defn-color}\vrule width 3pt} \hspace{10pt}}%
  \MakeFramed {\advance\hsize-\width \FrameRestore}}%
 {\endMakeFramed}
\newenvironment{warn-leftbar}{%
  \def\FrameCommand{{\color{warn-color}\vrule width 3pt} \hspace{10pt}}%
  \MakeFramed {\advance\hsize-\width \FrameRestore}}%
 {\endMakeFramed}
\newenvironment{exemple-leftbar}{%
  \def\FrameCommand{{\color{exemple-color}\vrule width 3pt} \hspace{10pt}}%
  \MakeFramed {\advance\hsize-\width \FrameRestore}}%
 {\endMakeFramed}
\newenvironment{props-leftbar}{%
  \def\FrameCommand{{\color{props-color}\vrule width 3pt} \hspace{10pt}}%
  \MakeFramed {\advance\hsize-\width \FrameRestore}}%
 {\endMakeFramed}
\newenvironment{corol-leftbar}{%
  \def\FrameCommand{{\color{corol-color}\vrule width 3pt} \hspace{10pt}}%
  \MakeFramed {\advance\hsize-\width \FrameRestore}}%
 {\endMakeFramed}

\def \freespace {1em}
\declaretheoremstyle[headfont=\sffamily\bfseries,%
 notefont=\sffamily\bfseries,%
 notebraces={}{},%
 headpunct=,%
 bodyfont=\sffamily,%
 headformat=\color{defn-color}Définition~\NUMBER\hfill\NOTE\smallskip\linebreak,%
 preheadhook=\vspace{\freespace}\begin{defn-leftbar},%
 postfoothook=\end{defn-leftbar},%
]{better-defn}
\declaretheoremstyle[headfont=\sffamily\bfseries,%
 notefont=\sffamily\bfseries,%
 notebraces={}{},%
 headpunct=,%
 bodyfont=\sffamily,%
 headformat=\color{warn-color}Attention~\NUMBER\hfill\NOTE\smallskip\linebreak,%
 preheadhook=\vspace{\freespace}\begin{warn-leftbar},%
 postfoothook=\end{warn-leftbar},%
]{better-warn}
\declaretheoremstyle[headfont=\sffamily\bfseries,%
 notefont=\sffamily\bfseries,%
notebraces={}{},%
headpunct=,%
 bodyfont=\sffamily,%
 headformat=\color{exemple-color}Exemple~\NUMBER\hfill\NOTE\smallskip\linebreak,%
 preheadhook=\vspace{\freespace}\begin{exemple-leftbar},%
 postfoothook=\end{exemple-leftbar},%
]{better-exemple}
\declaretheoremstyle[headfont=\sffamily\bfseries,%
 notefont=\sffamily\bfseries,%
 notebraces={}{},%
 headpunct=,%
 bodyfont=\sffamily,%
 headformat=\color{props-color}Proposition~\NUMBER\hfill\NOTE\smallskip\linebreak,%
 preheadhook=\vspace{\freespace}\begin{props-leftbar},%
 postfoothook=\end{props-leftbar},%
]{better-props}
\declaretheoremstyle[headfont=\sffamily\bfseries,%
 notefont=\sffamily\bfseries,%
 notebraces={}{},%
 headpunct=,%
 bodyfont=\sffamily,%
 headformat=\color{props-color}Théorème~\NUMBER\hfill\NOTE\smallskip\linebreak,%
 preheadhook=\vspace{\freespace}\begin{props-leftbar},%
 postfoothook=\end{props-leftbar},%
]{better-thm}
\declaretheoremstyle[headfont=\sffamily\bfseries,%
 notefont=\sffamily\bfseries,%
 notebraces={}{},%
 headpunct=,%
 bodyfont=\sffamily,%
 headformat=\color{corol-color}Corollaire~\NUMBER\hfill\NOTE\smallskip\linebreak,%
 preheadhook=\vspace{\freespace}\begin{corol-leftbar},%
 postfoothook=\end{corol-leftbar},%
]{better-corol}

\declaretheorem[style=better-defn]{defn}
\declaretheorem[style=better-warn]{warn}
\declaretheorem[style=better-exemple]{exemple}
\declaretheorem[style=better-corol]{corol}
\declaretheorem[style=better-props, numberwithin=defn]{props}
\declaretheorem[style=better-thm, sibling=props]{thm}
\newtheorem*{lemme}{Lemme}%[subsection]
%\newtheorem{props}{Propriétés}[defn]

\newenvironment{system}%
{\left\lbrace\begin{align}}%
{\end{align}\right.}

\newenvironment{AQT}{{\fontfamily{qbk}\selectfont AQT}}

\usepackage{LobsterTwo}
\titleformat{\section}{\newpage\LobsterTwo \huge\bfseries}{\thesection.}{1em}{}
\titleformat{\subsection}{\vspace{2em}\LobsterTwo \Large\bfseries}{\thesubsection.}{1em}{}
\titleformat{\subsubsection}{\vspace{1em}\LobsterTwo \large\bfseries}{\thesubsubsection.}{1em}{}

\newenvironment{lititle}%
{\vspace{7mm}\LobsterTwo \large}%
{\\}

\renewenvironment{proof}{$\square$ \footnotesize\textit{Démonstration.}}{\begin{flushright}$\blacksquare$\end{flushright}}

\title{Khôlle 1 - Un peu de formalisme}
\author{William Hergès\thanks{Sorbonne Universite}}

\begin{document}
	\maketitle
	\tableofcontents
	\newpage
	\section{Un peu de formalisme}
	Durant cette khôlle (ou colle, mais je préfère les mots pseudo-latin), je vais demander une rédaction particulièrement rigoureuse. Ce formalisme est ensentiel pour démontrer formalement des propositions et des théorèmes.
	\subsection{Assertion, proposition et théorème}
	Une assertion est un énoncé vrai ou faux, e.g. $\pi$ est un irrationnel (démonstration complexe mais faisable en fin de Terminal~: devoir maison des MPSI d'Henri-IV durant les vacances d'été précédent leur première année et devoir sur table des MP2I/MPSI à Saint-Louis vers janvier).

	Une énoncé indiquant la vérité d'une assertion est une proposition ou un théorème. Ce dernier est juste une proposition importante.

	Une propriété est une assertion détaillant les éléments fondamentaux découlant d'une définition. Toutes sommes de fonctions continues est continue est une propriété et une proposition.
	\subsection{Quantification}
	Tous les éléments doivent être quantifiés à l'aide de quantifieurs ($\forall$, $\exists$, $\exists!$, «~Soit~», etc.). Un élément non quantité est une faute de rigueur car noous ne savons pas dans quelle condition nous pouvons l'utiliser.

	Par exemple, nous n'écrirons pas $$f(x)=2x\cos(x)$$ car nous ne quantifions pas $x$ ici. Nous écrirons plutôt : $$ \forall x\in D,\quad f(x)=2x\cos(x) $$ où $D$ est un ensemble sur lequel $f$ est définie.
	\section{Logique}
	On admet la proposition suivante :
	$$ \exists x\in\varnothing,\quad P $$
	est toujours fausse pour toute assertion $P$.

	Démontrer que $$ \forall x\in\varnothing, P $$ où $P$ une assertion, est toujours vraie.
	\section{Rigueur en démonstration}
	Soit $(u_n)_{n\geqslant 0}$ telle que $u_1 = 1$ et :
	$$ \forall n\in\mathbb{N}^*,\quad u_{2n} = 2u_n\quad\land\quad \forall n\in\mathbb{N}^*,\quad u_{2n+1}=u_{n+1}+u_n $$
	Démontrer que pour tout naturel $n$ non nul, $u_n = n$
	\section{Intégration}
	Démontrer rigoureusement l'intégration par partie (et la primitivisation par partie).
	\section{Limite suite}
	Calculer le $\mathrm{DL}_0(3)$ de $x\longmapsto e^{\cos x}$.
	\section{Newton}
	Démontrer le binôme de Newton, i.e. pour tout $(a,b)\in\mathbb{C}^2$ et pour tout $n\in\mathbb{N}^*$ on a :
	$$ (a+b)^n = \sum_{k=0}^{n} \begin{pmatrix} n\\k \end{pmatrix} a^{k}b^{n-k} $$
\end{document}
