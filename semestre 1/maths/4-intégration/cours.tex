%%=====================================================================================
%%
%%       Filename:  cours.tex
%%
%%    Description:  
%%
%%        Version:  1.0
%%        Created:  03/06/2024
%%       Revision:  none
%%
%%         Author:  YOUR NAME (), 
%%   Organization:  
%%      Copyright:  Copyright (c) 2024, YOUR NAME
%%
%%          Notes:  
%%
%%=====================================================================================
\documentclass[a4paper, titlepage]{article}

\usepackage[utf8]{inputenc}
\usepackage[T1]{fontenc}
\usepackage{textcomp}
\usepackage[french]{babel}
\usepackage{amsmath, amssymb}
\usepackage{amsthm}
\usepackage[svgnames]{xcolor}
\usepackage{thmtools}
\usepackage{lipsum}
\usepackage{framed}
\usepackage{parskip}
\usepackage{titlesec}

\renewcommand{\familydefault}{\sfdefault}

% figure support
\usepackage{import}
\usepackage{xifthen}
\pdfminorversion=7
\usepackage{pdfpages}
\usepackage{transparent}
\newcommand{\incfig}[1]{%
	\def\svgwidth{\columnwidth}
	\import{./figures/}{#1.pdf_tex}
}

\pdfsuppresswarningpagegroup=1

\colorlet{defn-color}{DarkBlue}
\colorlet{props-color}{Blue}
\colorlet{warn-color}{Red}
\colorlet{exemple-color}{Green}
\colorlet{corol-color}{Orange}
\newenvironment{defn-leftbar}{%
  \def\FrameCommand{{\color{defn-color}\vrule width 3pt} \hspace{10pt}}%
  \MakeFramed {\advance\hsize-\width \FrameRestore}}%
 {\endMakeFramed}
\newenvironment{warn-leftbar}{%
  \def\FrameCommand{{\color{warn-color}\vrule width 3pt} \hspace{10pt}}%
  \MakeFramed {\advance\hsize-\width \FrameRestore}}%
 {\endMakeFramed}
\newenvironment{exemple-leftbar}{%
  \def\FrameCommand{{\color{exemple-color}\vrule width 3pt} \hspace{10pt}}%
  \MakeFramed {\advance\hsize-\width \FrameRestore}}%
 {\endMakeFramed}
\newenvironment{props-leftbar}{%
  \def\FrameCommand{{\color{props-color}\vrule width 3pt} \hspace{10pt}}%
  \MakeFramed {\advance\hsize-\width \FrameRestore}}%
 {\endMakeFramed}
\newenvironment{corol-leftbar}{%
  \def\FrameCommand{{\color{corol-color}\vrule width 3pt} \hspace{10pt}}%
  \MakeFramed {\advance\hsize-\width \FrameRestore}}%
 {\endMakeFramed}

\def \freespace {1em}
\declaretheoremstyle[headfont=\sffamily\bfseries,%
 notefont=\sffamily\bfseries,%
 notebraces={}{},%
 headpunct=,%
 bodyfont=\sffamily,%
 headformat=\color{defn-color}Définition~\NUMBER\hfill\NOTE\smallskip\linebreak,%
 preheadhook=\vspace{\freespace}\begin{defn-leftbar},%
 postfoothook=\end{defn-leftbar},%
]{better-defn}
\declaretheoremstyle[headfont=\sffamily\bfseries,%
 notefont=\sffamily\bfseries,%
 notebraces={}{},%
 headpunct=,%
 bodyfont=\sffamily,%
 headformat=\color{warn-color}Attention\hfill\NOTE\smallskip\linebreak,%
 preheadhook=\vspace{\freespace}\begin{warn-leftbar},%
 postfoothook=\end{warn-leftbar},%
]{better-warn}
\declaretheoremstyle[headfont=\sffamily\bfseries,%
 notefont=\sffamily\bfseries,%
notebraces={}{},%
headpunct=,%
 bodyfont=\sffamily,%
 headformat=\color{exemple-color}Exemple~\NUMBER\hfill\NOTE\smallskip\linebreak,%
 preheadhook=\vspace{\freespace}\begin{exemple-leftbar},%
 postfoothook=\end{exemple-leftbar},%
]{better-exemple}
\declaretheoremstyle[headfont=\sffamily\bfseries,%
 notefont=\sffamily\bfseries,%
 notebraces={}{},%
 headpunct=,%
 bodyfont=\sffamily,%
 headformat=\color{props-color}Proposition~\NUMBER\hfill\NOTE\smallskip\linebreak,%
 preheadhook=\vspace{\freespace}\begin{props-leftbar},%
 postfoothook=\end{props-leftbar},%
]{better-props}
\declaretheoremstyle[headfont=\sffamily\bfseries,%
 notefont=\sffamily\bfseries,%
 notebraces={}{},%
 headpunct=,%
 bodyfont=\sffamily,%
 headformat=\color{props-color}Théorème~\NUMBER\hfill\NOTE\smallskip\linebreak,%
 preheadhook=\vspace{\freespace}\begin{props-leftbar},%
 postfoothook=\end{props-leftbar},%
]{better-thm}
\declaretheoremstyle[headfont=\sffamily\bfseries,%
 notefont=\sffamily\bfseries,%
 notebraces={}{},%
 headpunct=,%
 bodyfont=\sffamily,%
 headformat=\color{corol-color}Corollaire~\NUMBER\hfill\NOTE\smallskip\linebreak,%
 preheadhook=\vspace{\freespace}\begin{corol-leftbar},%
 postfoothook=\end{corol-leftbar},%
]{better-corol}

\declaretheorem[style=better-defn]{defn}
\declaretheorem[style=better-warn]{warn}
\declaretheorem[style=better-exemple]{exemple}
\declaretheorem[style=better-corol]{corol}
\declaretheorem[style=better-props, numberwithin=defn]{props}
\declaretheorem[style=better-thm, sibling=props]{thm}
\newtheorem*{lemme}{Lemme}%[subsection]
%\newtheorem{props}{Propriétés}[defn]

\newenvironment{system}%
{\left\lbrace\begin{align}}%
{\end{align}\right.}

\newenvironment{AQT}{{\fontfamily{qbk}\selectfont AQT}}

\usepackage{LobsterTwo}
\titleformat{\section}{\newpage\LobsterTwo \huge\bfseries}{\thesection.}{1em}{}
\titleformat{\subsection}{\vspace{2em}\LobsterTwo \Large\bfseries}{\thesubsection.}{1em}{}
\titleformat{\subsubsection}{\vspace{1em}\LobsterTwo \large\bfseries}{\thesubsubsection.}{1em}{}

\newenvironment{lititle}%
{\vspace{7mm}\LobsterTwo \large}%
{\\}

\renewenvironment{proof}{$\square$ \footnotesize\textit{Démonstration.}}{\begin{flushright}$\blacksquare$\end{flushright}}

\title{Intégration}
\author{William Hergès\thanks{Sorbonne Université - Faculté des Sciences, Faculté des Lettres}}

\begin{document}
	\maketitle
	\tableofcontents
	\newpage
	\section{Définitions et théorèmes fondamentaux}
	\begin{defn}
		L'intégration de la fonction $f: [a,b]\to \mathbb{R}$ est l'aire de la région sous la courbe de $f$ et l'axe des absisses. On note ce nombre $$ \int^b_af(x)\mathrm{d}x $$

		L'aire sous la courbe est :
		$$ \lim_{n \to \infty} \sum_{i=1}^{n-1} f\left( \frac{b-a}{i} \right) \frac{b-a}{n} $$
	\end{defn}

	Ce calcul est apparenté au calcul des dérivés.
	\begin{defn}
		Soit $f:I\to \mathbb{R}$ avec $I$ un intervalle, on dit que $F:I\to \mathbb{R}$ est une primitive de $f$, i.e. $$ F'=f $$
	\end{defn}
	$F$ est toujours déterminée à une constante près.
	\begin{thm}
		Soit $f$ une fonction continue de $[a,b]$ dans $\mathbb{R}$.

		Il existe toujours une primitive $F$ de $f$ sur $[a,b]$.\\
		On a alors :
		$$ \int^b_a f(t)\mathrm{d}t = F(b)-F(a) = [F(x)]_a^b $$
	\end{thm}
	\begin{proof}
		Admis.
	\end{proof}
	Ce calcul ne dépend pas de la constance de $F$.
	\section{Calculs}
	\subsection{Propriétés}
	\begin{props}
		La propriété de Chasles est vraie pour les intégrales
	\end{props}
	\begin{proof}
		\AQT
	\end{proof}
	\begin{props}
		L'intégrale est linéaire
	\end{props}
	\begin{proof}
		\AQT
	\end{proof}
	\begin{props}
		Si $f\geqslant 0$, alors $\int f\geqslant 0$.

		Si $f \geqslant g$, alors $\int f \geqslant \int g$.
	\end{props}
	\begin{proof}
		\AQT
	\end{proof}
	\subsection{Intégration par partie}
	\begin{thm}[IPP]
		Soient $f,g$ deux fonctions intégrables sur $[a,b]$.

		Alors~:
		$$ \int_a^b f'(x)g(x)\mathrm{d}x = [(fg)(x)]^b_a-\int^b_af(x)g'(x)\mathrm{d}x $$
	\end{thm}
	\begin{proof}
		Admis.
	\end{proof}
	\begin{exemple}
		On a :
		$$ \int_a^b te^{-t}\mathrm{d}t = [-te^{-t}]^b_a-\int^a_be^{-t}\mathrm{d}t $$
		(en posant $f'(t)=e^{-t}$ et $g(t)=t$)
	\end{exemple}
	\subsection{Changement de variable}
	\begin{thm}
		Soit $f:[a,b]\to \mathbb{R}$ continue. 

		Pour toute fonction $\varphi:J\to [a;b]$ de classe $\mathcal{C}^1$ et tous $\alpha,\beta\in J$ tels que $\varphi(\alpha)=a$ et $\varphi(\beta)=b$ (sous réserve d'existence), on a : $$ \int^b_af(t)\mathrm{d}t = \int^{\beta}_{\alpha}f(\varphi(u))\varphi'(u)\mathrm{d}u $$
	\end{thm}
	Quand on a une intégrale trop compliqué, on fait un changement de variable.

	\begin{lititle}
		Méthode
	\end{lititle}
	On cherche à transformer $$ \int^b_a f(t)\mathrm{d}t $$ en $$ \int^{\beta}_{\alpha} g(u)\mathrm{d}u $$
	\begin{enumerate}
		\item On exprime $t$ en fonction de $u$ (i.e. il existe $\varphi$ tel que $t=\varphi(u)$).
		\item On cherche $\alpha,\beta$ tel que $\varphi(\alpha)=a$ et $\varphi(\beta)=b$.
		\item On vérifie que $\varphi$ est de classe $\mathcal{C}^1$ sur $[\alpha,\beta]$ et l’on calcule $\mathrm{d}t = \varphi'(u)\mathrm{d}u$, ce qui donne l’ancienne différentielle $\mathrm{d}t$ en fonction de la nouvelle $\mathrm{d}u$.
		\item On transforme l’intégrant $f(t) \mathrm{d}t$ en remplaçant $\mathrm{d}t$ par $\varphi'(u) \mathrm{d}u$ et $t$ par $\mathrm{d}(u)$. On obtient ainsi un nouvel intégrant de la forme $g(u) \mathrm{d}u$.
		\item On écrit la nouvelle intégrale et on obtient bien celle qu'on cherchait.
	\end{enumerate}
\end{document}
