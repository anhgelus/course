%%=====================================================================================
%%
%%       Filename:  cours.tex
%%
%%    Description:  
%%
%%        Version:  1.0
%%        Created:  03/06/2024
%%       Revision:  none
%%
%%         Author:  YOUR NAME (), 
%%   Organization:  
%%      Copyright:  Copyright (c) 2024, YOUR NAME
%%
%%          Notes:  
%%
%%=====================================================================================
\documentclass[a4paper, titlepage]{article}

\usepackage[utf8]{inputenc}
\usepackage[T1]{fontenc}
\usepackage{textcomp}
\usepackage[french]{babel}
\usepackage{amsmath, amssymb}
\usepackage{amsthm}
\usepackage[svgnames]{xcolor}
\usepackage{thmtools}
\usepackage{lipsum}
\usepackage{framed}
\usepackage{parskip}
\usepackage{titlesec}

\renewcommand{\familydefault}{\sfdefault}

% figure support
\usepackage{import}
\usepackage{xifthen}
\pdfminorversion=7
\usepackage{pdfpages}
\usepackage{transparent}
\newcommand{\incfig}[1]{%
	\def\svgwidth{\columnwidth}
	\import{./figures/}{#1.pdf_tex}
}

\pdfsuppresswarningpagegroup=1

\newenvironment{system}%
{\left\lbrace\begin{align}}%
{\end{align}\right.}

\newenvironment{AQT}{{\fontfamily{qbk}\selectfont AQT}}

\usepackage{LobsterTwo}
\titleformat{\section}{\LobsterTwo \huge\bfseries}{\thesection.}{1em}{}
\titleformat{\subsection}{\vspace{2em}\LobsterTwo \Large\bfseries}{\thesubsection.}{1em}{}
\titleformat{\subsubsection}{\vspace{1em}\LobsterTwo \large\bfseries}{\thesubsubsection.}{1em}{}

\newenvironment{lititle}%
{\vspace{7mm}\LobsterTwo \large}%
{\\}

\renewenvironment{proof}{$\square$ \footnotesize\textit{Démonstration.}}{\begin{flushright}$\blacksquare$\end{flushright}}

\title{Correction TD Complexes}
\author{William Hergès\thanks{Sorbonne Université - Faculté des Sciences, Faculté des Lettres}}

\begin{document}
	\maketitle
	\section*{Exercice 1 compléments}
	On a :
	\begin{align*}
		z_1 &= 5+12i \\
			&= (a+ib)^2 \\
			&= a^2-b^2+2ab
	\end{align*}
	Alors :
	\begin{align*}
		a^2-b^2 &= 5 \\
		2ab &= 12 \\
		a^2+b^2 &= |z_1| = 13
	\end{align*}
	En faisant $(1)+(3)$ et $(3)-(1)$, on obtient que :
	\begin{align*}
		a^2 &= 9 \\
		b^2 &= 4 \\
		ab &= 6
	\end{align*}
	Ainsi, $a$ et $b$ sont de même signe. Donc :
	$$ \{-3-2i;3+2i\} $$
	est l'ensemble solution de $z^2=\delta$.
	\section*{Exercice 2 compléments}
	On a :
	\begin{align*}
		z^3 &= 2-2i \\ 
			&= 2(1-i)\\
			&= 2\sqrt 2e^{\frac{i\pi}{4}}
	\end{align*}
	Donc :
	\begin{align*}
		z &= \sqrt 2\left( e^{\frac{i\pi}{4}} \right)^{\frac{1}{3}} \\
		  &= \sqrt 2\left( e^{\frac{i\pi}{4}+2k\pi} \right)^{\frac{1}{3}},\quad k\in\mathbb{Z} \\
		  &= \sqrt 2e^{\frac{i\pi}{12}+\frac{2}{3}k\pi},\quad k\in\mathbb{Z} \\
	\end{align*}
	Comme les angles sont périodiques sur $2k\pi$, on a que toutes les solutions sont couvertes par $k\in[|0;2|]$. Ainsi :
	$$ \left\{\sqrt 2e^{\frac{i\pi}{12}+\frac{2}{3}k\pi},k\in[|0;2|]\right\} $$
\end{document}
