%%=====================================================================================
%%
%%       Filename:  cours.tex
%%
%%    Description:  
%%
%%        Version:  1.0
%%        Created:  03/06/2024
%%       Revision:  none
%%
%%         Author:  YOUR NAME (), 
%%   Organization:  
%%      Copyright:  Copyright (c) 2024, YOUR NAME
%%
%%          Notes:  
%%
%%=====================================================================================
\documentclass[a4paper, titlepage]{article}

\usepackage[utf8]{inputenc}
\usepackage[T1]{fontenc}
\usepackage{textcomp}
\usepackage[french]{babel}
\usepackage{amsmath, amssymb}
\usepackage{amsthm}
\usepackage[svgnames]{xcolor}
\usepackage{thmtools}
\usepackage{lipsum}
\usepackage{framed}
\usepackage{parskip}
\usepackage{titlesec}

\renewcommand{\familydefault}{\sfdefault}

% figure support
\usepackage{import}
\usepackage{xifthen}
\pdfminorversion=7
\usepackage{pdfpages}
\usepackage{transparent}
\newcommand{\incfig}[1]{%
	\def\svgwidth{\columnwidth}
	\import{./figures/}{#1.pdf_tex}
}

\pdfsuppresswarningpagegroup=1

\newenvironment{system}%
{\left\lbrace\begin{align}}%
{\end{align}\right.}

\newenvironment{AQT}{{\fontfamily{qbk}\selectfont AQT}}

\usepackage{LobsterTwo}
\titleformat{\section}{\LobsterTwo \huge\bfseries}{\thesection.}{1em}{}
\titleformat{\subsection}{\vspace{2em}\LobsterTwo \Large\bfseries}{\thesubsection.}{1em}{}
\titleformat{\subsubsection}{\vspace{1em}\LobsterTwo \large\bfseries}{\thesubsubsection.}{1em}{}

\newenvironment{lititle}%
{\vspace{7mm}\LobsterTwo \large}%
{\\}

\renewenvironment{proof}{$\square$ \footnotesize\textit{Démonstration.}}{\begin{flushright}$\blacksquare$\end{flushright}}

\title{Correction TD 1}
\author{William Hergès\thanks{Sorbonne Université - Faculté des Sciences, Faculté des Lettres}}

\begin{document}
	\maketitle
	\section*{Exercice 1}
	\begin{enumerate}
		\item On a : 
			\begin{table}[htpb]
				\centering
				\caption{Angles remarquables}
				\label{tab:trigo}
				\begin{tabular}{|c|c|c|c|c|c|}
					\hline
					$\alpha$ & 0 & $\frac{\pi}{6}$ & $\frac{\pi}{4}$ & $\frac{\pi}{3}$ & $\frac{\pi}{2}$\\
				\hline
					$\cos\alpha$ & 1 & $\frac{\sqrt 3}{2}$ & $\frac{\sqrt 2}{2}$ & $\frac{1}{2}$ & 0 \\
					\hline
					$\sin\alpha$ & 0 & $\frac{1}{2}$ & $\frac{\sqrt 2}{2}$ & $\frac{\sqrt 3}{2}$ & 1 \\
					\hline
				\end{tabular}
			\end{table}
		\item Pour tout $x\in\mathbb{R}$, on a :
			\begin{align*}
				\cos\left( x+\frac{\pi}{2} \right) &= -\sin x \\
				\cos\left( x+\pi \right) &= -\cos x \\
				\cos\left( x+\frac{3\pi}{2} \right) &= \sin x \\
				\sin\left( x+\frac{\pi}{2} \right) &= \cos x \\
				\sin\left( x+\pi \right) &= -\sin x \\
				\sin\left( x+\frac{3\pi}{2} \right) &= -\cos x \\
			\end{align*}
		\item \AQT
	\end{enumerate}
	\section*{Exercice 2}
	\AQT
	\section*{Exercice 3}
	Tout ce que j'ai fait est bon.

	Soit $M$ un point de la droit $D$ de vecteur directeur $\vec u$ passant par $A$. On a donc :
	$$ \overrightarrow{AM} = \alpha\vec u\quad(\alpha\in\mathbb{R}) $$
	Si on note $(x,y)\in\mathbb{R}^2$ les coordonnées de $M$ et $(a,b)\in\mathbb{R}^2$ les coordonnées de $A$ dans la base canonique, alors on a :
	$$ \begin{pmatrix} x-a\\y-b \end{pmatrix} = \alpha\vec u\quad(\alpha\in\mathbb{R}) $$
	Ce qui nous donne l'équation paramétrique :
	$$ \begin{pmatrix} x\\y \end{pmatrix} = \begin{pmatrix} a\\b \end{pmatrix}+t\vec u\quad(\text{où}~t\in\mathbb{R}) $$
	Pour tous les points $M\in D$, on a :
	$$ \overrightarrow{AM}\cdot\vec v = 0 $$
	(où $v$ est un vecteur normal)\\
	L'équation cartésienne est donc une relation liant $x$ et $y$ satisfaisant la relation précédente. On peut l'obtenir en résolvant l'équation paramétrique ou en calculant le produit scalaire entre tous points $M\in D$ et un vecteur normal de $D$.
\end{document}
