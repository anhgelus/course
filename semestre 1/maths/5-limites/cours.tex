%%=====================================================================================
%%
%%       Filename:  cours.tex
%%
%%    Description:  
%%
%%        Version:  1.0
%%        Created:  03/06/2024
%%       Revision:  none
%%
%%         Author:  YOUR NAME (), 
%%   Organization:  
%%      Copyright:  Copyright (c) 2024, YOUR NAME
%%
%%          Notes:  
%%
%%=====================================================================================
\documentclass[a4paper, titlepage]{article}

\usepackage[utf8]{inputenc}
\usepackage[T1]{fontenc}
\usepackage{textcomp}
\usepackage[french]{babel}
\usepackage{amsmath, amssymb}
\usepackage{amsthm}
\usepackage[svgnames]{xcolor}
\usepackage{thmtools}
\usepackage{lipsum}
\usepackage{framed}
\usepackage{parskip}
\usepackage{titlesec}

\renewcommand{\familydefault}{\sfdefault}

% figure support
\usepackage{import}
\usepackage{xifthen}
\pdfminorversion=7
\usepackage{pdfpages}
\usepackage{transparent}
\newcommand{\incfig}[1]{%
	\def\svgwidth{\columnwidth}
	\import{./figures/}{#1.pdf_tex}
}

\pdfsuppresswarningpagegroup=1

\colorlet{defn-color}{DarkBlue}
\colorlet{props-color}{Blue}
\colorlet{warn-color}{Red}
\colorlet{exemple-color}{Green}
\colorlet{corol-color}{Orange}
\newenvironment{defn-leftbar}{%
  \def\FrameCommand{{\color{defn-color}\vrule width 3pt} \hspace{10pt}}%
  \MakeFramed {\advance\hsize-\width \FrameRestore}}%
 {\endMakeFramed}
\newenvironment{warn-leftbar}{%
  \def\FrameCommand{{\color{warn-color}\vrule width 3pt} \hspace{10pt}}%
  \MakeFramed {\advance\hsize-\width \FrameRestore}}%
 {\endMakeFramed}
\newenvironment{exemple-leftbar}{%
  \def\FrameCommand{{\color{exemple-color}\vrule width 3pt} \hspace{10pt}}%
  \MakeFramed {\advance\hsize-\width \FrameRestore}}%
 {\endMakeFramed}
\newenvironment{props-leftbar}{%
  \def\FrameCommand{{\color{props-color}\vrule width 3pt} \hspace{10pt}}%
  \MakeFramed {\advance\hsize-\width \FrameRestore}}%
 {\endMakeFramed}
\newenvironment{corol-leftbar}{%
  \def\FrameCommand{{\color{corol-color}\vrule width 3pt} \hspace{10pt}}%
  \MakeFramed {\advance\hsize-\width \FrameRestore}}%
 {\endMakeFramed}

\def \freespace {1em}
\declaretheoremstyle[headfont=\sffamily\bfseries,%
 notefont=\sffamily\bfseries,%
 notebraces={}{},%
 headpunct=,%
 bodyfont=\sffamily,%
 headformat=\color{defn-color}Définition~\NUMBER\hfill\NOTE\smallskip\linebreak,%
 preheadhook=\vspace{\freespace}\begin{defn-leftbar},%
 postfoothook=\end{defn-leftbar},%
]{better-defn}
\declaretheoremstyle[headfont=\sffamily\bfseries,%
 notefont=\sffamily\bfseries,%
 notebraces={}{},%
 headpunct=,%
 bodyfont=\sffamily,%
 headformat=\color{warn-color}Attention\hfill\NOTE\smallskip\linebreak,%
 preheadhook=\vspace{\freespace}\begin{warn-leftbar},%
 postfoothook=\end{warn-leftbar},%
]{better-warn}
\declaretheoremstyle[headfont=\sffamily\bfseries,%
 notefont=\sffamily\bfseries,%
notebraces={}{},%
headpunct=,%
 bodyfont=\sffamily,%
 headformat=\color{exemple-color}Exemple~\NUMBER\hfill\NOTE\smallskip\linebreak,%
 preheadhook=\vspace{\freespace}\begin{exemple-leftbar},%
 postfoothook=\end{exemple-leftbar},%
]{better-exemple}
\declaretheoremstyle[headfont=\sffamily\bfseries,%
 notefont=\sffamily\bfseries,%
 notebraces={}{},%
 headpunct=,%
 bodyfont=\sffamily,%
 headformat=\color{props-color}Proposition~\NUMBER\hfill\NOTE\smallskip\linebreak,%
 preheadhook=\vspace{\freespace}\begin{props-leftbar},%
 postfoothook=\end{props-leftbar},%
]{better-props}
\declaretheoremstyle[headfont=\sffamily\bfseries,%
 notefont=\sffamily\bfseries,%
 notebraces={}{},%
 headpunct=,%
 bodyfont=\sffamily,%
 headformat=\color{props-color}Théorème~\NUMBER\hfill\NOTE\smallskip\linebreak,%
 preheadhook=\vspace{\freespace}\begin{props-leftbar},%
 postfoothook=\end{props-leftbar},%
]{better-thm}
\declaretheoremstyle[headfont=\sffamily\bfseries,%
 notefont=\sffamily\bfseries,%
 notebraces={}{},%
 headpunct=,%
 bodyfont=\sffamily,%
 headformat=\color{corol-color}Corollaire~\NUMBER\hfill\NOTE\smallskip\linebreak,%
 preheadhook=\vspace{\freespace}\begin{corol-leftbar},%
 postfoothook=\end{corol-leftbar},%
]{better-corol}

\declaretheorem[style=better-defn]{defn}
\declaretheorem[style=better-warn]{warn}
\declaretheorem[style=better-exemple]{exemple}
\declaretheorem[style=better-corol]{corol}
\declaretheorem[style=better-props, numberwithin=defn]{props}
\declaretheorem[style=better-thm, sibling=props]{thm}
\newtheorem*{lemme}{Lemme}%[subsection]
%\newtheorem{props}{Propriétés}[defn]

\newenvironment{system}%
{\left\lbrace\begin{align}}%
{\end{align}\right.}

\newenvironment{AQT}{{\fontfamily{qbk}\selectfont AQT}}

\usepackage{LobsterTwo}
\titleformat{\section}{\newpage\LobsterTwo \huge\bfseries}{\thesection.}{1em}{}
\titleformat{\subsection}{\vspace{2em}\LobsterTwo \Large\bfseries}{\thesubsection.}{1em}{}
\titleformat{\subsubsection}{\vspace{1em}\LobsterTwo \large\bfseries}{\thesubsubsection.}{1em}{}

\newenvironment{lititle}%
{\vspace{7mm}\LobsterTwo \large}%
{\\}

\renewenvironment{proof}{$\square$ \footnotesize\textit{Démonstration.}}{\begin{flushright}$\blacksquare$\end{flushright}}

\title{Limites}
\author{William Hergès\thanks{Sorbonne Université - Faculté des Sciences, Faculté des Lettres}}

\begin{document}
	\maketitle
	\tableofcontents
	\newpage
	\section{Classe d'une fonction}
	\begin{defn}
		Une fonction est de classe $\mathcal{C}^n$ (où $n\in\mathbb{N}^*$) si et seulement si sa dérivée $n$-ième est continue.\\
		Une fonction de classe $\mathcal{C}^0$ ne possède pas de dérivée continue.\\
		Une fonction est de classe $\mathcal{C}^{\infty}$ si et seulement si elle est dérivable une infinité de fois et que cette dériviée est continue.
	\end{defn}
	\begin{thm}[Théorème des accroissements finis]
		Soit $f:[a,b]\to \mathbb{R}$ de classe $\mathcal{C}^1_{[a,b]}$.

		Il existe $c\in]a,b[$ tel que :
		$$ f'(x) = \frac{f(b)-f(a)}{b-a} $$
	\end{thm}
	\begin{thm}[Inégalité des accroissements finis]
		Soit $f:[a,b]\to \mathbb{R}$ de classe $\mathcal{C}^1_{[a,b]}$.

		S'il existe $M\in\mathbb{R}_+$ tel que :
		$$ \forall x\in]a,b[,\quad |f'(x)|\leqslant M $$
		alors
		$$ |f(b)-f(a)|\leqslant M(b-a) $$
	\end{thm}
	\section{Comparaison d'ordre de grandeur}
	\begin{defn}
		Soit $x$ de $\mathbb{R}$.

		Un voisinage de $x$ est un intervalle ouvert contenant $x$.
	\end{defn}
	\begin{defn}
		Une limite $l$ en $a$ de la fonction $f$ défini dans voisinage de $a$ est :
		$$ \forall \varepsilon>0,\quad\exists N\in\mathbb{R},\quad \forall x>N,\quad f(x)\in]l-\varepsilon,l+\varepsilon[ $$
	\end{defn}
	\begin{defn}
		Soient $x$ de $\mathbb{R}$ et $f,g$ deux fonctions définies sur un voisinage $I$ de $x$.

		On dit que :
		\begin{enumerate}
			\item $f$ est un petit $o$ de $g$ au voisinage de $x$ (noté $f=o_x(g)$) s'il existe une fonction $\varepsilon:I\to \mathbb{R}$ tel que :
				$$ \forall x\in I,\quad f(x)=\varepsilon(x)g(x)\quad\land\quad \lim_{x_0 \to x} \varepsilon(x_0) = 0 $$
			\item $f$ est équivalente à $g$ au voisinage de $x$ (noté $f\sim_x g$) s'il existe une fonction $\varepsilon:I\to \mathbb{R}$ tel que :
				$$ \forall x\in I,\quad f(x)=(1+\varepsilon(x))g(x)\quad\land\quad \lim_{x_0 \to x} \varepsilon = 0 $$
		\end{enumerate}
	\end{defn}
	On note $\overline{\mathbb{R}}$ l'ensemble $\mathbb{R}\cup\{+\infty,-\infty\}$.
	\begin{thm}
		Soient $x\in\overline{\mathbb{R}}$ et $f,g$ deux fontions définis sur un voisinage $I$ de $x$ avec $g$ ne s'annulant pas en $x$.

		On dit que :
		\begin{enumerate}
			\item $f=o_x(g)$ si $\lim_{x_0 \to x} \frac{f(x_0)}{g(x_0)}=0$
			\item $f\sim_x g$ si $\lim_{x_0 \to x} \frac{f(x_0)}{g(x_0)}=1$
		\end{enumerate}
	\end{thm}
	\begin{defn}
		Un développement limité d'ordre $n$ (noté DL$_n$) en $a$ est une fonction telle que
		$$ f(a+h) = c_0+c_1h+c_2h^2+\ldots+c_nh^n+o_{h\to 0}(h^n) $$
		où $f$ admet $c_0,\ldots,c_n\in\mathbb{R}$.
	\end{defn}
	\begin{thm}[Théorème de Taylor]
		Soient $n\in\mathbb{N}$ et $f:I\to \mathbb{R}$ de classe $\mathcal{C}^n$ sur $I$.

		On a que $f$ admet un unique DL$_n$ de forme :
		$$ f(a+h)=\sum_{k=0}^{n} \frac{f^{(k)}(a)}{k!}h^k + o_{h\to 0}(h^n) $$
	\end{thm}
	On a :
	\begin{center}
		\fbox{$\displaystyle\sin(x) = x - \frac{x^3}{3!} + \frac{x^5}{5!} +\ldots+ (-1)^{n} \frac{x^{2n+1}}{(2n+1)!}+o(x^{2n+2})$}
	\end{center}
	\begin{center}
		\fbox{$\displaystyle\cos(x) = 1 - \frac{x^2}{2!} + \frac{x^4}{4!} +\ldots+ (-1)^{n} \frac{x^{2n}}{(2n)!}+o(x^{2n+1})$}
	\end{center}
	\begin{center}
		\fbox{$\displaystyle e^x = 1+\frac{x}{1!}+\frac{x^2}{x!}+\ldots+\frac{x^n}{n!}+o(x^n)$}
	\end{center}
	\begin{center}
		\fbox{$\displaystyle \ln x = \frac{x}{1!}-\frac{x^2}{x!}+\ldots+(-1)^{n-1}\frac{x^n}{n!}+o(x^n)$}
	\end{center}
	\begin{center}
		\fbox{$\displaystyle(a+x)^{\alpha}=1+\alpha x+\binom{\alpha}{2}x+\ldots+\binom{\alpha}{n}x+o(x^n)$}\\
		\fbox{$\displaystyle\binom{\alpha}{n}=\frac{\prod_{k=0}^{n} \alpha-k}{n!}$}
	\end{center}
	\begin{center}
		\fbox{$\displaystyle\frac{1}{x-1}=1-x+x^2+\ldots+(-1)^nx^n+o(x^n)$}
	\end{center}
	\begin{center}
		\fbox{Les fonctions hyperboliques sont comme les fonctions circulaires,}
		\fbox{mais sans alternance du signe}
	\end{center}
\end{document}
