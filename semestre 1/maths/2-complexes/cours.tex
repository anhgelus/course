%%=====================================================================================
%%
%%       Filename:  cours.tex
%%
%%    Description:  
%%
%%        Version:  1.0
%%        Created:  03/06/2024
%%       Revision:  none
%%
%%         Author:  YOUR NAME (), 
%%   Organization:  
%%      Copyright:  Copyright (c) 2024, YOUR NAME
%%
%%          Notes:  
%%
%%=====================================================================================
\documentclass[a4paper, titlepage]{article}

\usepackage[utf8]{inputenc}
\usepackage[T1]{fontenc}
\usepackage{textcomp}
\usepackage[french]{babel}
\usepackage{amsmath, amssymb}
\usepackage{amsthm}
\usepackage[svgnames]{xcolor}
\usepackage{thmtools}
\usepackage{lipsum}
\usepackage{framed}
\usepackage{parskip}
\usepackage{titlesec}

\renewcommand{\familydefault}{\sfdefault}

% figure support
\usepackage{import}
\usepackage{xifthen}
\pdfminorversion=7
\usepackage{pdfpages}
\usepackage{transparent}
\newcommand{\incfig}[1]{%
	\def\svgwidth{\columnwidth}
	\import{./figures/}{#1.pdf_tex}
}

\pdfsuppresswarningpagegroup=1

\colorlet{defn-color}{DarkBlue}
\colorlet{props-color}{Blue}
\colorlet{warn-color}{Red}
\colorlet{exemple-color}{Green}
\colorlet{corol-color}{Orange}
\newenvironment{defn-leftbar}{%
  \def\FrameCommand{{\color{defn-color}\vrule width 3pt} \hspace{10pt}}%
  \MakeFramed {\advance\hsize-\width \FrameRestore}}%
 {\endMakeFramed}
\newenvironment{warn-leftbar}{%
  \def\FrameCommand{{\color{warn-color}\vrule width 3pt} \hspace{10pt}}%
  \MakeFramed {\advance\hsize-\width \FrameRestore}}%
 {\endMakeFramed}
\newenvironment{exemple-leftbar}{%
  \def\FrameCommand{{\color{exemple-color}\vrule width 3pt} \hspace{10pt}}%
  \MakeFramed {\advance\hsize-\width \FrameRestore}}%
 {\endMakeFramed}
\newenvironment{props-leftbar}{%
  \def\FrameCommand{{\color{props-color}\vrule width 3pt} \hspace{10pt}}%
  \MakeFramed {\advance\hsize-\width \FrameRestore}}%
 {\endMakeFramed}
\newenvironment{corol-leftbar}{%
  \def\FrameCommand{{\color{corol-color}\vrule width 3pt} \hspace{10pt}}%
  \MakeFramed {\advance\hsize-\width \FrameRestore}}%
 {\endMakeFramed}

\def \freespace {1em}
\declaretheoremstyle[headfont=\sffamily\bfseries,%
 notefont=\sffamily\bfseries,%
 notebraces={}{},%
 headpunct=,%
 bodyfont=\sffamily,%
 headformat=\color{defn-color}Définition~\NUMBER\hfill\NOTE\smallskip\linebreak,%
 preheadhook=\vspace{\freespace}\begin{defn-leftbar},%
 postfoothook=\end{defn-leftbar},%
]{better-defn}
\declaretheoremstyle[headfont=\sffamily\bfseries,%
 notefont=\sffamily\bfseries,%
 notebraces={}{},%
 headpunct=,%
 bodyfont=\sffamily,%
 headformat=\color{warn-color}Attention\hfill\NOTE\smallskip\linebreak,%
 preheadhook=\vspace{\freespace}\begin{warn-leftbar},%
 postfoothook=\end{warn-leftbar},%
]{better-warn}
\declaretheoremstyle[headfont=\sffamily\bfseries,%
 notefont=\sffamily\bfseries,%
notebraces={}{},%
headpunct=,%
 bodyfont=\sffamily,%
 headformat=\color{exemple-color}Exemple~\NUMBER\hfill\NOTE\smallskip\linebreak,%
 preheadhook=\vspace{\freespace}\begin{exemple-leftbar},%
 postfoothook=\end{exemple-leftbar},%
]{better-exemple}
\declaretheoremstyle[headfont=\sffamily\bfseries,%
 notefont=\sffamily\bfseries,%
 notebraces={}{},%
 headpunct=,%
 bodyfont=\sffamily,%
 headformat=\color{props-color}Proposition~\NUMBER\hfill\NOTE\smallskip\linebreak,%
 preheadhook=\vspace{\freespace}\begin{props-leftbar},%
 postfoothook=\end{props-leftbar},%
]{better-props}
\declaretheoremstyle[headfont=\sffamily\bfseries,%
 notefont=\sffamily\bfseries,%
 notebraces={}{},%
 headpunct=,%
 bodyfont=\sffamily,%
 headformat=\color{props-color}Théorème~\NUMBER\hfill\NOTE\smallskip\linebreak,%
 preheadhook=\vspace{\freespace}\begin{props-leftbar},%
 postfoothook=\end{props-leftbar},%
]{better-thm}
\declaretheoremstyle[headfont=\sffamily\bfseries,%
 notefont=\sffamily\bfseries,%
 notebraces={}{},%
 headpunct=,%
 bodyfont=\sffamily,%
 headformat=\color{corol-color}Corollaire~\NUMBER\hfill\NOTE\smallskip\linebreak,%
 preheadhook=\vspace{\freespace}\begin{corol-leftbar},%
 postfoothook=\end{corol-leftbar},%
]{better-corol}

\declaretheorem[style=better-defn]{defn}
\declaretheorem[style=better-warn]{warn}
\declaretheorem[style=better-exemple]{exemple}
\declaretheorem[style=better-corol]{corol}
\declaretheorem[style=better-props, numberwithin=defn]{props}
\declaretheorem[style=better-thm, sibling=props]{thm}
\newtheorem*{lemme}{Lemme}%[subsection]
%\newtheorem{props}{Propriétés}[defn]

\newenvironment{system}%
{\left\lbrace\begin{align}}%
{\end{align}\right.}

\newenvironment{AQT}{{\fontfamily{qbk}\selectfont AQT}}

\usepackage{LobsterTwo}
\titleformat{\section}{\newpage\LobsterTwo \huge\bfseries}{\thesection.}{1em}{}
\titleformat{\subsection}{\vspace{2em}\LobsterTwo \Large\bfseries}{\thesubsection.}{1em}{}
\titleformat{\subsubsection}{\vspace{1em}\LobsterTwo \large\bfseries}{\thesubsubsection.}{1em}{}

\newenvironment{lititle}%
{\vspace{7mm}\LobsterTwo \large}%
{\\}

\renewenvironment{proof}{$\square$ \footnotesize\textit{Démonstration.}}{\begin{flushright}$\blacksquare$\end{flushright}}

\title{Complexes}
\author{William Hergès\thanks{Sorbonne Université - Faculté des Sciences, Faculté des Lettres}}

\begin{document}
	\maketitle
	\tableofcontents
	\newpage
	\section{Présentation}
	\begin{defn}
		L'ensemble des nombres complexes est : $$ \mathbb{C} = \{a+ib|a,b\in\mathbb{R}\} $$
		où $i^2 = -1$.

		On a :
		\begin{align*}
			a+ib+a'+ib' &= a+a'+(b+b')i \\
			(a+ib)(a'+ib') &= aa'+aib'+a'ib-bb'
		\end{align*}
	\end{defn}
	On utilise la lettre $z$ pour les nombres complexes.
	\begin{defn}
		Soit $z\in\mathbb{C}$ tel que $z=a+bi$ (où $a,b\in\mathbb{R}$).

		On note $\mathfrak{Re}(z)$ la partie réelle de $z$ qui est $a$.\\
		On note $\mathfrak{Im}(z)$ la partie imaginaire de $z$ qui est $b$.\\
		On note $|z|$ le module de $z$ qui est $\sqrt{a^2+b^2}$.\\
		On note $\arg z$ l'argument de $z$ qui est l'angle entre la droite $OZ$ et la droite $\mathbb{R}^+$ (où $Z$ est le point d'affixe $z$).
	\end{defn}
	\begin{props}[Forme trigonométrique]
		On peut donc écrire $z$ comme :
		$$ z = |z|(\cos(\arg z)+i\sin(\arg z)) $$
	\end{props}
	\begin{proof}
		\AQT
	\end{proof}
	\begin{props}
		On a :
		$$ \arg z + \arg w = \arg(zw) $$
		(où $z$ et $w$ sont deux nombres complexes.)
	\end{props}
	\begin{proof}
		\AQT
	\end{proof}
	\begin{defn}[Forme exponentielle]
		On note $z\in\mathbb{C}$ maintenant :
		$$ z = |z|e^{i\arg z} $$
		avec $e^{i\alpha} := \cos\alpha+i\sin\alpha$
	\end{defn}
	On utilise cette notation car les calculs sont les mêmes que ceux de la forme trigonométrique (cf. la proposition précédente).
	\begin{exemple}
		$z=|z|(\cos\alpha+i\sin\alpha)$ et $w=|w|(\cos\beta+i\sin\beta)$.

		On a :
		$$ zw = |z||w|(\cos(\alpha+\beta)+i\sin(\alpha+\beta)) = |z||w|e^{i(\alpha+\beta)} $$
	\end{exemple}
	\section{Racines}
	\begin{thm}[Racines de l'unité]
		On a que toutes les solutions de :
		$$ z^n = 1 $$
		où $n\in\mathbb{N}$ et $z\in\mathbb{C}$, sont :
		$$ \{e^{\frac{2ik\pi}{n}}|k\in\mathbb{N}\} $$
	\end{thm}
	\begin{proof}
		\AQT
	\end{proof}
	On peut réduire l'ensemble des $k$ à $[|0;n-1|]$ car l'argument de $z$ est modulo $2\pi$.
	\begin{props}
		La somme des racines de l'unité est nulle
	\end{props}
	\begin{proof}
		\AQT
	\end{proof}
	\subsection{Racine carré d'un complexe}
	\subsubsection{Complexe sous forme exponentielle}
	Soit $z\in\mathbb{C}$. On cherche $x\in\mathbb{C}$. On note $x=re^{i\alpha}$ et $z=se^{i\beta}$.
	\begin{align*}
		x^2 &= z \\
		(re^{i\alpha})^2 &= re^{i\alpha} \\
		r^2e^{2i\alpha}	&= se^{i\beta}
	\end{align*}
	$x$ a comme module $\sqrt{|z|}$ et a comme argument $\frac{\beta}{2}$ ou $\frac{\beta}{2}+\pi$.

	$x$ est donc $$ \left\{ \sqrt{|z|}e^{i\frac{\beta}{2}},\sqrt{|z|}e^{i\frac{\beta}{2}+\pi} \right\}  $$
	\subsubsection{Complexe sous forme cartésienne}
	Soit $X\in\mathbb{C}$ tel que $X=x+iy$ (où $x,y\in\mathbb{R}$). Soit $z\in\mathbb{C}$ tel que $z=a+ib$ (où $a,b\in\mathbb{R}$).
	\begin{align*}
		X^2&= z \\
		(x+iy)^2&= a+ib \\
		x^2+2ixy-y^2 &= a+ib \\
	\end{align*}
	Et on a :
	\begin{align*}
		x^2-y^2 &=a\\
		2xy &= b\\
		x^2+y^2 &= \sqrt{a^2+b^2}~\text{ car on $|x|^2=|z|$}
	\end{align*}
	Et on résout.
	\section{Polynômes}
	\begin{defn}
		Soit $(\lambda_0,\ldots,\lambda_n)$ une famille de nombres complexes.

		On note le polynôme lié à la famille $$P(X)=\sum_{i=0}^{n} \lambda_i x^i$$
	\end{defn}
	\begin{defn}
		Le degré d'un polynôme $P$ est noté $\mathrm{deg}P$ tel que :
		$$ \lambda_n\neq 0,\quad\forall i\in\mathbb{N}\geqslant n,\quad\lambda_i = 0 $$
		où $n$ est le degré du polynôme.
	\end{defn}
	\begin{defn}
		Une racine $r\in\mathbb{K}$ du polynôme $P$ est défini telle que $P(r)=0$.
	\end{defn}
	\begin{thm}[Théorème de d'Alembert-Gauss]
		Pour tout polynôme $P$ de degré $n$, il existe exactement $n$ racines compté avec leur ordre de multiplicité. i.e.
		$$ P(X)=\lambda_n\prod_{k=1}^{n} (x-x_k) $$
		où la famille $(x_1,\ldots,x_n)$ sont les racines de $P$.
	\end{thm}
	\begin{proof}
		Admis.
	\end{proof}
\end{document}
