\documentclass[a4paper]{article}

\usepackage[utf8]{inputenc}
\usepackage[T1]{fontenc}
\usepackage{textcomp}
\usepackage[french]{babel}
\usepackage{amsmath, amssymb}
\usepackage{amsthm}
\usepackage[svgnames]{xcolor}
\usepackage{thmtools}
\usepackage{lipsum}
\usepackage{framed}
\usepackage{parskip}

\renewcommand{\familydefault}{\sfdefault}

\newenvironment{AQT}{{\fontfamily{qbk}\selectfont AQT}}

\usepackage{titlesec}
\usepackage{LobsterTwo}
\titleformat{\section}{\newpage\LobsterTwo \huge\bfseries}{\thesection.}{1em}{}
\titleformat{\subsection}{\vspace{2em}\LobsterTwo \Large\bfseries}{\thesubsection.}{1em}{}
\titleformat{\subsubsection}{\vspace{1em}\LobsterTwo \large\bfseries}{\thesubsubsection.}{1em}{}

\title{TD Maths discrètes}
\author{William Hergès\thanks{Sorbonne Université}}

\begin{document}
	\maketitle
    \section*{Exercice 1}
    \begin{enumerate}
        \item Maj = $\{1,2,3\}$
        \item Min = $\{6,7,8\}$
        \item $\sup V = 3$
        \item $\inf V = 6$
    \end{enumerate}
    \section*{Exercice 3}
    Soit $(x,y)\in \mathbb{N}^2$.
    On a que $(x,y)\preceq(x,y)$ car $(x,y)=(x,y)$.
    $\preceq$ est réflexive.

    Soit $(x,y)\in\mathbb{N}^2$ et $(x',y')\in\mathbb{N}^2$ tel que $(x,y)\preceq (x',y')$ et $(x',y')\preceq (x,x)$.
    On a que $x+y < x'+y' \land x'+y' < x+y$ ou $(x,y)=(x',y')$.
    La première possibilité est impossible.
    Donc la deuxième est forcément vraie.
    Ainsi, $\preceq$ est anti-symétrique.

    Soit $(a,b)\in\mathbb{N}^2$, $(c,d)\in\mathbb{N}^2$ et $(e,f)\in\mathbb{N}^2$ tel que~:
    $$ (a,b)\preceq (c,d)\land (c,d)\preceq (e,f) $$
    Alors, soit $a+b < c+d$, soit $(a,b) = (c,d)$ et soit $c+d<e+f$, soit $(c,d)=(e,f)$
    Si $a+b < c+d$, alors $a+b < e+f$ dans tous les cas.
    Si $(a,b) = (c,d)$, on a que $(a,b)\preceq (e,f)$ est vraie.

    Ainsi, $\preceq$ est une relation d'ordre.

    Cet ordre n'est pas total car $(1,0)$ et $(0,1)$ ne sont pas en relations.

    Soit $(a,b)$ un élément plus petit que $(0,0)$.
    On a donc que $a+b < 0+0 \iff a + b < 0$ ou $(a,b)=(0,0)$.
    La première possibilité est impossible, donc $\min\{\mathbb{N}^2,\preceq\} = (0,0)$.
    Ainsi, cet élément est bien fondé.
    \section*{Exercice 4}
    Soit $x$. On a que $x$ divise $x$. Donc $|$ est réflexive.

    Soient $(x,y)\in(\mathbb{N}\backslash\{0,1\})^2$ tels que~: $$ x | y \land y | x $$
    Il existe donc $k_1$ et $k_2$ dans $\mathbb{N}^*$ tels que~: $$ x = k_1y \land y = k_2x $$
    Or $$ k_1k_2y = y $$
    Donc $$ k_1=k_2=1 $$
    i.e. $$ x = y $$
    Ainsi, $|$ est anti-symétrique.

    Soient $(x,y,z)\in(\mathbb{N}\backslash\{0,1\})^3$ tels que ~: $$ x | y \land y | z $$
    Il existe donc $k_1$ et $k_2$ dans $\mathbb{N}^*$ tels que~: $$ x = yk_1 \land y = zk_2 $$
    Or $$ x = zk_2k_1 $$
    Donc $$ x | z $$
    Ainsi, $|$ est transitive.

    Alors, $|$ est une relation d'ordre.

    Les éléments minimaux de $E$ sont les nombres premiers car ils ne sont divisibles que par $1$ (qui n'est pas dans 
    $E$) et par lui-même.

    $E$ ne possède pas d'éléments maximaux.
    Si $x$ est un élément maximal, alors $2x$ est plus grand que $x$ et est divisé par $x$, donc $x$ n'est pas un 
    élément maximal.
    \section*{Exercice 5}
    Ici, $\inf$ est le $\mathrm{PGCD}$ de $x$ et $y$ et $\sup$ est le $\mathrm{PPCM}$.

    Les minorants de $A$ sont~: $$ \{1,3\} $$
    Les minorants de $B$ sont~: $$ \{1\} $$
    Les majorants de $A$ sont~: $$ \{\sup(A)k | k \in\mathbb{N}^*\} = \left\{\frac{15\times 21}3k | k \in\mathbb{N}^*\right\} $$
    Les majorants de $B$ sont~: $$ \{\sup(B)k| k \in\mathbb{N}^*\} = \left\{\frac{14\times 21}7k | k \in\mathbb{N}^*\right\}$$

    $A$ ne possède ni de plus petit, ni de plus grand élément.
    $B$ possède un plus petit élément ($1$), mais pas de plus grand.

    $\{1,3\}$ sont les minorants de $A$.
    Les majorants de $A$ sont~: $$ \{\sup(A)k | k\in\mathbb{N}^*\} = \left\{\frac{12\times 15}{3}k | k\in\mathbb{N}^*\right\} $$
    $\inf A=\min A = 3$, et les éléments minimaux sont $\{3\}$.
    Il n'a pas de borne sup, ni d'éléments maximaux, ni de plus grands éléments.

    Les minorants et majorants ne sont pas forcément dans $A$.

    Les éléments minimaux et maximaux sont dans $A$.
    Ce sont les minorants et les majorants dans $A$.

    Les bornes ne sont pas forcément dans $A$.

    Le minimum et le maximum sont dans $A$.
    C'est la borne inférieur et supérieur dans $A$.
    Ils sont aussi appelés le plus petit et le plus grand élément.
\end{document}
