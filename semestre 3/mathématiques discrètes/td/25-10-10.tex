\documentclass[a4paper]{article}

\usepackage[utf8]{inputenc}
\usepackage[T1]{fontenc}
\usepackage{textcomp}
\usepackage[french]{babel}
\usepackage{amsmath, amssymb}
\usepackage{amsthm}
\usepackage[svgnames]{xcolor}
\usepackage{thmtools}
\usepackage{lipsum}
\usepackage{framed}
\usepackage{parskip}

\renewcommand{\familydefault}{\sfdefault}

\newenvironment{AQT}{{\fontfamily{qbk}\selectfont AQT}}

\usepackage{titlesec}
\usepackage{LobsterTwo}
\titleformat{\section}{\newpage\LobsterTwo \huge\bfseries}{\thesection.}{1em}{}
\titleformat{\subsection}{\vspace{2em}\LobsterTwo \Large\bfseries}{\thesubsection.}{1em}{}
\titleformat{\subsubsection}{\vspace{1em}\LobsterTwo \large\bfseries}{\thesubsubsection.}{1em}{}

\title{TD Maths discrètes}
\author{William Hergès\thanks{Sorbonne Université}}

\begin{document}
	\maketitle
    \section*{Retour sur les copies}
    Les résultats ne sont pas bons.
    Tout le monde devrait pouvoir avoir $5/10$ (la moyenne est à $4/10$).
    Il y a un problème de vitesse et d'apprentissage du cours.
    \section*{Exercice 11}
    Soient $a$ un plus grand élément de $A$ et $m$ le maximum de $A$.
    Par anti-symétrie, $a\preceq M$ et $M\preceq a$ par leurs définitions, alors $a=M$.
    Si $m_1$ et $m_2$ sont deux maximums de $A$, par anti-symétrie, ils sont égaux.

    $\mathbb{N}\cup \{a\}$ munie de $\leqslant$ possède un élément maximal ($a$), mais pas de maximum (il n'existe pas
    de majorant dans $\mathbb{N}$).

    Soient $m_1$ et $m_2$ deux éléments maximals de $A$.
    Comme $\preceq$ est totale, on a $m_1\preceq m_2$ et $m_2\preceq m_1$.
    Par anti-symétrie, $m_1=m_2$, i.e. l'élément maximal est unique.
    \section*{Exercice 10}
    Il y a le même nombre d'élément dans $P=\mathcal{P}(\{a,b,c\})$ et dans $E=\{1,2,3,6,7,14,21,42\}$.
    Si on définit $f$ de $P$ dans $E$ tels que $f(p) = \prod_{a\in p} a$ où l'équivalent dans $\mathbb{N}$ est 
    $\varnothing = 1$, $\{a\}=2$, $\{b\}=3$ et $\{c\}=7$.
    Cette application est bijective (la réciproque est $f^{-1}(e) = \{p,p|e\}$ avec la même clé).
    Elle est aussi monotone car $p_1\subseteq p_2$ et $f(p_1) | f(p_2)$ car $f(p_2)$ est une mulitple de $p_1$.

    Sinon, on dessine le graphe couvrant~: c'est plus propre.

    Version propre de mon idée~: après avoir défini les valeurs, on définit $f$ comme étant
    $f(A\cup B) = f(A)\times f(B)$.
    \section*{Exercice 16}
    Elle n'est clairement pas bijective car $8 = f(\{1,3,4\}) = f(\{8\})$.
    \section*{Exercice 8}
    «~$\alpha$ est un préfixe de $\beta$~» est~:
    $$ \forall k < \min(n_1,n_2),\quad \alpha_k = \beta_k $$
    avec $n_1$ la taille de $\alpha$ et $n_2$ est la taille de $\beta$.
    On note cette relation $\preceq$.

    $$ (\exists i, \forall k < i, a_k = b_k \land a_i \preceq_A b_i) \lor a = b$$
    C'est un ordre total car $\preceq_A$ est total.

    Soit $n\in\mathbb{N}$.
    On a $a^nb \preceq a^{n+1}b$.
    Donc $\preceq$ n'est pas bien fondé.
    \section*{Exercice en plus}
    Soit $A$ un alphabet composé de $\{a,b\}$.
    On définit un langage $L\subseteq A^*$ par induction structurelle~:
    \begin{itemize}
        \item $a\in L$
        \item $\forall (u,v)\in L^2, uvb\in L$
    \end{itemize}
    Soit $u\in L$ où $u$ possède une taille de $2$.
    Ainsi, $u$ est soit égal à $a$, soit de la forme $v_1v_2b$.
    Donc $b$ est forcément dans $u$.
    Ainsi, $v_1 = a$ ou $v_1 = \varepsilon$ et $v_2$ est le choix restant.
    Or, $\varepsilon$ n'est pas dans $L$, donc $u$ est impossible.

    $u\in L$.
    Donc $aua$ est aussi dans $L$. Ainsi, le miroir de $aua$ est $a\tilde u a$.
    Donc $bub$ est aussi dans $L$. Ainsi, le miroir de $bub$ est $b\tilde u b$.
    Comme $\tilde u = u$, on a que la propriété est vraie.
\end{document}
