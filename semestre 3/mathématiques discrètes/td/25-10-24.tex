\documentclass[a4paper]{article}

\usepackage[utf8]{inputenc}
\usepackage[T1]{fontenc}
\usepackage{textcomp}
\usepackage[french]{babel}
\usepackage{amsmath, amssymb}
\usepackage{amsthm}
\usepackage[svgnames]{xcolor}
\usepackage{thmtools}
\usepackage{lipsum}
\usepackage{framed}
\usepackage{parskip}

\renewcommand{\familydefault}{\sfdefault}

\newenvironment{AQT}{{\fontfamily{qbk}\selectfont AQT}}

\usepackage{titlesec}
\usepackage{LobsterTwo}
\titleformat{\section}{\newpage\LobsterTwo \huge\bfseries}{\thesection.}{1em}{}
\titleformat{\subsection}{\vspace{2em}\LobsterTwo \Large\bfseries}{\thesubsection.}{1em}{}
\titleformat{\subsubsection}{\vspace{1em}\LobsterTwo \large\bfseries}{\thesubsubsection.}{1em}{}

\title{TD Maths discrètes}
\author{William Hergès\thanks{Sorbonne Université}}

\begin{document}
	\maketitle
    \section*{Exercice 1}
    $(0,0)$, $(0,1)$, $(0,2)$, $(1,2)$, $(2,2)$

    $n$ tq $(n,n)$ est dans $\mathrm{Inf}_1$.
    Donc $(n+1,n+1)$ est dans $\mathrm{Inf}_1$.
    \section*{Exercice 2}
    $u$ et $v$ dans $L$.
    $$ w = a.u.v $$
    $$ |w|_a = 1+|u|_a+|v|_b = |u|_b + |v|_b - 1 $$
    $$ |w|_b = |u|_b + |v|_b $$
    Donc 
    $$ |w|_a + 1 = |w_b| $$
    \section*{Exercice 3}
    $$ h(\{t,a,b\}) = \left\{\begin{matrix}h(\varepsilon) = 0\\ h(\{t, a, b\}) = 1 + \max\{h(a),h(b)\}\end{matrix}\right. $$
    $$ n(\{t,a,b\}) = \left\{\begin{matrix}h(\varepsilon) = 0\\ h(\{t, a, b\}) = 1 + n(a)+n(b)\end{matrix}\right. $$
    $$ ar(\{t,a,b\}) = \max\{0,h(\{t,a,b\})-1\} $$
\end{document}
