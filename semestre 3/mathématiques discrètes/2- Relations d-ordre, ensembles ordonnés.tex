\documentclass[a4paper, titlepage]{article}

\usepackage[utf8]{inputenc}
\usepackage[T1]{fontenc}
\usepackage{textcomp}
\usepackage[french]{babel}
\usepackage{amsmath, amssymb}
\usepackage{amsthm}
\usepackage[svgnames]{xcolor}
\usepackage{thmtools}
\usepackage{lipsum}
\usepackage{framed}
\usepackage{parskip}
\usepackage{titlesec}

\renewcommand{\familydefault}{\sfdefault}

% figure support
\usepackage{import}
\usepackage{xifthen}
\pdfminorversion=7
\usepackage{pdfpages}
\usepackage{transparent}
\newcommand{\incfig}[1]{%
	\def\svgwidth{\columnwidth}
	\import{./figures/}{#1.pdf_tex}
}

\pdfsuppresswarningpagegroup=1

\colorlet{defn-color}{DarkBlue}
\colorlet{props-color}{Blue}
\colorlet{warn-color}{Red}
\colorlet{exemple-color}{Green}
\colorlet{corol-color}{Orange}
\newenvironment{defn-leftbar}{%
  \def\FrameCommand{{\color{defn-color}\vrule width 3pt} \hspace{10pt}}%
  \MakeFramed {\advance\hsize-\width \FrameRestore}}%
 {\endMakeFramed}
\newenvironment{warn-leftbar}{%
  \def\FrameCommand{{\color{warn-color}\vrule width 3pt} \hspace{10pt}}%
  \MakeFramed {\advance\hsize-\width \FrameRestore}}%
 {\endMakeFramed}
\newenvironment{exemple-leftbar}{%
  \def\FrameCommand{{\color{exemple-color}\vrule width 3pt} \hspace{10pt}}%
  \MakeFramed {\advance\hsize-\width \FrameRestore}}%
 {\endMakeFramed}
\newenvironment{props-leftbar}{%
  \def\FrameCommand{{\color{props-color}\vrule width 3pt} \hspace{10pt}}%
  \MakeFramed {\advance\hsize-\width \FrameRestore}}%
 {\endMakeFramed}
\newenvironment{corol-leftbar}{%
  \def\FrameCommand{{\color{corol-color}\vrule width 3pt} \hspace{10pt}}%
  \MakeFramed {\advance\hsize-\width \FrameRestore}}%
 {\endMakeFramed}

\def \freespace {1em}
\declaretheoremstyle[headfont=\sffamily\bfseries,%
 notefont=\sffamily\bfseries,%
 notebraces={}{},%
 headpunct=,%
 bodyfont=\sffamily,%
 headformat=\color{defn-color}Définition~\NUMBER\hfill\NOTE\smallskip\linebreak,%
 preheadhook=\vspace{\freespace}\begin{defn-leftbar},%
 postfoothook=\end{defn-leftbar},%
]{better-defn}
\declaretheoremstyle[headfont=\sffamily\bfseries,%
 notefont=\sffamily\bfseries,%
 notebraces={}{},%
 headpunct=,%
 bodyfont=\sffamily,%
 headformat=\color{warn-color}Attention~\NUMBER\hfill\NOTE\smallskip\linebreak,%
 preheadhook=\vspace{\freespace}\begin{warn-leftbar},%
 postfoothook=\end{warn-leftbar},%
]{better-warn}
\declaretheoremstyle[headfont=\sffamily\bfseries,%
 notefont=\sffamily\bfseries,%
notebraces={}{},%
headpunct=,%
 bodyfont=\sffamily,%
 headformat=\color{exemple-color}Exemple~\NUMBER\hfill\NOTE\smallskip\linebreak,%
 preheadhook=\vspace{\freespace}\begin{exemple-leftbar},%
 postfoothook=\end{exemple-leftbar},%
]{better-exemple}
\declaretheoremstyle[headfont=\sffamily\bfseries,%
 notefont=\sffamily\bfseries,%
 notebraces={}{},%
 headpunct=,%
 bodyfont=\sffamily,%
 headformat=\color{props-color}Proposition~\NUMBER\hfill\NOTE\smallskip\linebreak,%
 preheadhook=\vspace{\freespace}\begin{props-leftbar},%
 postfoothook=\end{props-leftbar},%
]{better-props}
\declaretheoremstyle[headfont=\sffamily\bfseries,%
 notefont=\sffamily\bfseries,%
 notebraces={}{},%
 headpunct=,%
 bodyfont=\sffamily,%
 headformat=\color{props-color}Théorème~\NUMBER\hfill\NOTE\smallskip\linebreak,%
 preheadhook=\vspace{\freespace}\begin{props-leftbar},%
 postfoothook=\end{props-leftbar},%
]{better-thm}
\declaretheoremstyle[headfont=\sffamily\bfseries,%
 notefont=\sffamily\bfseries,%
 notebraces={}{},%
 headpunct=,%
 bodyfont=\sffamily,%
 headformat=\color{corol-color}Corollaire~\NUMBER\hfill\NOTE\smallskip\linebreak,%
 preheadhook=\vspace{\freespace}\begin{corol-leftbar},%
 postfoothook=\end{corol-leftbar},%
]{better-corol}

\declaretheorem[style=better-defn]{defn}
\declaretheorem[style=better-warn]{warn}
\declaretheorem[style=better-exemple]{exemple}
\declaretheorem[style=better-corol]{corol}
\declaretheorem[style=better-props, numberwithin=defn]{props}
\declaretheorem[style=better-thm, sibling=props]{thm}
\newtheorem*{lemme}{Lemme}%[subsection]
%\newtheorem{props}{Propriétés}[defn]

\newenvironment{system}%
{\left\lbrace\begin{align}}%
{\end{align}\right.}

\newenvironment{AQT}{{\fontfamily{qbk}\selectfont AQT}}

\usepackage{LobsterTwo}
\titleformat{\section}{\newpage\LobsterTwo \huge\bfseries}{\thesection.}{1em}{}
\titleformat{\subsection}{\vspace{2em}\LobsterTwo \Large\bfseries}{\thesubsection.}{1em}{}
\titleformat{\subsubsection}{\vspace{1em}\LobsterTwo \large\bfseries}{\thesubsubsection.}{1em}{}

\newenvironment{lititle}%
{\vspace{7mm}\LobsterTwo \large}%
{\\}

\renewenvironment{proof}{\par$\square$ \footnotesize\textit{Démonstration.}}{\begin{flushright}$\blacksquare$\end{flushright}\par}

\title{Relations d'ordre, ensembles ordonnés}
\author{William Hergès\thanks{Sorbonne Université}}

\begin{document}
	\maketitle
	\tableofcontents
	\newpage
    \section{ensemble ordoné}
    \begin{defn}
        Une relation d'ordre $\preceq$ est une relation binaire sur $E$ si et seulement si~:
        \begin{itemize}
            \item réflexive
            \item anti-symétrique
            \item transitive
        \end{itemize}

        L'ordre strict $\prec$ est associé à $\preceq$~: c'est la même, sauf qu'elle n'est pas réflexive~:
        $$ \prec = \preceq\backslash\mathrm{Id}_E $$
    \end{defn}
    \begin{defn}
        Une relation d'ordre $\preceq$ est~:
        \begin{itemize}
            \item totale si et seulement si $\preceq$ permet toujours de comparer deux éléments quelconques de $E$
            \item partielle s'il existe au moins deux éléments de $E$ incomparables avec $\preceq$
        \end{itemize}
    \end{defn}
    \begin{defn}
        $(E,\preceq)$ est un ensemble~:
        \begin{itemize}
            \item totalement ordonné si $\preceq$ est un ordre total.
            \item partiellement ordonné si $\preceq$ est un ordre partiel.
        \end{itemize}
    \end{defn}
    \begin{exemple}
        $(\mathbb{N},\leqslant )$ est un ensemble totalement ordonné.

        $(\mathcal{P}(F),\subseteq)$ est un ensemble partiellement ordoné (pour $F$ un ensemble quelconque).

        $(\mathbb{N}^*, |)$ est aussi partiellement ordoné, où
        $$ | = \{(a,b)|\exists k\in\mathbb{N}^*, b = na\} $$
        (c'est la relation divise.)
    \end{exemple}
    \begin{proof}
        Preuve du deuxième exemple.

        Soit $F$ un ensemble.

        Montrons que $\subseteq$ est un ordre pour $\mathcal{P}(F)$.
        \begin{itemize}
            \item Soit $A\in\mathcal{P}(F)$.
                Triviallement, $A\subseteq A$.
                Alors, $\subseteq$ est réflexive.
            \item Soit $(A,B)\in\mathcal{P}(F)^2$.
                Supposons que $A\subseteq B$ et que $B\subseteq A$.
                Alors, $A=B$ par définition.
            \item Soit $(A,B,C)\in\mathcal{P}(F)^3$ avec $A\subseteq B$ et $B\subseteq C$.
                Si $A$ est l'ensemble vide, il est inclu dans tous les ensembles. 
                Donc $A\subseteq C$.
                Si $A$ n'est pas l'ensemble vide, tous ses éléments sont dans $B$. 
                Or, tous les éléments de $B$ sont dans $C$.
                Donc, tous les éléments de $A$ sont dans $C$.
                Alors, $A\subseteq C$.
        \end{itemize}
        Ainsi, $\subseteq$ est bien un ordre pour $\mathcal{P}(F)$.

        Montrons que $\subseteq$ est un ordre partiel.

        Supposons que $F$ contient au moins deux éléments.

        Soit $(x,y)\in F^2$, deux éléments différents.
        Soient $A=\{x\}$ et $B=\{y\}$.

        On a que $A\not\subseteq B$ et que $B\subseteq A$.
        Donc, $\subseteq$ est partiel dans ce cas.

        Si $F$ est vide, alors $\mathcal{P}(F)$ contient un unique élément.
        Cet ensemble est totalement ordonné.

        Si $F$ est un singleton, alors $\mathcal{P}(F)$ contient $F$ et l'ensemble vide.
        Cet ensemble est totalement ordonné.

        La slide est ainsi fausse, mais les ensembles à moins de deux éléments sont peux intéressants.
    \end{proof}
    Revoir les slides 4 - 6.
    \begin{defn}
        Soient $(E_1,\preceq_1)$ et $(E_2,\preceq_2)$ deux ensemble ordonnés.

        L'application $f:E_1\to E_2$ est dite monotone si~:
        $$ \forall (x,y)\in E_1^2,\quad x\preceq_1 y \implies f(x)\preceq_2 f(y) $$
    \end{defn}
    Une application monotone préserve les relations d'ordre.
    \begin{exemple}
        On se place dans $(\mathbb{N},\leqslant)$ et dans $(\mathcal{P}(\mathbb{N}))$

        $f:\mathbb{N}\to \mathcal{P}(\mathbb{N})$ tel que $f(n) = \{k\in\mathbb{N}|k\leqslant n\}$ est monotone.

        $g:\mathbb{N}\to \mathcal{P}(\mathbb{N})$ tel que $g(n)=\{n\}$ ne l'est pas par contre~!
    \end{exemple}
    \begin{props}
        Deux ensembles ordonnés $(E_1,\preceq_1)$ et $(E_2,\preceq_2)$ sont isomorphes s'il existe une bijection
        $f:E_1\to E_2$ telle que $f$ et $f^{-1}$ sont monotones.
    \end{props}
    Slide 8 pour des exemples et pour le retour des graphes.
    \begin{warn}
        Une bijection $f$ peut être monotone sans que $f^{-1}$ ne le soit~!
    \end{warn}
    \begin{proof}
        Fin de la slide 8 pour la preuve.
    \end{proof}
\end{document}
