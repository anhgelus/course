\documentclass[a4paper, titlepage]{article}

\usepackage[utf8]{inputenc}
\usepackage[T1]{fontenc}
\usepackage{textcomp}
\usepackage[french]{babel}
\usepackage{amsmath, amssymb}
\usepackage{amsthm}
\usepackage[svgnames]{xcolor}
\usepackage{thmtools}
\usepackage{lipsum}
\usepackage{framed}
\usepackage{parskip}
\usepackage{titlesec}
\usepackage{tikz}

\renewcommand{\familydefault}{\sfdefault}

% figure support
\usepackage{import}
\usepackage{xifthen}
\pdfminorversion=7
\usepackage{pdfpages}
\usepackage{transparent}
\newcommand{\incfig}[1]{%
	\def\svgwidth{\columnwidth}
	\import{./figures/}{#1.pdf_tex}
}

\pdfsuppresswarningpagegroup=1

\colorlet{defn-color}{DarkBlue}
\colorlet{props-color}{Blue}
\colorlet{warn-color}{Red}
\colorlet{exemple-color}{Green}
\colorlet{corol-color}{Orange}
\newenvironment{defn-leftbar}{%
  \def\FrameCommand{{\color{defn-color}\vrule width 3pt} \hspace{10pt}}%
  \MakeFramed {\advance\hsize-\width \FrameRestore}}%
 {\endMakeFramed}
\newenvironment{warn-leftbar}{%
  \def\FrameCommand{{\color{warn-color}\vrule width 3pt} \hspace{10pt}}%
  \MakeFramed {\advance\hsize-\width \FrameRestore}}%
 {\endMakeFramed}
\newenvironment{exemple-leftbar}{%
  \def\FrameCommand{{\color{exemple-color}\vrule width 3pt} \hspace{10pt}}%
  \MakeFramed {\advance\hsize-\width \FrameRestore}}%
 {\endMakeFramed}
\newenvironment{props-leftbar}{%
  \def\FrameCommand{{\color{props-color}\vrule width 3pt} \hspace{10pt}}%
  \MakeFramed {\advance\hsize-\width \FrameRestore}}%
 {\endMakeFramed}
\newenvironment{corol-leftbar}{%
  \def\FrameCommand{{\color{corol-color}\vrule width 3pt} \hspace{10pt}}%
  \MakeFramed {\advance\hsize-\width \FrameRestore}}%
 {\endMakeFramed}

\def \freespace {1em}
\declaretheoremstyle[headfont=\sffamily\bfseries,%
 notefont=\sffamily\bfseries,%
 notebraces={}{},%
 headpunct=,%
 bodyfont=\sffamily,%
 headformat=\color{defn-color}Définition~\NUMBER\hfill\NOTE\smallskip\linebreak,%
 preheadhook=\vspace{\freespace}\begin{defn-leftbar},%
 postfoothook=\end{defn-leftbar},%
]{better-defn}
\declaretheoremstyle[headfont=\sffamily\bfseries,%
 notefont=\sffamily\bfseries,%
 notebraces={}{},%
 headpunct=,%
 bodyfont=\sffamily,%
 headformat=\color{warn-color}Attention~\NUMBER\hfill\NOTE\smallskip\linebreak,%
 preheadhook=\vspace{\freespace}\begin{warn-leftbar},%
 postfoothook=\end{warn-leftbar},%
]{better-warn}
\declaretheoremstyle[headfont=\sffamily\bfseries,%
 notefont=\sffamily\bfseries,%
notebraces={}{},%
headpunct=,%
 bodyfont=\sffamily,%
 headformat=\color{exemple-color}Exemple~\NUMBER\hfill\NOTE\smallskip\linebreak,%
 preheadhook=\vspace{\freespace}\begin{exemple-leftbar},%
 postfoothook=\end{exemple-leftbar},%
]{better-exemple}
\declaretheoremstyle[headfont=\sffamily\bfseries,%
 notefont=\sffamily\bfseries,%
 notebraces={}{},%
 headpunct=,%
 bodyfont=\sffamily,%
 headformat=\color{props-color}Proposition~\NUMBER\hfill\NOTE\smallskip\linebreak,%
 preheadhook=\vspace{\freespace}\begin{props-leftbar},%
 postfoothook=\end{props-leftbar},%
]{better-props}
\declaretheoremstyle[headfont=\sffamily\bfseries,%
 notefont=\sffamily\bfseries,%
 notebraces={}{},%
 headpunct=,%
 bodyfont=\sffamily,%
 headformat=\color{props-color}Théorème~\NUMBER\hfill\NOTE\smallskip\linebreak,%
 preheadhook=\vspace{\freespace}\begin{props-leftbar},%
 postfoothook=\end{props-leftbar},%
]{better-thm}
\declaretheoremstyle[headfont=\sffamily\bfseries,%
 notefont=\sffamily\bfseries,%
 notebraces={}{},%
 headpunct=,%
 bodyfont=\sffamily,%
 headformat=\color{corol-color}Corollaire~\NUMBER\hfill\NOTE\smallskip\linebreak,%
 preheadhook=\vspace{\freespace}\begin{corol-leftbar},%
 postfoothook=\end{corol-leftbar},%
]{better-corol}

\declaretheorem[style=better-defn]{defn}
\declaretheorem[style=better-warn]{warn}
\declaretheorem[style=better-exemple]{exemple}
\declaretheorem[style=better-corol]{corol}
\declaretheorem[style=better-props, numberwithin=defn]{props}
\declaretheorem[style=better-thm, sibling=props]{thm}
\newtheorem*{lemme}{Lemme}%[subsection]
%\newtheorem{props}{Propriétés}[defn]

\newenvironment{system}%
{\left\lbrace\begin{align}}%
{\end{align}\right.}

\newenvironment{AQT}{{\fontfamily{qbk}\selectfont AQT}}

\usepackage{LobsterTwo}
\titleformat{\section}{\newpage\LobsterTwo \huge\bfseries}{\thesection.}{1em}{}
\titleformat{\subsection}{\vspace{2em}\LobsterTwo \Large\bfseries}{\thesubsection.}{1em}{}
\titleformat{\subsubsection}{\vspace{1em}\LobsterTwo \large\bfseries}{\thesubsubsection.}{1em}{}

\newenvironment{lititle}%
{\vspace{7mm}\LobsterTwo \large}%
{\\}

\renewenvironment{proof}{\par$\square$ \footnotesize\textit{Démonstration.}}{\begin{flushright}$\blacksquare$\end{flushright}\par}

\title{Relations d'ordre, ensembles ordonnés}
\author{William Hergès\thanks{Sorbonne Université}}

\begin{document}
	\maketitle
	\tableofcontents
	\newpage
    \section{Ensemble ordonné}
    \subsection{Définition}
    \begin{defn}
        Une relation d'ordre $\preceq$ est une relation binaire sur $E$ si et seulement si~:
        \begin{itemize}
            \item réflexive
            \item anti-symétrique
            \item transitive
        \end{itemize}

        L'ordre strict $\prec$ est associé à $\preceq$~: c'est la même, sauf qu'elle n'est pas réflexive~:
        $$ \prec = \preceq\backslash\mathrm{Id}_E $$
    \end{defn}
    \begin{defn}
        Une relation d'ordre $\preceq$ est~:
        \begin{itemize}
            \item totale si et seulement si $\preceq$ permet toujours de comparer deux éléments quelconques de $E$
            \item partielle s'il existe au moins deux éléments de $E$ incomparables avec $\preceq$
        \end{itemize}
    \end{defn}
    \begin{defn}
        $(E,\preceq)$ est un ensemble~:
        \begin{itemize}
            \item totalement ordonné si $\preceq$ est un ordre total.
            \item partiellement ordonné si $\preceq$ est un ordre partiel.
        \end{itemize}
    \end{defn}
    \begin{exemple}
        $(\mathbb{N},\leqslant )$ est un ensemble totalement ordonné.

        $(\mathcal{P}(F),\subseteq)$ est un ensemble partiellement ordoné (pour $F$ un ensemble quelconque).

        $(\mathbb{N}^*, |)$ est aussi partiellement ordoné, où
        $$ | = \{(a,b)|\exists k\in\mathbb{N}^*, b = na\} $$
        (c'est la relation divise.)
    \end{exemple}
    \begin{proof}
        Preuve du deuxième exemple.

        Soit $F$ un ensemble.

        Montrons que $\subseteq$ est un ordre pour $\mathcal{P}(F)$.
        \begin{itemize}
            \item Soit $A\in\mathcal{P}(F)$.
                Triviallement, $A\subseteq A$.
                Alors, $\subseteq$ est réflexive.
            \item Soit $(A,B)\in\mathcal{P}(F)^2$.
                Supposons que $A\subseteq B$ et que $B\subseteq A$.
                Alors, $A=B$ par définition.
            \item Soit $(A,B,C)\in\mathcal{P}(F)^3$ avec $A\subseteq B$ et $B\subseteq C$.
                Si $A$ est l'ensemble vide, il est inclu dans tous les ensembles. 
                Donc $A\subseteq C$.
                Si $A$ n'est pas l'ensemble vide, tous ses éléments sont dans $B$. 
                Or, tous les éléments de $B$ sont dans $C$.
                Donc, tous les éléments de $A$ sont dans $C$.
                Alors, $A\subseteq C$.
        \end{itemize}
        Ainsi, $\subseteq$ est bien un ordre pour $\mathcal{P}(F)$.

        Montrons que $\subseteq$ est un ordre partiel.

        Supposons que $F$ contient au moins deux éléments.

        Soit $(x,y)\in F^2$, deux éléments différents.
        Soient $A=\{x\}$ et $B=\{y\}$.

        On a que $A\not\subseteq B$ et que $B\subseteq A$.
        Donc, $\subseteq$ est partiel dans ce cas.

        Si $F$ est vide, alors $\mathcal{P}(F)$ contient un unique élément.
        Cet ensemble est totalement ordonné.

        Si $F$ est un singleton, alors $\mathcal{P}(F)$ contient $F$ et l'ensemble vide.
        Cet ensemble est totalement ordonné.

        La slide est ainsi fausse, mais les ensembles à moins de deux éléments sont peux intéressants.
    \end{proof}
    \subsection{Représentation d'une relation d'ordre}
    \begin{defn}
        La représentation d'une relation d'ordre $R$ sur un ensemble $E$ est le graphe \textit{minimal} représentant
        une relation $\to$, telle que~:
        \begin{itemize}
            \item la fermeture réflexive et transitive $\to^*$ de $\to$ correspond exactement à la relation $R$
            \item $\to$ est la plus petite relation dont la fermeture réflexo-transitive est égale à la relation $R$
            \item $\to$ contient tous les couples $(a,b)\in R$ tels que $a\neq b$ et que~:
                $$ \forall c\in E, c\neq a, c\neq b,\quad (a,c)\not\in R\lor (c,b)\not\in R $$
        \end{itemize}
        On a donc que $\to$ s'obtient en supprimant de $R$~:
        \begin{itemize}
            \item les couples $(x,x)\in R$ (réflexivité)
            \item les couples pouvant se déduire par transitivité 
        \end{itemize}
        On dit que ce graphe est couvrant.
    \end{defn}
    \begin{props}
        Le coupe $(a,b)$ appartient à $R$ si, et seulement si, il existe un chemin dans le graphe couvrant.
    \end{props}
    \begin{exemple}
        Graphe couvrant de $\leqslant$ sur $\mathbb{N}$~:

        \begin{tikzpicture}
            \node (start) {0};
            \node (1) [right of=start] {$1$};
            \draw [->] (start) -- (1);
            \node (2) [right of=1] {$2$};
            \draw [->] (1) -- (2);
            \node (3) [right of=2] {$3$};
            \draw [->] (2) -- (3);
            \node (4) [right of=3] {$\ldots$};
            \draw [->] (3) -- (4);
        \end{tikzpicture}
    \end{exemple}
    \subsection{Monotonie}
    \begin{defn}
        Soient $(E_1,\preceq_1)$ et $(E_2,\preceq_2)$ deux ensemble ordonnés.

        L'application $f:E_1\to E_2$ est dite monotone si~:
        $$ \forall (x,y)\in E_1^2,\quad x\preceq_1 y \implies f(x)\preceq_2 f(y) $$
    \end{defn}
    Une application monotone préserve les relations d'ordre.
    \begin{exemple}
        On se place dans $(\mathbb{N},\leqslant)$ et dans $(\mathcal{P}(\mathbb{N}))$

        $f:\mathbb{N}\to \mathcal{P}(\mathbb{N})$ tel que $f(n) = \{k\in\mathbb{N}|k\leqslant n\}$ est monotone.

        $g:\mathbb{N}\to \mathcal{P}(\mathbb{N})$ tel que $g(n)=\{n\}$ ne l'est pas par contre~!
    \end{exemple}
    \begin{props}
        Deux ensembles ordonnés $(E_1,\preceq_1)$ et $(E_2,\preceq_2)$ sont isomorphes s'il existe une bijection
        $f:E_1\to E_2$ telle que $f$ et $f^{-1}$ sont monotones.
    \end{props}
    \begin{exemple}
        Soient $(\mathbb{N},\leqslant)$ et $(\mathcal{P}(\mathbb{N}),\subseteq)$ deux ensembles ordonnés.

        $f:\mathbb{N}\to\mathcal{P}(\mathbb{N})$ telle que $f(n) = \{k|k\leqslant n\}$ est monotone.

        $g:\mathbb{N}\to\mathcal{P}(\mathbb{N})$ telle que $g(n) = \{ n\}$ n'est pas monotone.
    \end{exemple}
    \begin{warn}
        Une bijection $f$ peut être monotone sans que $f^{-1}$ ne le soit~!
    \end{warn}
    \subsection{Minimum, maximum, bornes}
    \begin{defn}
        Soit $(E,\preceq)$ un ensemble ordonné et $X$ une partie de $E$.

        Un élément minimal $x$ de $X$ est un élément tel que~:
        $$ \forall y\in X,\quad y\preceq x\implies y=x $$

        Un élément maximale $x$ de $X$ est un tel que~:
        $$ \forall y\in X,\quad x \preceq y \implies x=y $$
    \end{defn}
    \begin{defn}
        On dit que $e\in E$ est un minorant de $X$ si~:
        $$ \forall x\in X,\quad e\preceq x $$

        On dit que $e\in E$ est un majorant de $X$ si~:
        $$ \forall x\in X,\quad x\preceq e $$
    \end{defn}
    La différence avec l'élément minimal, c'est que $e$ n'est pas forcément dans $X$~!
    La différence avec l'élément maximal, c'est que $e$ n'est pas forcément dans $X$~!
    \begin{defn}
        Le plus petit élément (aussi appelé minimum) de $X$, s'il existe, est l'unique élément de l'intersection de $X$
        et de ses minorants.

        Le plus grand élément (aussi appelé maximum) de $X$, s'il existe, est l'unique élément de l'intersection de $X$
        et de ses majorants.
    \end{defn}
    Le minimum est le minorant dans $X$~!
    Le maximum est le majorant dans $X$~!

    \begin{proof}
        Soient $x_1$ et $x_2$ deux minorants (resp. deux majorants) de $X$.

        Par défintion, on a~:
        $$ x_1\preceq x_2\quad\land\quad x_2\preceq x_1$$
        Par anti-symétrie, on obtient $x_1=x_2$.

        Ainsi, le minorant (resp. majorant) est unique.
    \end{proof}

    \begin{defn}
        La borne inférieure de $X$ est le plus grand élément des minorants de $X$ (s'il existe).
        On la note $\inf$.

        La borne supérieure de $X$ est le plus petit élément des majorants de $X$ (s'il existe).
        On la note $\sup$.
    \end{defn}
    \section{Ordre bien fondé}
    \subsection{Relation d'ordre bien fondée et induction}
    \begin{defn}
        Une relation d'ordre $\preceq$ sur un ensemble $E$ est dite bien fondée s'il n'existe pas de suite infinie
        strictement décroissante d'éléments de $E$.
    \end{defn}
    \begin{exemple}
        $\leqslant$ sur $\mathbb{N}$ est une relation bien fondée.
        
        $\leqslant$ sur $\mathbb{Z}$ n'est pas une relation bien fondée.
    \end{exemple}
    \begin{thm}
        La relation d'ordre sur $E$ est bien fondée si, et seulement si, toute partie non vide de $E$ admet un élément
        minimal (pour cet ordre).
    \end{thm}
    \begin{proof}
        Soit $E$ un ensemble ordonné par $\prec$, une relation d'ordre bien fondée.
        Soit $X$ une partie non vide de $E$.

        \fbox{$\implies$}
        Par l'absurde, supposons que $X$ n'admet pas d'élément minimal.

        Comme $X$ n'est pas vide, il existe $x_0$ dans $X$.
        Comme $x_0$ n'est pas minimal, il existe $x_1$ dans $X$.
        On peut ainsi construire \textit{de proche en proche} une suite infinie strictement décroissante, ce qui
        contredit la définition de $\preceq$.

        \fbox{$\impliedby$}
        Si toute partie non vide de $E$ admet un élément minimal, c'est en particulier le cas pour une suite 
        strictement décroissante.
        Soit $(u_n)$ une suite strictement décroissante à valeur dans $X$.
        Soit $p\in\mathbb{N}$ l'indice de l'élément minimal de $(u_n)$.
        
        Tous les éléments d'indice supérieur à $p$ doivent être strictement plus petit que $u_p$, ce qui est impossible.
        $(u_n)$ est donc finie.

        Ainsi, le théorème est vrai.
    \end{proof}
    \begin{thm}[Induction]
        Soit $E$ un ensemble muni d'une relation d'ordre bien fondée $\preceq$ et $P$ une propriété de $E$.

        Si pour tout $x\in E$ et pour tout $y \prec x$ telle que $P(y)$ soit vraie, on a alors que $P$ est vraie pour
        tous les éléments de $E$.
    \end{thm}
    Ceci est une version généralisée de la récurrence.
    \begin{proof}
        Soit $X$ l'ensemble des $x$ tels que $P(x)$ soit faux.

        Si $X$ est non vide, $X$ admet un élément minimal (car $\preceq$ est bien fondée).
        Donc, tous les $y$ strictement plus petits que $x$ sont vrais.
        En utilisant l'hypothèse, $P(x)$ est aussi vraie.
        $X$ est donc vide.

        Ainsi, pour tout $e\in E$, $P(e)$ est vraie.
    \end{proof}
    \begin{lititle}
        Démonstration utilisant l'induction
    \end{lititle}
    Elle fonctionne de la même manière qu'une récurrence~:
    \begin{itemize}
        \item Si $x$ est un élément minimal, on démontre $P(x)$ sans aucune hypothèse.
        \item On suppose $P(y)$ pour tous les éléments plus petit que $x$ et on démontre $P(x)$.
    \end{itemize}
    \begin{exemple}
        Toutes les démonstrations par récurrence sur $(\mathbb{N},\leqslant)$ sont des inductions~!
    \end{exemple}
    \subsection{Ordre lexicographique}
    \begin{defn}
        Soient $(E_1,\preceq_1)\ldots(E_n,\preceq_n)$ des ensembles ordonnés.

        La relation d'ordre lexicographique $\preceq$ sur $E_1\times\ldots\times E_n$ est définie par~:
        $$ \exists i<n,\forall k<i, \quad (e_k=f_k\land e_i\preceq f_i)\quad\lor\quad(e_1,\ldots,e_n) = (f_1,\ldots,f_n)$$
        avec $(e_1,\ldots,e_n)\preceq(f_1,\ldots,f_n)$
    \end{defn}
    C'est-à-dire, soit ils sont tous égaux, soit il existe un indice $i$ où $e_i \preceq f_i$.
    \begin{props}
        L'ordre lexicographique est une relation d'ordre.
    \end{props}
    On a bien choisi son nom :D
    \begin{exemple}
        Flemme de recopier des exemples, voir le diapo 24 (page 46).
    \end{exemple}
    \begin{thm}
        L'ordre lexicographique est bien fondée si $(\preceq_1,\ldots,\preceq_n)$ sont bien fondés.
    \end{thm}
    L'ordre du dictionnaire n'est pas bien fondée par contre.
\end{document}
