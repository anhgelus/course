%%=====================================================================================
%%
%%       Filename:  cours.tex
%%
%%    Description:  
%%
%%        Version:  1.0
%%        Created:  03/06/2024
%%       Revision:  none
%%
%%         Author:  YOUR NAME (), 
%%   Organization:  
%%      Copyright:  Copyright (c) 2024, YOUR NAME
%%
%%          Notes:  
%%
%%=====================================================================================
\documentclass[a4paper, titlepage]{article}

\usepackage[utf8]{inputenc}
\usepackage[T1]{fontenc}
\usepackage{textcomp}
\usepackage[french]{babel}
\usepackage{amsmath, amssymb}
\usepackage{amsthm}
\usepackage[svgnames]{xcolor}
\usepackage{thmtools}
\usepackage{lipsum}
\usepackage{framed}
\usepackage{parskip}
\usepackage{titlesec}
\usepackage{newtxtext}

% \renewcommand{\familydefault}{\sfdefault}

% figure support
\usepackage{import}
\usepackage{xifthen}
\pdfminorversion=7
\usepackage{pdfpages}
\usepackage{transparent}
\newcommand{\incfig}[1]{%
	\def\svgwidth{\columnwidth}
	\import{./figures/}{#1.pdf_tex}
}

\pdfsuppresswarningpagegroup=1

\colorlet{defn-color}{DarkBlue}
\colorlet{props-color}{Blue}
\colorlet{warn-color}{Red}
\colorlet{exemple-color}{Green}
\colorlet{corol-color}{Orange}
\newenvironment{defn-leftbar}{%
  \def\FrameCommand{{\color{defn-color}\vrule width 3pt} \hspace{10pt}}%
  \MakeFramed {\advance\hsize-\width \FrameRestore}}%
 {\endMakeFramed}
\newenvironment{warn-leftbar}{%
  \def\FrameCommand{{\color{warn-color}\vrule width 3pt} \hspace{10pt}}%
  \MakeFramed {\advance\hsize-\width \FrameRestore}}%
 {\endMakeFramed}
\newenvironment{exemple-leftbar}{%
  \def\FrameCommand{{\color{exemple-color}\vrule width 3pt} \hspace{10pt}}%
  \MakeFramed {\advance\hsize-\width \FrameRestore}}%
 {\endMakeFramed}
\newenvironment{props-leftbar}{%
  \def\FrameCommand{{\color{props-color}\vrule width 3pt} \hspace{10pt}}%
  \MakeFramed {\advance\hsize-\width \FrameRestore}}%
 {\endMakeFramed}
\newenvironment{corol-leftbar}{%
  \def\FrameCommand{{\color{corol-color}\vrule width 3pt} \hspace{10pt}}%
  \MakeFramed {\advance\hsize-\width \FrameRestore}}%
 {\endMakeFramed}

\def \freespace {1em}
\declaretheoremstyle[headfont=\sffamily\bfseries,%
 notefont=\sffamily\bfseries,%
 notebraces={}{},%
 headpunct=,%
% bodyfont=\sffamily,%
 headformat=\color{defn-color}Définition~\NUMBER\hfill\NOTE\smallskip\linebreak,%
 preheadhook=\vspace{\freespace}\begin{defn-leftbar},%
 postfoothook=\end{defn-leftbar},%
]{better-defn}
\declaretheoremstyle[headfont=\sffamily\bfseries,%
 notefont=\sffamily\bfseries,%
 notebraces={}{},%
 headpunct=,%
% bodyfont=\sffamily,%
 headformat=\color{warn-color}Attention\hfill\NOTE\smallskip\linebreak,%
 preheadhook=\vspace{\freespace}\begin{warn-leftbar},%
 postfoothook=\end{warn-leftbar},%
]{better-warn}
\declaretheoremstyle[headfont=\sffamily\bfseries,%
 notefont=\sffamily\bfseries,%
notebraces={}{},%
headpunct=,%
% bodyfont=\sffamily,%
 headformat=\color{exemple-color}Exemple~\NUMBER\hfill\NOTE\smallskip\linebreak,%
 preheadhook=\vspace{\freespace}\begin{exemple-leftbar},%
 postfoothook=\end{exemple-leftbar},%
]{better-exemple}
\declaretheoremstyle[headfont=\sffamily\bfseries,%
 notefont=\sffamily\bfseries,%
 notebraces={}{},%
 headpunct=,%
% bodyfont=\sffamily,%
 headformat=\color{props-color}Proposition~\NUMBER\hfill\NOTE\smallskip\linebreak,%
 preheadhook=\vspace{\freespace}\begin{props-leftbar},%
 postfoothook=\end{props-leftbar},%
]{better-props}
\declaretheoremstyle[headfont=\sffamily\bfseries,%
 notefont=\sffamily\bfseries,%
 notebraces={}{},%
 headpunct=,%
% bodyfont=\sffamily,%
 headformat=\color{props-color}Théorème~\NUMBER\hfill\NOTE\smallskip\linebreak,%
 preheadhook=\vspace{\freespace}\begin{props-leftbar},%
 postfoothook=\end{props-leftbar},%
]{better-thm}
\declaretheoremstyle[headfont=\sffamily\bfseries,%
 notefont=\sffamily\bfseries,%
 notebraces={}{},%
 headpunct=,%
% bodyfont=\sffamily,%
 headformat=\color{corol-color}Corollaire~\NUMBER\hfill\NOTE\smallskip\linebreak,%
 preheadhook=\vspace{\freespace}\begin{corol-leftbar},%
 postfoothook=\end{corol-leftbar},%
]{better-corol}

\declaretheorem[style=better-defn]{defn}
\declaretheorem[style=better-warn]{warn}
\declaretheorem[style=better-exemple]{exemple}
\declaretheorem[style=better-corol]{corol}
\declaretheorem[style=better-props, numberwithin=defn]{props}
\declaretheorem[style=better-thm, sibling=props]{thm}
\newtheorem*{lemme}{Lemme}%[subsection]
%\newtheorem{props}{Propriétés}[defn]

\newenvironment{system}%
{\left\lbrace\begin{align}}%
{\end{align}\right.}

\newenvironment{AQT}{{\fontfamily{qbk}\selectfont AQT}}

\usepackage{LobsterTwo}
\titleformat{\section}{\newpage\LobsterTwo \huge\bfseries}{\thesection.}{1em}{}
\titleformat{\subsection}{\vspace{2em}\LobsterTwo \Large\bfseries}{\thesubsection.}{1em}{}
\titleformat{\subsubsection}{\vspace{1em}\LobsterTwo \large\bfseries}{\thesubsubsection.}{1em}{}

\newenvironment{lititle}%
{\vspace{7mm}\LobsterTwo \large}%
{\\}

\renewenvironment{proof}{$\square$ \footnotesize\textit{Démonstration.}}{\begin{flushright}$\blacksquare$\end{flushright}}

\title{Probabilités}
\author{William Hergès\thanks{Sorbonne Université - Faculté des Sciences, Faculté des Lettres}}

\begin{document}
	\maketitle
	\tableofcontents
	\newpage
	\section{Espace probabilisé}
	Les probabilités sont très semblables à la théorie des ensembles.
	\begin{defn}
		On note $\Omega$ l'ensemble des issues possibles d'une expérience aléatoire. $\Omega$ est l'univers.
	\end{defn}

	\begin{lititle}
		Dictionnaire des probabilités
	\end{lititle}
	Cette petite partie est piquée de mon cours de maths au CPES (créé par M. Kerner et M. Cote, deux professeurs de mathématiques au Lycée Henri-IV).
	\begin{table}[htpb]
		\centering
		\caption{Dictionnaire}
		\label{tab:dico-proba}
		\begin{tabular}{|c|c|}
			\hline
			Théorie des ensembles & Probabilités \\
			\hline
			$w\in\Omega$ & issue de l'expérience \\
			$A\in\mathcal{P}(\Omega)$ & événement \\
			$\mathcal{P}(\Omega)$ ou l'ensemble des parties & tribu \\
			$\bar A = \Omega\backslash A$ ou complémentaire & contraire \\
			$A\cup B$ & $A$ ou $B$ \\
			$A\cap B$ & $A$ et $B$ \\
			$B\backslash A$ & $B$ mais pas $A$\\
			$A\subset B$ & $A$ implique $B$ \\
			$A\cap B = \varnothing$ ou $A$ et $B$ sont disjoints & $A$ et $B$ sont incompatibles \\
			$A\sqcup B$ & $A$ ou bien $B$ \\
			$E_1\sqcup E_2\sqcup\ldots\sqcup E_n = \Omega$ ou un partage & système complet d'évènements (s.e.c.)\\
			\hline
		\end{tabular}
	\end{table}

	\begin{defn}
		Un espace probabilisé est un univers $\Omega$ possédant une fonction $\mathbb{P}$ de $\mathcal{P}(\Omega)$ dans $[0;1]$ tel que
		\begin{enumerate}
			\item $\mathbb{P}(\Omega)=1$
			\item $\mathbb{P}(A\cup B)=\mathbb{P}(A)+\mathbb{P}(B)$, où $A$ et $B$ sont deux événements de $\Omega$ incompatibles.
		\end{enumerate}
	\end{defn}
	Un espace probabilisé est donc un univers avec une fonction assignant une probabilité à tous les événements de l'univers~!
	\begin{props}
		On a :
		$$ \mathbb{P}(\varnothing) = 0 $$
	\end{props}
	\begin{proof}
		On a :
		$$ \mathbb{P}(\Omega) = \mathbb{P}(\varnothing\cup\Omega) = \mathbb{P}(\varnothing) + \mathbb{P}(\Omega) = 1 $$
		Donc $\mathbb{P}(\varnothing) = 0$, car, par définition, $\mathbb{P}(\Omega) = 1$.
	\end{proof}
	\begin{props}
		On a pour tous événements $A$ et $B$ de $\Omega$, un espace probabilisé~:
		$$ \mathbb{P}(A\cup B) = \mathbb{P}(A) + \mathbb{P}(B) + \mathbb{P}(A\cap B) $$
	\end{props}
	\begin{proof}
		\AQT
	\end{proof}
	\section{Probabilités conditionnelles}
	\begin{defn}
		On dit que $A\subset\Omega$ est certain si et seulement si $\mathbb{P}(A) = 1$.

		On dit que $B\subset\Omega$ est impossible si et seulement si $\mathbb{P}(B) = 0$.
	\end{defn}
	\begin{props}
		À partir d'un événement $B$ non-impossible, on peut définir un espace probabilisé.

		Soit $(\Omega,\mathbb{P})$ un espace probabilisé. Soit $B$ un événement non-impossible de $(\Omega,\mathbb{P})$ (i.e. $\mathbb{P}(B)\neq 0$). On a que $(\Omega,\mathbb{P}_B)$ est un espace probabilisé avec $\mathbb{P}_B$ de $\mathcal{P}(\Omega)$ dans $[0;1]$ tel que :
		$$ \mathbb{P}_B(A) = \frac{\mathbb{P}(A\cap B)}{\mathbb{P}(B)} $$
		où $A$ est un événement de $\Omega$. $\mathbb{P}_B(A)$ est la probabilité de $A$ sachant $B$.
	\end{props}
	\begin{proof}
		\AQT
	\end{proof}
	On peut aussi dire que $\mathbb{P}_B(A)$ est la probabilité de $A$ conditionnelle à $B$. 

	On peut aussi noter $\mathbb{P}_B(A) = \mathbb{P}(A|B)$.
	\section{Événements indépendants}
	\begin{defn}
		Soient $A$ et $B$ deux événements de l'espace probabilisé $(\Omega,\mathbb{P})$.

		On dit qu'ils sont indépendants si et seulement si~:
		$$ \mathbb{P}(A\cap B) = \mathbb{P}(A)\times\mathbb{P}(B) $$
	\end{defn}
	\begin{props}
		Si $A$ et $B$ sont indépendants et si $B$ n'est pas impossible, alors $$ \mathbb{P}_B(A) = \mathbb{P}(A) $$
	\end{props}
	\begin{proof}
		\AQT
	\end{proof}
	\begin{thm}[Théorème de Bayes]
		On note $\Omega$ l'union des $(C_i)_{i\in[|1,n|]}$ disjoints deux à deux.

		On note $R$ un événement de $\Omega$ sachant $\mathbb{P}_{C_i}(R)$ pour tout $i$ dans $[|i,n|]$.

		On a~:
		$$ \forall i\in[|1,n|],\quad\mathbb{P}_{R}(C_i) = \frac{\mathbb{P}_{C_i}(R)\mathbb{P}(C_i)}{\displaystyle \sum_{i=1}^{n} \mathbb{P}_{C_i}(R)\mathbb{P}(C_i)}$$
	\end{thm}
	\begin{exemple}
		On veut $\mathbb{P}_D(I)$. On sait que~:
		\begin{enumerate}
			\item $\mathbb{P}_F(D) = 1/2$
			\item $\mathbb{P}_I(D) = 3/4$
			\item $\mathbb{P}_A(D) = 1/4$
		\end{enumerate}
		On sait aussi que~:
		\begin{enumerate}
			\item $\mathbb{P}(F)=1/4$
			\item $\mathbb{P}(I)=1/4$
			\item $\mathbb{P}(A)=1/2$
		\end{enumerate}
	Donc~:
	\begin{align*}
		\mathbb{P}_D(I) &= \frac{\mathbb{P}_I(D)\mathbb{P}(I)}{\mathbb{P}_F(D)\mathbb{P}(F)+\mathbb{P}_I(D)\mathbb{P}(I)+\mathbb{P}_A(D)\mathbb{P}(A)} \\
						&= \frac{\frac{3}{4}\cdot \frac{1}{4}}{\frac{1}{4}\cdot \frac{1}{4}+\frac{3}{4}\cdot \frac{1}{4}+\frac{1}{4}\cdot \frac{1}{2}} \\
						&= \frac{3}{7}
	\end{align*}
	\end{exemple}
\end{document}
