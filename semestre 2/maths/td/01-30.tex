%%=====================================================================================
%%
%%       Filename:  cours.tex
%%
%%    Description:  
%%
%%        Version:  1.0
%%        Created:  03/06/2024
%%       Revision:  none
%%
%%         Author:  YOUR NAME (), 
%%   Organization:  
%%      Copyright:  Copyright (c) 2024, YOUR NAME
%%
%%          Notes:  
%%
%%=====================================================================================
\documentclass[a4paper, titlepage]{article}

\usepackage[utf8]{inputenc}
\usepackage[T1]{fontenc}
\usepackage{textcomp}
\usepackage[french]{babel}
\usepackage{amsmath, amssymb}
\usepackage{amsthm}
\usepackage[svgnames]{xcolor}
\usepackage{thmtools}
\usepackage{lipsum}
\usepackage{framed}
\usepackage{parskip}
\usepackage{titlesec}

\renewcommand{\familydefault}{\sfdefault}

% figure support
\usepackage{import}
\usepackage{xifthen}
\pdfminorversion=7
\usepackage{pdfpages}
\usepackage{transparent}
\newcommand{\incfig}[1]{%
	\def\svgwidth{\columnwidth}
	\import{./figures/}{#1.pdf_tex}
}

\pdfsuppresswarningpagegroup=1

\colorlet{defn-color}{DarkBlue}
\colorlet{props-color}{Blue}
\colorlet{warn-color}{Red}
\colorlet{exemple-color}{Green}
\colorlet{corol-color}{Orange}
\newenvironment{defn-leftbar}{%
  \def\FrameCommand{{\color{defn-color}\vrule width 3pt} \hspace{10pt}}%
  \MakeFramed {\advance\hsize-\width \FrameRestore}}%
 {\endMakeFramed}
\newenvironment{warn-leftbar}{%
  \def\FrameCommand{{\color{warn-color}\vrule width 3pt} \hspace{10pt}}%
  \MakeFramed {\advance\hsize-\width \FrameRestore}}%
 {\endMakeFramed}
\newenvironment{exemple-leftbar}{%
  \def\FrameCommand{{\color{exemple-color}\vrule width 3pt} \hspace{10pt}}%
  \MakeFramed {\advance\hsize-\width \FrameRestore}}%
 {\endMakeFramed}
\newenvironment{props-leftbar}{%
  \def\FrameCommand{{\color{props-color}\vrule width 3pt} \hspace{10pt}}%
  \MakeFramed {\advance\hsize-\width \FrameRestore}}%
 {\endMakeFramed}
\newenvironment{corol-leftbar}{%
  \def\FrameCommand{{\color{corol-color}\vrule width 3pt} \hspace{10pt}}%
  \MakeFramed {\advance\hsize-\width \FrameRestore}}%
 {\endMakeFramed}

\def \freespace {1em}
\declaretheoremstyle[headfont=\sffamily\bfseries,%
 notefont=\sffamily\bfseries,%
 notebraces={}{},%
 headpunct=,%
 bodyfont=\sffamily,%
 headformat=\color{defn-color}Définition~\NUMBER\hfill\NOTE\smallskip\linebreak,%
 preheadhook=\vspace{\freespace}\begin{defn-leftbar},%
 postfoothook=\end{defn-leftbar},%
]{better-defn}
\declaretheoremstyle[headfont=\sffamily\bfseries,%
 notefont=\sffamily\bfseries,%
 notebraces={}{},%
 headpunct=,%
 bodyfont=\sffamily,%
 headformat=\color{warn-color}Attention\hfill\NOTE\smallskip\linebreak,%
 preheadhook=\vspace{\freespace}\begin{warn-leftbar},%
 postfoothook=\end{warn-leftbar},%
]{better-warn}
\declaretheoremstyle[headfont=\sffamily\bfseries,%
 notefont=\sffamily\bfseries,%
notebraces={}{},%
headpunct=,%
 bodyfont=\sffamily,%
 headformat=\color{exemple-color}Exemple~\NUMBER\hfill\NOTE\smallskip\linebreak,%
 preheadhook=\vspace{\freespace}\begin{exemple-leftbar},%
 postfoothook=\end{exemple-leftbar},%
]{better-exemple}
\declaretheoremstyle[headfont=\sffamily\bfseries,%
 notefont=\sffamily\bfseries,%
 notebraces={}{},%
 headpunct=,%
 bodyfont=\sffamily,%
 headformat=\color{props-color}Proposition~\NUMBER\hfill\NOTE\smallskip\linebreak,%
 preheadhook=\vspace{\freespace}\begin{props-leftbar},%
 postfoothook=\end{props-leftbar},%
]{better-props}
\declaretheoremstyle[headfont=\sffamily\bfseries,%
 notefont=\sffamily\bfseries,%
 notebraces={}{},%
 headpunct=,%
 bodyfont=\sffamily,%
 headformat=\color{props-color}Théorème~\NUMBER\hfill\NOTE\smallskip\linebreak,%
 preheadhook=\vspace{\freespace}\begin{props-leftbar},%
 postfoothook=\end{props-leftbar},%
]{better-thm}
\declaretheoremstyle[headfont=\sffamily\bfseries,%
 notefont=\sffamily\bfseries,%
 notebraces={}{},%
 headpunct=,%
 bodyfont=\sffamily,%
 headformat=\color{corol-color}Corollaire~\NUMBER\hfill\NOTE\smallskip\linebreak,%
 preheadhook=\vspace{\freespace}\begin{corol-leftbar},%
 postfoothook=\end{corol-leftbar},%
]{better-corol}

\declaretheorem[style=better-defn]{defn}
\declaretheorem[style=better-warn]{warn}
\declaretheorem[style=better-exemple]{exemple}
\declaretheorem[style=better-corol]{corol}
\declaretheorem[style=better-props, numberwithin=defn]{props}
\declaretheorem[style=better-thm, sibling=props]{thm}
\newtheorem*{lemme}{Lemme}%[subsection]
%\newtheorem{props}{Propriétés}[defn]

\newenvironment{system}%
{\left\lbrace\begin{align}}%
{\end{align}\right.}

\newenvironment{AQT}{{\fontfamily{qbk}\selectfont AQT}}

\usepackage{LobsterTwo}
\titleformat{\section}{\newpage\LobsterTwo \huge\bfseries}{\thesection.}{1em}{}
\titleformat{\subsection}{\vspace{2em}\LobsterTwo \Large\bfseries}{\thesubsection.}{1em}{}
\titleformat{\subsubsection}{\vspace{1em}\LobsterTwo \large\bfseries}{\thesubsubsection.}{1em}{}

\newenvironment{lititle}%
{\vspace{7mm}\LobsterTwo \large}%
{\\}

\renewenvironment{proof}{$\square$ \footnotesize\textit{Démonstration.}}{\begin{flushright}$\blacksquare$\end{flushright}}

\title{TD du 30 janvier}
\author{William Hergès\thanks{Sorbonne Université - Faculté des Sciences, Faculté des Lettres}}

\begin{document}
	\maketitle
	\tableofcontents
	\newpage
	\section*{Exercice 1}
	\begin{enumerate}
		\item On a :
			$$ \begin{pmatrix} 1&1\\2&-3 \end{pmatrix} \begin{pmatrix} x\\y \end{pmatrix} = \begin{pmatrix} 0\\0 \end{pmatrix} $$
			$$ \begin{pmatrix} 0&-2&-3\\ -2&3&0\\3&0&2 \end{pmatrix} \begin{pmatrix} x\\y\\z \end{pmatrix} = \begin{pmatrix} 1\\2\\3 \end{pmatrix} $$
			$$ \begin{pmatrix} 1&1&1&0\\1&1&0&1\\0&0&1&t \end{pmatrix} \begin{pmatrix} x\\y\\z\\t \end{pmatrix} = \begin{pmatrix} 0\\-1\\1 \end{pmatrix} $$
		\item On fait l'opération inverse (flemme de le faire).
	\end{enumerate}
	\section*{Exercice 2}
	\begin{enumerate}
		\item $y=\frac{3}{2}$ et $x=\frac{1}{5}$
		\item $\begin{pmatrix} -1&1&|~1\\ 1&1&|~2 \end{pmatrix}$
		\item En faisant $l_1+l_2$, on obtient $2y=3$, ce qui nous donne les mêmes $x$ et $y$
		\item Son rang est $2$.
	\end{enumerate}
	\section*{Exercice 3}
	Refaire ces transformations chez soi. $S_1$ n'a pas de solution (car on a une ligne $0\neq 0$). $S_2$ possède une infinité de solutions. $S_3$ possède une infinité de solutions.

	On a que la quatrième ligne est la somme des deux premières, on peut donc la supprimer en écrivant $0=0$.
			$$ \begin{pmatrix} 1&-1&1&1&|~3\\ 5&2&-1&-3&|~5\\ -3&-4&3&2&|~1\\ 0&0&0&0&|~0 \end{pmatrix} \iff \begin{pmatrix} 1&-1&1&1&|~3\\ 0&7&-6&-8&|~-10\\ 0&-7&6&5&|~10\\ 0&0&0&0&|~0 \end{pmatrix}  $$
			$$ \iff\begin{pmatrix} 1&-1&1&1&|~3\\ 0&7&-6&-8&|~-10\\ 0&0&0&-3&|~0\\ 0&0&0&0&|~0 \end{pmatrix}$$

	L'ensemble solution de $S_3$ est donc $\left\{ \left( \frac{11-z}{7},\frac{6z-10}{7}, z, 0 \right), z\in\mathbb{R} \right\} $
	
	Technique de la matrice échelonnée réduite pour trouver les inconnues :
	\begin{enumerate}
		\item on fait en sorte que tous les pivots soient égaux à $1$ ;
		\item on refait des combinaisons linéaires pour que tous les coefficiants (à part les pivots) soient nuls (c'est plus simple de commencer par le dernier pivot).
	\end{enumerate}
	Il n'est pas possible de faire en sorte que tous les coefficiants soient nuls si le rang de la matrice n'est pas assez élevée. Dans ce cas, on a que les coefficiants nuls doivent forcément être ceux dans les colonnes des pivots.
	\section*{Exercice 5}
	\begin{enumerate}
		\item $\begin{pmatrix} -1&-2&3&1&|~b_1\\1&1&2&2&|~b_2\\3&4&1&3&|~b_3 \end{pmatrix} $
		\item On a que $L_1 = 2L_2-L_3$, donc $b_1=2b_2-b_3$ si le système admet au moins une solution.
		\item On a donc que $\begin{pmatrix} 1&1&2&2&|~b_2\\3&4&1&3&|~b_3\\0&0&0&0&|~~0 \end{pmatrix}\iff \begin{pmatrix} 1&1&2&2&|~b_2\\0&1&-5&-3&|~b_3-3b_2\\0&0&0&0&|~~0 \end{pmatrix} $. Donc, $x_3$ et $x_4$ sont libres.
			$$x_2 = b_3-3b_2+5x_3+3x_4$$
			et
			$$ x_1 = b_3+4b_2-7x_3-6x_4 $$
			$$ S = \left\{ (b_4+4b_2-7x_3-6x_4,b_3-3b_2+5x_3+3x_4,x_3,x_4), (x_3,x_4)\in\mathbb{R}^2 \right\}  $$
		\item $\mathrm{rg}(S) = 2$
	\end{enumerate}
	\section*{Exercice 8}
	Si $A^{-1}$ existe, alors il existe $v_1,v_2$ et $v_3$ tel que $Av_1 = \begin{pmatrix} 1\\0\\0 \end{pmatrix}$, $Av_2 = \begin{pmatrix} 0\\1\\0 \end{pmatrix}$ et $Av_3 = \begin{pmatrix} 0\\0\\1 \end{pmatrix}$.

	Résolvons $\begin{pmatrix} 1&1&1&|~b_1\\2&1&1&|~b_2\\1&2&1&|~b_3 \end{pmatrix}$. On a donc :
	$$ \begin{pmatrix} 1&1&1&|~b_1\\2&1&1&|~b_2\\1&2&1&|~b_3 \end{pmatrix} = \begin{pmatrix} 1&1&1&|~b_1\\0&-1&-1&|~b_2-2b_1\\0&1&0&|~b_3-b_1 \end{pmatrix} = \begin{pmatrix} 1&1&1&|~b_1\\0&0&-1&|~b_2-3b_1+b_3\\0&1&0&|~b_3-b_1 \end{pmatrix} $$
	Donc, $y=b_3-b_1$, $z=-b_2+3b_1-b_3$ et $x=-b_2-b_1$.
	Ainsi,
	$$ v_1= \begin{pmatrix} -1\\-1\\3 \end{pmatrix}\quad v_2= \begin{pmatrix} -1\\0\\-1 \end{pmatrix} \quad v_3 = \begin{pmatrix} 0\\1\\-1 \end{pmatrix}  $$
	D'où $$ A^{-1} = \begin{pmatrix} -1&-1&0\\-1&0&1\\3&-1&-1 \end{pmatrix}  $$
	\section*{Exercice 11}
	Le système associé à ce problème est :
	$$ \begin{pmatrix} 3&4&5&|~66\\7&4&3&|~74\\8&8&9&|136 \end{pmatrix}  $$
\end{document}
