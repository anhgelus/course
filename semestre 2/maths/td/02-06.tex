%%=====================================================================================
%%
%%       Filename:  cours.tex
%%
%%    Description:  
%%
%%        Version:  1.0
%%        Created:  03/06/2024
%%       Revision:  none
%%
%%         Author:  YOUR NAME (), 
%%   Organization:  
%%      Copyright:  Copyright (c) 2024, YOUR NAME
%%
%%          Notes:  
%%
%%=====================================================================================
\documentclass[a4paper, titlepage]{article}

\usepackage[utf8]{inputenc}
\usepackage[T1]{fontenc}
\usepackage{textcomp}
\usepackage[french]{babel}
\usepackage{amsmath, amssymb}
\usepackage{amsthm}
\usepackage[svgnames]{xcolor}
\usepackage{thmtools}
\usepackage{lipsum}
\usepackage{framed}
\usepackage{parskip}
\usepackage{titlesec}

\renewcommand{\familydefault}{\sfdefault}

% figure support
\usepackage{import}
\usepackage{xifthen}
\pdfminorversion=7
\usepackage{pdfpages}
\usepackage{transparent}
\newcommand{\incfig}[1]{%
	\def\svgwidth{\columnwidth}
	\import{./figures/}{#1.pdf_tex}
}

\pdfsuppresswarningpagegroup=1

\colorlet{defn-color}{DarkBlue}
\colorlet{props-color}{Blue}
\colorlet{warn-color}{Red}
\colorlet{exemple-color}{Green}
\colorlet{corol-color}{Orange}
\newenvironment{defn-leftbar}{%
  \def\FrameCommand{{\color{defn-color}\vrule width 3pt} \hspace{10pt}}%
  \MakeFramed {\advance\hsize-\width \FrameRestore}}%
 {\endMakeFramed}
\newenvironment{warn-leftbar}{%
  \def\FrameCommand{{\color{warn-color}\vrule width 3pt} \hspace{10pt}}%
  \MakeFramed {\advance\hsize-\width \FrameRestore}}%
 {\endMakeFramed}
\newenvironment{exemple-leftbar}{%
  \def\FrameCommand{{\color{exemple-color}\vrule width 3pt} \hspace{10pt}}%
  \MakeFramed {\advance\hsize-\width \FrameRestore}}%
 {\endMakeFramed}
\newenvironment{props-leftbar}{%
  \def\FrameCommand{{\color{props-color}\vrule width 3pt} \hspace{10pt}}%
  \MakeFramed {\advance\hsize-\width \FrameRestore}}%
 {\endMakeFramed}
\newenvironment{corol-leftbar}{%
  \def\FrameCommand{{\color{corol-color}\vrule width 3pt} \hspace{10pt}}%
  \MakeFramed {\advance\hsize-\width \FrameRestore}}%
 {\endMakeFramed}

\def \freespace {1em}
\declaretheoremstyle[headfont=\sffamily\bfseries,%
 notefont=\sffamily\bfseries,%
 notebraces={}{},%
 headpunct=,%
 bodyfont=\sffamily,%
 headformat=\color{defn-color}Définition~\NUMBER\hfill\NOTE\smallskip\linebreak,%
 preheadhook=\vspace{\freespace}\begin{defn-leftbar},%
 postfoothook=\end{defn-leftbar},%
]{better-defn}
\declaretheoremstyle[headfont=\sffamily\bfseries,%
 notefont=\sffamily\bfseries,%
 notebraces={}{},%
 headpunct=,%
 bodyfont=\sffamily,%
 headformat=\color{warn-color}Attention\hfill\NOTE\smallskip\linebreak,%
 preheadhook=\vspace{\freespace}\begin{warn-leftbar},%
 postfoothook=\end{warn-leftbar},%
]{better-warn}
\declaretheoremstyle[headfont=\sffamily\bfseries,%
 notefont=\sffamily\bfseries,%
notebraces={}{},%
headpunct=,%
 bodyfont=\sffamily,%
 headformat=\color{exemple-color}Exemple~\NUMBER\hfill\NOTE\smallskip\linebreak,%
 preheadhook=\vspace{\freespace}\begin{exemple-leftbar},%
 postfoothook=\end{exemple-leftbar},%
]{better-exemple}
\declaretheoremstyle[headfont=\sffamily\bfseries,%
 notefont=\sffamily\bfseries,%
 notebraces={}{},%
 headpunct=,%
 bodyfont=\sffamily,%
 headformat=\color{props-color}Proposition~\NUMBER\hfill\NOTE\smallskip\linebreak,%
 preheadhook=\vspace{\freespace}\begin{props-leftbar},%
 postfoothook=\end{props-leftbar},%
]{better-props}
\declaretheoremstyle[headfont=\sffamily\bfseries,%
 notefont=\sffamily\bfseries,%
 notebraces={}{},%
 headpunct=,%
 bodyfont=\sffamily,%
 headformat=\color{props-color}Théorème~\NUMBER\hfill\NOTE\smallskip\linebreak,%
 preheadhook=\vspace{\freespace}\begin{props-leftbar},%
 postfoothook=\end{props-leftbar},%
]{better-thm}
\declaretheoremstyle[headfont=\sffamily\bfseries,%
 notefont=\sffamily\bfseries,%
 notebraces={}{},%
 headpunct=,%
 bodyfont=\sffamily,%
 headformat=\color{corol-color}Corollaire~\NUMBER\hfill\NOTE\smallskip\linebreak,%
 preheadhook=\vspace{\freespace}\begin{corol-leftbar},%
 postfoothook=\end{corol-leftbar},%
]{better-corol}

\declaretheorem[style=better-defn]{defn}
\declaretheorem[style=better-warn]{warn}
\declaretheorem[style=better-exemple]{exemple}
\declaretheorem[style=better-corol]{corol}
\declaretheorem[style=better-props, numberwithin=defn]{props}
\declaretheorem[style=better-thm, sibling=props]{thm}
\newtheorem*{lemme}{Lemme}%[subsection]
%\newtheorem{props}{Propriétés}[defn]

\newenvironment{system}%
{\left\lbrace\begin{align}}%
{\end{align}\right.}

\newenvironment{AQT}{{\fontfamily{qbk}\selectfont AQT}}

\usepackage{LobsterTwo}
\titleformat{\section}{\newpage\LobsterTwo \huge\bfseries}{\thesection.}{1em}{}
\titleformat{\subsection}{\vspace{2em}\LobsterTwo \Large\bfseries}{\thesubsection.}{1em}{}
\titleformat{\subsubsection}{\vspace{1em}\LobsterTwo \large\bfseries}{\thesubsubsection.}{1em}{}

\newenvironment{lititle}%
{\vspace{7mm}\LobsterTwo \large}%
{\\}

\renewenvironment{proof}{$\square$ \footnotesize\textit{Démonstration.}}{\begin{flushright}$\blacksquare$\end{flushright}}

\title{TD du 6 février}
\author{William Hergès\thanks{Sorbonne Université - Faculté des Sciences, Faculté des Lettres}}

\begin{document}
	\maketitle
	\section*{Exercice 1}
	$$ \begin{pmatrix} -3&5&6&|&1&0&0\\-1&2&2&|&0&1&0\\ 1&-1&-1&|&0&0&1 \end{pmatrix}  $$
	$$ \begin{pmatrix} 0&2&3&|&1&0&3\\0&1&1&|&0&1&1\\ 1&-1&-1&|&0&0&1 \end{pmatrix}  $$
	$$ \begin{pmatrix} 0&0&1&|&1&-2&1\\0&1&1&|&0&1&1\\ 1&-1&-1&|&0&0&1 \end{pmatrix}  $$
	$$ \begin{pmatrix} 0&0&1&|&1&-2&1\\0&1&0&|&-1&3&0\\ 1&0&0&|&0&1&2 \end{pmatrix}  $$
	$$ \begin{pmatrix} 1&0&0&|&0&1&2\\0&1&0&|&-1&3&0\\ 0&0&1&|&1&-2&1 \end{pmatrix}  $$
	On a donc que :
	$$ \begin{pmatrix} 0&1&2\\-1&3&0\\1&-2&1 \end{pmatrix}  $$ est l'inverse de la matrice $A$.

	\begin{lititle}
		Comment passer de la matrice échelonnée à la matrice identité~?
	\end{lititle}
	On applique la méthode vu la semaine dernière~: on passe à la matrice échelonnée réduite.
	\section*{Exercice 3}
	$\mathrm{det}B = -5$, $\mathrm{det}C=5$, $\mathrm{det}\begin{pmatrix} 5&0\\5&-5 \end{pmatrix} = -25$, $\mathrm{det}\begin{pmatrix} 4&0\\3&0 \end{pmatrix}=0$ ; ces formules sont généralisables, i.e.
	$$ \forall (A,B)\in\mathcal{M}_n(\mathbb{K})^2,\quad \mathrm{det}(A+B) = \mathrm{det}A+\mathrm{det}B $$
	$$ \forall (A,B)\in\mathcal{M}_n(\mathbb{K})^2,\quad \mathrm{det}(AB) = \mathrm{det}A\times\mathrm{det}B $$
	\section*{Exercice 4}
	\subsection*{Calculer le déterminent de toutes les matrices}
	Pour calculer le déterminent d'une matrice, on va sommer un ensemble de produit. Chacun de ces produits sera un déterminent plus petit multiplié par les cœfficients de la première ligne.

	Pour choisir quel déterminent on doit calculer, on prend tous les cœfficients de la matrice dans l'ordre dans lequel ils apparaissent sans prendre ceux dans la ligne et la colonne du cœfficient de facteur. Dans la suite, on notera cette fonction $D$.

	Soient $A$ une matrice carrée de taille $n$ et $F=(a_{1,1}~\text{\----}~a_{1,n})$ les cœfficients de sa première ligne, la formule générale est :
	$$ \sum_{i=1}^{n} S(i)a_{1,i}\mathrm{det}(D(i)) $$
	où $S$ est une fonction donnant le signe à appliquer en fonction de cette matrice de signe :
	$$ \begin{pmatrix} +&-&+&-\\-&+&-&+\\+&-&+&-\\-&+&-&+ \end{pmatrix} $$
	(pour une matrice $4\times 4$)

	On a donc que le déterminent de $\begin{pmatrix} a_{1,1}&a_{1,2}&a_{1,3}\\a_{2,1}&a_{2,2}&a_{2,3}\\a_{3,1}&a_{2,3}&a_{3,3} \end{pmatrix} $ est $$ a_{1,1}\mathrm{det}\begin{pmatrix} a_{2,2}&a_{2,3}\\a_{3,2}&a_{3,3} \end{pmatrix} - a_{1,2}\mathrm{det}\begin{pmatrix} a_{2,1}&a_{2,3}\\a_{3,1}&a_{3,3} \end{pmatrix} + a_{1,3}\mathrm{det}\begin{pmatrix} a_{2,1}&a_{2,2}\\a_{3,1}&a_{3,2} \end{pmatrix}  $$

	\fbox{Choix des cœfficients} \---- En réalité, on n'est pas obligé de pendre la première ligne. On peut prendre n'importe qu'elle ligne ou colonne si elle nous arrange (notamment si elle a beaucoup de 0). De plus, les combinaisons linéaires ne changent pas le déterminent, on peut donc en abuser pour simplifier nos calculs.
	\subsection*{Exercices}
	\begin{enumerate}
		\item $\mathrm{det}\begin{pmatrix} 2&3\\2&3 \end{pmatrix} - \mathrm{det}\begin{pmatrix} 1&3\\0&3 \end{pmatrix} + \mathrm{det}\begin{pmatrix} 1&2\\0&2 \end{pmatrix} = -1$
		\item On s'amuse avec les combinaisons linéaires et on obtient que son $\mathrm{det}$ vaut $1$.
		\item Si $x=4$, alors la ligne 1 et 2 sont identiques. Si $x=5$, alors la ligne 2 et 3 sont identiques. Pour tout autre valeur de $n$, il n'existe pas de combinaisons linéaires entre les lignes 1 et 2 donnant la ligne 3. Donc $A_xX=B$ possède une solution pour tout $x\in\mathbb{R}\backslash\{4, 5\}$.
	\end{enumerate}
	\section*{Exercice 5}
	On a cas que :
	$$ L_2 = -2L_3+4L_1 $$
	donc $A_1$ n'est pas inversible.

	$$ \begin{pmatrix} 1&2&3&|&1&0&0\\-1&1&0&|&0&1&0\\-1&4&4&|&0&0&1 \end{pmatrix}  $$
	$$ \begin{pmatrix} 0&3&3&|&1&1&0\\-1&1&0&|&0&1&0\\0&3&4&|&0&-1&1 \end{pmatrix}  $$
	$$ \begin{pmatrix} 0&3&3&|&1&1&0\\-1&1&0&|&0&1&0\\0&0&1&|&-1&0&1 \end{pmatrix}  $$
	$$ \begin{pmatrix} -1&1&0&|&0&1&0\\0&3&3&|&1&1&0\\0&0&1&|&-1&0&1 \end{pmatrix}  $$
	$$ \begin{pmatrix} -1&1&0&|&0&1&0\\0&3&0&|&-2&1&3\\0&0&1&|&-1&0&1 \end{pmatrix}  $$
	$$ \begin{pmatrix} 1&0&0&|&-2/3&-2/3&1\\0&3&0&|&-2&1&3\\0&0&1&|&-1&0&1 \end{pmatrix}  $$
	$$ \begin{pmatrix} 1&0&0&|&-2/3&-2/3&1\\0&1&0&|&-2/3&1&1\\0&0&1&|&-1&0&1 \end{pmatrix}  $$
	ERREUR DE CALCUL QLQ PART
	\section*{Exercice 6}
	Le det de $A$ est $3\mathrm{det}\begin{pmatrix} 1&1&1\\1&0&1\\1&1&0 \end{pmatrix}$, i.e. $6\mathrm{det}\begin{pmatrix} 1&1\\0&1 \end{pmatrix} = 6 $
\end{document}
