%%=====================================================================================
%%
%%       Filename:  cours.tex
%%
%%    Description:  
%%
%%        Version:  1.0
%%        Created:  03/06/2024
%%       Revision:  none
%%
%%         Author:  YOUR NAME (), 
%%   Organization:  
%%      Copyright:  Copyright (c) 2024, YOUR NAME
%%
%%          Notes:  
%%
%%=====================================================================================
\documentclass[a4paper, titlepage]{article}

\usepackage[utf8]{inputenc}
\usepackage[T1]{fontenc}
\usepackage{textcomp}
\usepackage[french]{babel}
\usepackage{amsmath, amssymb}
\usepackage{amsthm}
\usepackage[svgnames]{xcolor}
\usepackage{thmtools}
\usepackage{lipsum}
\usepackage{framed}
\usepackage{parskip}
\usepackage{titlesec}

\renewcommand{\familydefault}{\sfdefault}

% figure support
\usepackage{import}
\usepackage{xifthen}
\pdfminorversion=7
\usepackage{pdfpages}
\usepackage{transparent}
\newcommand{\incfig}[1]{%
	\def\svgwidth{\columnwidth}
	\import{./figures/}{#1.pdf_tex}
}

\pdfsuppresswarningpagegroup=1

\colorlet{defn-color}{DarkBlue}
\colorlet{props-color}{Blue}
\colorlet{warn-color}{Red}
\colorlet{exemple-color}{Green}
\colorlet{corol-color}{Orange}
\newenvironment{defn-leftbar}{%
  \def\FrameCommand{{\color{defn-color}\vrule width 3pt} \hspace{10pt}}%
  \MakeFramed {\advance\hsize-\width \FrameRestore}}%
 {\endMakeFramed}
\newenvironment{warn-leftbar}{%
  \def\FrameCommand{{\color{warn-color}\vrule width 3pt} \hspace{10pt}}%
  \MakeFramed {\advance\hsize-\width \FrameRestore}}%
 {\endMakeFramed}
\newenvironment{exemple-leftbar}{%
  \def\FrameCommand{{\color{exemple-color}\vrule width 3pt} \hspace{10pt}}%
  \MakeFramed {\advance\hsize-\width \FrameRestore}}%
 {\endMakeFramed}
\newenvironment{props-leftbar}{%
  \def\FrameCommand{{\color{props-color}\vrule width 3pt} \hspace{10pt}}%
  \MakeFramed {\advance\hsize-\width \FrameRestore}}%
 {\endMakeFramed}
\newenvironment{corol-leftbar}{%
  \def\FrameCommand{{\color{corol-color}\vrule width 3pt} \hspace{10pt}}%
  \MakeFramed {\advance\hsize-\width \FrameRestore}}%
 {\endMakeFramed}

\def \freespace {1em}
\declaretheoremstyle[headfont=\sffamily\bfseries,%
 notefont=\sffamily\bfseries,%
 notebraces={}{},%
 headpunct=,%
 bodyfont=\sffamily,%
 headformat=\color{defn-color}Définition~\NUMBER\hfill\NOTE\smallskip\linebreak,%
 preheadhook=\vspace{\freespace}\begin{defn-leftbar},%
 postfoothook=\end{defn-leftbar},%
]{better-defn}
\declaretheoremstyle[headfont=\sffamily\bfseries,%
 notefont=\sffamily\bfseries,%
 notebraces={}{},%
 headpunct=,%
 bodyfont=\sffamily,%
 headformat=\color{warn-color}Attention\hfill\NOTE\smallskip\linebreak,%
 preheadhook=\vspace{\freespace}\begin{warn-leftbar},%
 postfoothook=\end{warn-leftbar},%
]{better-warn}
\declaretheoremstyle[headfont=\sffamily\bfseries,%
 notefont=\sffamily\bfseries,%
notebraces={}{},%
headpunct=,%
 bodyfont=\sffamily,%
 headformat=\color{exemple-color}Exemple~\NUMBER\hfill\NOTE\smallskip\linebreak,%
 preheadhook=\vspace{\freespace}\begin{exemple-leftbar},%
 postfoothook=\end{exemple-leftbar},%
]{better-exemple}
\declaretheoremstyle[headfont=\sffamily\bfseries,%
 notefont=\sffamily\bfseries,%
 notebraces={}{},%
 headpunct=,%
 bodyfont=\sffamily,%
 headformat=\color{props-color}Proposition~\NUMBER\hfill\NOTE\smallskip\linebreak,%
 preheadhook=\vspace{\freespace}\begin{props-leftbar},%
 postfoothook=\end{props-leftbar},%
]{better-props}
\declaretheoremstyle[headfont=\sffamily\bfseries,%
 notefont=\sffamily\bfseries,%
 notebraces={}{},%
 headpunct=,%
 bodyfont=\sffamily,%
 headformat=\color{props-color}Théorème~\NUMBER\hfill\NOTE\smallskip\linebreak,%
 preheadhook=\vspace{\freespace}\begin{props-leftbar},%
 postfoothook=\end{props-leftbar},%
]{better-thm}
\declaretheoremstyle[headfont=\sffamily\bfseries,%
 notefont=\sffamily\bfseries,%
 notebraces={}{},%
 headpunct=,%
 bodyfont=\sffamily,%
 headformat=\color{corol-color}Corollaire~\NUMBER\hfill\NOTE\smallskip\linebreak,%
 preheadhook=\vspace{\freespace}\begin{corol-leftbar},%
 postfoothook=\end{corol-leftbar},%
]{better-corol}

\declaretheorem[style=better-defn]{defn}
\declaretheorem[style=better-warn]{warn}
\declaretheorem[style=better-exemple]{exemple}
\declaretheorem[style=better-corol]{corol}
\declaretheorem[style=better-props, numberwithin=defn]{props}
\declaretheorem[style=better-thm, sibling=props]{thm}
\newtheorem*{lemme}{Lemme}%[subsection]
%\newtheorem{props}{Propriétés}[defn]

\newenvironment{system}%
{\left\lbrace\begin{align}}%
{\end{align}\right.}

\newenvironment{AQT}{{\fontfamily{qbk}\selectfont AQT}}

\usepackage{LobsterTwo}
\titleformat{\section}{\newpage\LobsterTwo \huge\bfseries}{\thesection.}{1em}{}
\titleformat{\subsection}{\vspace{2em}\LobsterTwo \Large\bfseries}{\thesubsection.}{1em}{}
\titleformat{\subsubsection}{\vspace{1em}\LobsterTwo \large\bfseries}{\thesubsubsection.}{1em}{}

\newenvironment{lititle}%
{\vspace{7mm}\LobsterTwo \large}%
{\\}

\renewenvironment{proof}{$\square$ \footnotesize\textit{Démonstration.}}{\begin{flushright}$\blacksquare$\end{flushright}}

\title{TD du 23 janvier 2025}
\author{William Hergès\thanks{Sorbonne Université - Faculté des Sciences, Faculté des Lettres}}

\begin{document}
	\maketitle
	\newpage
	\section*{Exercice 1}
	Les produits possibles sont $AX$, $BX$, $BA$, $AB$ et $DZ$, i.e.
	$$ AX = \begin{pmatrix} 1&2\\ 2&1 \end{pmatrix}\begin{pmatrix} 1\\-1 \end{pmatrix} = \begin{pmatrix} -1\\1 \end{pmatrix} $$
	$$ BX = \begin{pmatrix} -1&1\\0&1 \end{pmatrix}\begin{pmatrix} 1\\-1 \end{pmatrix} = \begin{pmatrix} -2\\-1 \end{pmatrix} $$
	$$ BA = \begin{pmatrix} -1&1\\0&1 \end{pmatrix}\begin{pmatrix} 1&2\\2&1 \end{pmatrix}= \begin{pmatrix} 1&-1\\2&1 \end{pmatrix} $$
	$$ AB =  \begin{pmatrix} 1&2\\2&1 \end{pmatrix}\begin{pmatrix} -1&1\\0&1 \end{pmatrix} = \begin{pmatrix} -1&3\\-2&3  \end{pmatrix} $$ 
	$$ DZ = \begin{pmatrix} 1&2&3\\3&2&1\\2&1&3 \end{pmatrix} \begin{pmatrix} 0\\2\\3 \end{pmatrix} = \begin{pmatrix} 13\\7\\11  \end{pmatrix}  $$
	\section*{Exercice 2}
	\begin{enumerate}
		\item On a :
			$$ AB = \begin{pmatrix} 1&1\\0&1 \end{pmatrix} \begin{pmatrix} 0&1\\-1&0 \end{pmatrix} = \begin{pmatrix} -1&1\\-1&0 \end{pmatrix}  $$
			$$ BA =  \begin{pmatrix} 0&1\\-1&0 \end{pmatrix} \begin{pmatrix} 1&1\\0&1 \end{pmatrix}= \begin{pmatrix} 0&1\\-1&-1 \end{pmatrix}  $$
			Donc $AB \neq BA$
		\item On a :
			$$ (A + B)^2 = \begin{pmatrix} 1&2\\-1&1 \end{pmatrix} \begin{pmatrix} 1&2\\-1&1 \end{pmatrix} = \begin{pmatrix} -1&3\\-2&-1 \end{pmatrix}  $$
		\item On a :
			$$ A^2 = \begin{pmatrix} 1&1\\0&1 \end{pmatrix} \begin{pmatrix} 1&1\\0&1 \end{pmatrix} = \begin{pmatrix} 1&2\\0&1 \end{pmatrix} $$
			$$ B^2 = \begin{pmatrix} 0&1\\-1&0 \end{pmatrix} \begin{pmatrix} 0&1\\-1&0 \end{pmatrix} = \begin{pmatrix} -1&0\\0&-1 \end{pmatrix}  $$
			Après calcul, on obtient que $(A+B)^2 = A^2+AB+BA+B^2 \neq A^2+2AB+B^2$.
		\item Cherchons l'inverse de $A$ :
			$$ ad-bc = 1-0 = 1 $$
			Donc l'inverse existe, i.e.
			$$ A^{-1} = \begin{pmatrix} 1&-1\\0&1 \end{pmatrix}  $$
			Cherchons maintenant l'inverse de $B$ :
			$$ ad-bc = 0+1 = 1 $$
			Donc l'inverse existe, i.e.
			$$ B^{-1} = \begin{pmatrix} 0&-1\\1&0 \end{pmatrix}  $$
	\end{enumerate}
	\section*{Exercice 3}
	$$ A^t = \begin{pmatrix} -5&-1&3\\-4&0&4\\-3&1&5\\-2&2&6 \end{pmatrix}  $$
	puis
	$$ BA^t = \begin{pmatrix} -1&0&1&2\\ 2&1&0&-1\\ 1&0&-1&2 \end{pmatrix}A^t = \begin{pmatrix} -2&6&14\\-12&-4&4\\-8&2&10 \end{pmatrix}   $$
	\section*{Exercice 4}
	\begin{enumerate}
		\item Ce vecteur $v$ peut être représentée par un complexe $z=a+ib$. Or il existe $r,\theta\in\mathbb{R}$ tel que $z=re^{i\theta}$, d'où $r\cos\theta+ri\sin\theta$.
		\item On a 
			$$ R_{\theta}\cdot v = r(\cos\phi\cos\theta-\sin\phi\sin\theta)+ir(\sin\phi\cos\theta+\cos\theta\sin\theta) $$
			i.e.
			$$ R_{\theta}\cdot v = r\cos(\phi+\theta)+ir\sin(\phi+\theta) $$
	\end{enumerate}
	\section*{Exercice 5}
	On a :
	$$ AT = \begin{pmatrix} 2&4&8\\ 1&2&5 \end{pmatrix} $$
	$$ BT = \begin{pmatrix} 1&2&3\\-1&-2&-1 \end{pmatrix}  $$
	$$ CT = \begin{pmatrix} 0&0&2\\1&2&1 \end{pmatrix}  $$
	$$ DT = \begin{pmatrix} \frac{\sqrt{3}-1}{2}&\sqrt{3}-1&\frac{3\sqrt{3}-1}{2}\\\frac{1+\sqrt{3}}{2}&1+\sqrt{3}&\frac{3+\sqrt{3}}{2} \end{pmatrix}  $$
	\section*{Exercice 6}
	\begin{enumerate}
		\item Soit $P_n: M^n\begin{pmatrix} 0\\1 \end{pmatrix} = \begin{pmatrix} f_n\\f_{n+1} \end{pmatrix}$ avec $n\in\mathbb{N}^*$. Montrons que $P_n$ est vraie pour tout $n$.

			\fbox{Initialisation} On a $M^1\begin{pmatrix} 0\\1 \end{pmatrix} = \begin{pmatrix} 1\\1 \end{pmatrix} = \begin{pmatrix} f_{1}\\f_{2} \end{pmatrix}$. Donc $P_1$ est vraie.

			\fbox{Hérédité} Posons $n\in\mathbb{N}^*$ tel que $P_n$ soit vraie. On a :
			$$ M^n\begin{pmatrix} 0\\1 \end{pmatrix} = \begin{pmatrix} f_{n}\\f_{n+1} \end{pmatrix}  $$
			Donc,
			$$ M^{n+1}\begin{pmatrix} 0\\1 \end{pmatrix} = M\begin{pmatrix} f_n\\f_{n+1} \end{pmatrix} $$
			Or,
			$$ M\begin{pmatrix} f_n\\f_{n+1} \end{pmatrix} = \begin{pmatrix} f_{n+1}\\f_n+f_{n+1} \end{pmatrix} $$
			i.e.,
			$$ M^{n+1}\begin{pmatrix} 0\\1 \end{pmatrix} = \begin{pmatrix} f_{n+1}\\f_n+f_{n+1} \end{pmatrix} $$
			Alors,
			$$ M^{n+1}\begin{pmatrix} 0\\1 \end{pmatrix} = \begin{pmatrix} f_{n+1}\\f_{n+2} \end{pmatrix} $$
			Ainsi, $P_{n+1}$ est vraie 

			\fbox{Conclusion} $P_n$ est vraie pour tout $n\in\mathbb{N}^*$.

		\item Montrons que pour tout $n\in\mathbb{N}^*$, on a que
			$$ P_n:\quad M^n = \begin{pmatrix} f_{n-1}&f_n\\f_n&f_{n+1} \end{pmatrix}  $$
			est vraie.

			\fbox{Initialisation} On a $M^1 = \begin{pmatrix} 0&1\\1&1 \end{pmatrix} = \begin{pmatrix} f_0&f_1\\f_1&f_2 \end{pmatrix}  $. Donc $P_1$ est vraie.

			\fbox{Hérédité} Fixons $n$ dans $\mathbb{N}^*$ tel que $P_n$ soit vraie. On a :
			$$ M^n = \begin{pmatrix} f_{n-1}&f_n\\f_n&f_{n+1} \end{pmatrix} $$
			Donc,
			$$ M^nM = \begin{pmatrix} f_{n-1}&f_n\\f_n&f_{n+1} \end{pmatrix} M $$
			i.e.,
			$$ M^{n+1} = \begin{pmatrix} f_{n}&f_{n+1}\\f_{n+1}&f_n+f_{n+1} \end{pmatrix} $$
			Or,
			$$ M^{n+1} = \begin{pmatrix} f_{n}&f_{n+1}\\f_{n+1}&f_{n+2} \end{pmatrix} $$
			Ainsi, $P_{n+1}$ est vraie.

			\fbox{Conclusion} $P_n$ est vraie pour tout $n\in\mathbb{N}^*$.
	\end{enumerate}
\end{document}
