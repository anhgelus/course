%%=====================================================================================
%%
%%       Filename:  cours.tex
%%
%%    Description:  
%%
%%        Version:  1.0
%%        Created:  03/06/2024
%%       Revision:  none
%%
%%         Author:  YOUR NAME (), 
%%   Organization:  
%%      Copyright:  Copyright (c) 2024, YOUR NAME
%%
%%          Notes:  
%%
%%=====================================================================================
\documentclass[a4paper, titlepage]{article}

\usepackage[utf8]{inputenc}
\usepackage[T1]{fontenc}
\usepackage{textcomp}
\usepackage[french]{babel}
\usepackage{amsmath, amssymb}
\usepackage{amsthm}
\usepackage[svgnames]{xcolor}
\usepackage{thmtools}
\usepackage{lipsum}
\usepackage{framed}
\usepackage{parskip}
\usepackage{titlesec}
% \usepackage{inter}

\renewcommand{\familydefault}{\sfdefault}
% \renewcommand{\familydefault}{\sffamily}

% figure support
\usepackage{import}
\usepackage{xifthen}
\pdfminorversion=7
\usepackage{pdfpages}
\usepackage{transparent}
\newcommand{\incfig}[1]{%
	\def\svgwidth{\columnwidth}
	\import{./figures/}{#1.pdf_tex}
}

\pdfsuppresswarningpagegroup=1

\colorlet{defn-color}{DarkBlue}
\colorlet{props-color}{Blue}
\colorlet{warn-color}{Red}
\colorlet{exemple-color}{Green}
\colorlet{corol-color}{Orange}
\newenvironment{defn-leftbar}{%
  \def\FrameCommand{{\color{defn-color}\vrule width 3pt} \hspace{10pt}}%
  \MakeFramed {\advance\hsize-\width \FrameRestore}}%
 {\endMakeFramed}
\newenvironment{warn-leftbar}{%
  \def\FrameCommand{{\color{warn-color}\vrule width 3pt} \hspace{10pt}}%
  \MakeFramed {\advance\hsize-\width \FrameRestore}}%
 {\endMakeFramed}
\newenvironment{exemple-leftbar}{%
  \def\FrameCommand{{\color{exemple-color}\vrule width 3pt} \hspace{10pt}}%
  \MakeFramed {\advance\hsize-\width \FrameRestore}}%
 {\endMakeFramed}
\newenvironment{props-leftbar}{%
  \def\FrameCommand{{\color{props-color}\vrule width 3pt} \hspace{10pt}}%
  \MakeFramed {\advance\hsize-\width \FrameRestore}}%
 {\endMakeFramed}
\newenvironment{corol-leftbar}{%
  \def\FrameCommand{{\color{corol-color}\vrule width 3pt} \hspace{10pt}}%
  \MakeFramed {\advance\hsize-\width \FrameRestore}}%
 {\endMakeFramed}

\def \freespace {1em}
\declaretheoremstyle[headfont=\sffamily\bfseries,%
 notefont=\sffamily\bfseries,%
 notebraces={}{},%
 headpunct=,%
 bodyfont=\sffamily,%
 headformat=\color{defn-color}Définition~\NUMBER\hfill\NOTE\smallskip\linebreak,%
 preheadhook=\vspace{\freespace}\begin{defn-leftbar},%
 postfoothook=\end{defn-leftbar},%
]{better-defn}
\declaretheoremstyle[headfont=\sffamily\bfseries,%
 notefont=\sffamily\bfseries,%
 notebraces={}{},%
 headpunct=,%
 bodyfont=\sffamily,%
 headformat=\color{warn-color}Attention\hfill\NOTE\smallskip\linebreak,%
 preheadhook=\vspace{\freespace}\begin{warn-leftbar},%
 postfoothook=\end{warn-leftbar},%
]{better-warn}
\declaretheoremstyle[headfont=\sffamily\bfseries,%
 notefont=\sffamily\bfseries,%
notebraces={}{},%
headpunct=,%
 bodyfont=\sffamily,%
 headformat=\color{exemple-color}Exemple~\NUMBER\hfill\NOTE\smallskip\linebreak,%
 preheadhook=\vspace{\freespace}\begin{exemple-leftbar},%
 postfoothook=\end{exemple-leftbar},%
]{better-exemple}
\declaretheoremstyle[headfont=\sffamily\bfseries,%
 notefont=\sffamily\bfseries,%
 notebraces={}{},%
 headpunct=,%
 bodyfont=\sffamily,%
 headformat=\color{props-color}Proposition~\NUMBER\hfill\NOTE\smallskip\linebreak,%
 preheadhook=\vspace{\freespace}\begin{props-leftbar},%
 postfoothook=\end{props-leftbar},%
]{better-props}
\declaretheoremstyle[headfont=\sffamily\bfseries,%
 notefont=\sffamily\bfseries,%
 notebraces={}{},%
 headpunct=,%
 bodyfont=\sffamily,%
 headformat=\color{props-color}Théorème~\NUMBER\hfill\NOTE\smallskip\linebreak,%
 preheadhook=\vspace{\freespace}\begin{props-leftbar},%
 postfoothook=\end{props-leftbar},%
]{better-thm}
\declaretheoremstyle[headfont=\sffamily\bfseries,%
 notefont=\sffamily\bfseries,%
 notebraces={}{},%
 headpunct=,%
 bodyfont=\sffamily,%
 headformat=\color{corol-color}Corollaire~\NUMBER\hfill\NOTE\smallskip\linebreak,%
 preheadhook=\vspace{\freespace}\begin{corol-leftbar},%
 postfoothook=\end{corol-leftbar},%
]{better-corol}

\declaretheorem[style=better-defn]{defn}
\declaretheorem[style=better-warn]{warn}
\declaretheorem[style=better-exemple]{exemple}
\declaretheorem[style=better-corol]{corol}
\declaretheorem[style=better-props, numberwithin=defn]{props}
\declaretheorem[style=better-thm, sibling=props]{thm}
\newtheorem*{lemme}{Lemme}%[subsection]
%\newtheorem{props}{Propriétés}[defn]

\newenvironment{system}%
{\left\lbrace\begin{align}}%
{\end{align}\right.}

\newenvironment{AQT}{{\fontfamily{qbk}\selectfont AQT}}

\usepackage{LobsterTwo}
\titleformat{\section}{\newpage\LobsterTwo \huge\bfseries}{\thesection.}{1em}{}
\titleformat{\subsection}{\vspace{2em}\LobsterTwo \Large\bfseries}{\thesubsection.}{1em}{}
\titleformat{\subsubsection}{\vspace{1em}\LobsterTwo \large\bfseries}{\thesubsubsection.}{1em}{}

\newenvironment{lititle}%
{\vspace{7mm}\LobsterTwo \large}%
{\\}

\renewenvironment{proof}{$\square$ \footnotesize\textit{Démonstration.}}{\begin{flushright}$\blacksquare$\end{flushright}}

\title{TD du 6 mars}
\author{William Hergès\thanks{Sorbonne Université - Faculté des Sciences, Faculté des Lettres}}

\begin{document}
	\maketitle
	\section{Feuille du 20 février}
	\subsection*{Exercice 4}
	Soient $v_1,v_2$ dans $P$ et $\lambda$ dans $\mathbb{K}$. On a :
	$$ v_1+\lambda v_2 = x+2y+4z+\lambda x'+\lambda2y'+\lambda4z' = 0 $$
	car $0+0$ vaut $0$. De plus, $0_P\in P$, donc $P$ est un espace vectoriel.

	On peut écrire $x+2y+4z=0$ comme $x+2y=-4z$. Ainsi, $z$ est lié, donc $P=\mathrm{Vect}\left( \small\begin{pmatrix} 1/4\\0\\0 \end{pmatrix} ,\small\begin{pmatrix} 0\\1/2\\0 \end{pmatrix} \right)$, ce qui est aussi la base de $P$.

	Soient $v_1,v_2$ dans $D$ et $\lambda$ dans $\mathbb{K}$. On a :
	$$ v_1+\lambda v_2 = \begin{pmatrix} x\\-y\\z \end{pmatrix} + \lambda\begin{pmatrix} x'\\-y'\\z' \end{pmatrix} \land\ldots $$
	ce qui vaut bien $0$ car $0+0=0$. De plus, $O_D\in D$, donc $D$ est un ev.

	On a :
	$$ \left\{\begin{matrix} x&-y&+z&=&0\\ x&+y&-z&=&0 \end{matrix} \right.\iff \left\{\begin{matrix} x&-y&=&-z\\ x&+y&=&z \end{matrix} \right. $$
	En additionnant $L_1$ et $L_2$, on obtient que $x=0$ et donc que $y=z$. Ainsi, $(0,1,1)$ est une base de $D$ et $\mathrm{dim}(D) = 1$.

	\textit{Mutadis mutandis}, $H$ est un ev.

	On a :
	$$ x+y+z+t=0\iff x+y+z=-t $$
	Donc $t$ est fixé. Une base de $H$ est $\{(1,0,0),(0,1,0),(0,0,1)\}$ et donc sa dimension vaut 3.
	\subsection*{Exercice 6}
	Trouver une équation de l'image signifie, trouver une équation dépendant uniquement des variables de l'image. Par exemple, quand on a un ensemble d'équation avec $x_n'$ dépendant de $(x_n)$ décrivant l'image, une équation de l'image dépendra uniquement de $(x_n')$.
	\section{Feuille du 6 mars}
	On a : $(1-\lambda)(3-\lambda)-8 = 0 \iff \lambda^2-4\lambda-5 = 0$.\\
	$-1$ est une valeur propre (solution évidente).\\
	$\Delta = 36 = 6^2$, donc $5$ est aussi une solution.
	
	Le sous-espace propre lié à $\lambda = 5$ est l'ensemble des $u$ tels que $(A-5I)u = 0$, i.e.
	$$ \begin{pmatrix} -4&4\\-2&2 \end{pmatrix} u = 0 $$
	donc, $u\in\{(a,a),a\in\mathbb{R}\}$.
	
	Le sous-espace propre lié à $\lambda = -1$ est l'ensemble des $u$ tels que $(A+1I)u = 0$, i.e.
	$$ \begin{pmatrix} 2&4\\2&4 \end{pmatrix} u = 0 $$
	donc, $u\in\{(-2a,a),a\in\mathbb{R}\}$.

	Ainsi,
	$$ \underset{D}{\underbrace{\begin{pmatrix} 5&0\\0&-1 \end{pmatrix}}} = P^{-1}A\underset{P}{\underbrace{\begin{pmatrix} 1&-2\\1&1 \end{pmatrix}}} $$
	On a aussi que $P^{-1} = \frac{1}{3}\tiny\begin{pmatrix} 1&2\\-1&1 \end{pmatrix} $
\end{document}
