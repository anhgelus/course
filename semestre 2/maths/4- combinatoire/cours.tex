%%=====================================================================================
%%
%%       Filename:  cours.tex
%%
%%    Description:  
%%
%%        Version:  1.0
%%        Created:  03/06/2024
%%       Revision:  none
%%
%%         Author:  YOUR NAME (), 
%%   Organization:  
%%      Copyright:  Copyright (c) 2024, YOUR NAME
%%
%%          Notes:  
%%
%%=====================================================================================
\documentclass[a4paper, titlepage]{article}

\usepackage[utf8]{inputenc}
\usepackage[T1]{fontenc}
\usepackage{textcomp}
\usepackage[french]{babel}
\usepackage{amsmath, amssymb}
\usepackage{amsthm}
\usepackage[svgnames]{xcolor}
\usepackage{thmtools}
\usepackage{lipsum}
\usepackage{framed}
\usepackage{parskip}
\usepackage{titlesec}
\usepackage{newtxtext}

% \renewcommand{\familydefault}{\sfdefault}

% figure support
\usepackage{import}
\usepackage{xifthen}
\pdfminorversion=7
\usepackage{pdfpages}
\usepackage{transparent}
\newcommand{\incfig}[1]{%
	\def\svgwidth{\columnwidth}
	\import{./figures/}{#1.pdf_tex}
}

\pdfsuppresswarningpagegroup=1

\colorlet{defn-color}{DarkBlue}
\colorlet{props-color}{Blue}
\colorlet{warn-color}{Red}
\colorlet{exemple-color}{Green}
\colorlet{corol-color}{Orange}
\newenvironment{defn-leftbar}{%
  \def\FrameCommand{{\color{defn-color}\vrule width 3pt} \hspace{10pt}}%
  \MakeFramed {\advance\hsize-\width \FrameRestore}}%
 {\endMakeFramed}
\newenvironment{warn-leftbar}{%
  \def\FrameCommand{{\color{warn-color}\vrule width 3pt} \hspace{10pt}}%
  \MakeFramed {\advance\hsize-\width \FrameRestore}}%
 {\endMakeFramed}
\newenvironment{exemple-leftbar}{%
  \def\FrameCommand{{\color{exemple-color}\vrule width 3pt} \hspace{10pt}}%
  \MakeFramed {\advance\hsize-\width \FrameRestore}}%
 {\endMakeFramed}
\newenvironment{props-leftbar}{%
  \def\FrameCommand{{\color{props-color}\vrule width 3pt} \hspace{10pt}}%
  \MakeFramed {\advance\hsize-\width \FrameRestore}}%
 {\endMakeFramed}
\newenvironment{corol-leftbar}{%
  \def\FrameCommand{{\color{corol-color}\vrule width 3pt} \hspace{10pt}}%
  \MakeFramed {\advance\hsize-\width \FrameRestore}}%
 {\endMakeFramed}

\def \freespace {1em}
\declaretheoremstyle[headfont=\sffamily\bfseries,%
 notefont=\sffamily\bfseries,%
 notebraces={}{},%
 headpunct=,%
% bodyfont=\sffamily,%
 headformat=\color{defn-color}Définition~\NUMBER\hfill\NOTE\smallskip\linebreak,%
 preheadhook=\vspace{\freespace}\begin{defn-leftbar},%
 postfoothook=\end{defn-leftbar},%
]{better-defn}
\declaretheoremstyle[headfont=\sffamily\bfseries,%
 notefont=\sffamily\bfseries,%
 notebraces={}{},%
 headpunct=,%
% bodyfont=\sffamily,%
 headformat=\color{warn-color}Attention\hfill\NOTE\smallskip\linebreak,%
 preheadhook=\vspace{\freespace}\begin{warn-leftbar},%
 postfoothook=\end{warn-leftbar},%
]{better-warn}
\declaretheoremstyle[headfont=\sffamily\bfseries,%
 notefont=\sffamily\bfseries,%
notebraces={}{},%
headpunct=,%
% bodyfont=\sffamily,%
 headformat=\color{exemple-color}Exemple~\NUMBER\hfill\NOTE\smallskip\linebreak,%
 preheadhook=\vspace{\freespace}\begin{exemple-leftbar},%
 postfoothook=\end{exemple-leftbar},%
]{better-exemple}
\declaretheoremstyle[headfont=\sffamily\bfseries,%
 notefont=\sffamily\bfseries,%
 notebraces={}{},%
 headpunct=,%
% bodyfont=\sffamily,%
 headformat=\color{props-color}Proposition~\NUMBER\hfill\NOTE\smallskip\linebreak,%
 preheadhook=\vspace{\freespace}\begin{props-leftbar},%
 postfoothook=\end{props-leftbar},%
]{better-props}
\declaretheoremstyle[headfont=\sffamily\bfseries,%
 notefont=\sffamily\bfseries,%
 notebraces={}{},%
 headpunct=,%
% bodyfont=\sffamily,%
 headformat=\color{props-color}Théorème~\NUMBER\hfill\NOTE\smallskip\linebreak,%
 preheadhook=\vspace{\freespace}\begin{props-leftbar},%
 postfoothook=\end{props-leftbar},%
]{better-thm}
\declaretheoremstyle[headfont=\sffamily\bfseries,%
 notefont=\sffamily\bfseries,%
 notebraces={}{},%
 headpunct=,%
% bodyfont=\sffamily,%
 headformat=\color{corol-color}Corollaire~\NUMBER\hfill\NOTE\smallskip\linebreak,%
 preheadhook=\vspace{\freespace}\begin{corol-leftbar},%
 postfoothook=\end{corol-leftbar},%
]{better-corol}

\declaretheorem[style=better-defn]{defn}
\declaretheorem[style=better-warn]{warn}
\declaretheorem[style=better-exemple]{exemple}
\declaretheorem[style=better-corol]{corol}
\declaretheorem[style=better-props, numberwithin=defn]{props}
\declaretheorem[style=better-thm, sibling=props]{thm}
\newtheorem*{lemme}{Lemme}%[subsection]
%\newtheorem{props}{Propriétés}[defn]

\newenvironment{system}%
{\left\lbrace\begin{align}}%
{\end{align}\right.}

\newenvironment{AQT}{{\fontfamily{qbk}\selectfont AQT}}

\usepackage{LobsterTwo}
\titleformat{\section}{\newpage\LobsterTwo \huge\bfseries}{\thesection.}{1em}{}
\titleformat{\subsection}{\vspace{2em}\LobsterTwo \Large\bfseries}{\thesubsection.}{1em}{}
\titleformat{\subsubsection}{\vspace{1em}\LobsterTwo \large\bfseries}{\thesubsubsection.}{1em}{}

\newenvironment{lititle}%
{\vspace{7mm}\LobsterTwo \large}%
{\\}

\renewenvironment{proof}{$\square$ \footnotesize\textit{Démonstration.}}{\begin{flushright}$\blacksquare$\end{flushright}}

\title{Combinatoire}
\author{William Hergès\thanks{Sorbonne Université - Faculté des Sciences, Faculté des Lettres}}

\begin{document}
	\maketitle
	\tableofcontents
	\newpage
	\section{Notation}
	Dans cette section, on parle de théorie des ensembles.

	On notera toujours $E$ l'ensemble ambient. Soit $A$ un sous-ensemble de $E$. On le note~: $A\subseteq E$ (ou $A\subset E$, mais on l'aime moins celle-là). On note l'inclusion stricte $A\subsetneq E$.
	\begin{defn}
		$A\cup B$ est défini comme~:
		$$ x\in A\cup B\quad\implies\quad x\in A\quad\lor\quad x\in B $$
		On a :
		$$ A\cup B = \{x\in E\quad|\quad x\in A\lor x\in B\} $$
	\end{defn}
	\begin{defn}
		$A\cap B$ est défini comme~:
		$$ x\in A\cap B\quad\implies\quad x\in A\quad\land\quad x\in B $$
		On a :
		$$ A\cup B = \{x\in E\quad|\quad x\in A\land x\in B\} $$
	\end{defn}
	\begin{defn}
		$A\backslash B$ est défini comme~:
		$$ x\in A\backslash B\quad\implies\quad x\in A\quad\land\quad x\not\in B$$
		On a~:
		$$ A\backslash B = \{x\in E\quad|\quad x\in A\land x\not\in B\} $$
	\end{defn}
	\begin{defn}
		$E\backslash A$ est le complémentaire de $A$ et est défini comme~:
		$$ \forall x\in E\backslash A\quad\implies\quad x\in E\quad\land\quad x\not\in A $$
		On a~:
		$$ E\backslash A=\{x\in E\quad|\quad x\not\in A\} $$
	\end{defn}
	\begin{defn}
		Si $E$ est un ensemble fini, on a que $\mathrm{card}(E)$ ou $|E|$ est le nombre d'éléments de $E$.
	\end{defn}
	\begin{props}
		On a :
		$$ \mathrm{card}(A\cup B) = \mathrm{card}(A) + \mathrm{card}(B) - \mathrm{card}(A\cap B) $$
	\end{props}
	\begin{defn}
		Le produit cartésien est noté $\times$ et est :
		$$ A\times B = \{(x,y)\quad|\quad x\in A,y\in B\} $$
	\end{defn}
	\begin{props}
		On a~:
		$$ \mathrm{card}(A\times B) = \mathrm{card}(A)\times\mathrm{card}(B) $$
	\end{props}
	\section{Combinaisons}
	Soit $\Omega=\{1,2,\ldots,n\}$, un ensemble de $n$ éléments.

	\begin{defn}
		Une combinaison de $k$ éléments parmis les éléments de $\Omega$ est un sous-ensemble $A\subseteq\Omega$ tel que $\mathrm{card}(A) = k$.
	\end{defn}
	\begin{props}
		Le nombre de combinaisons de $k$ éléments parmis les éléments de $\Omega$ est~:
		$$ \frac{\mathrm{card}(\Omega)!}{k!(\mathrm{card}(\Omega)-k)!} = \frac{n!}{k!(n-k)!} $$
		pour $n = \mathrm{card}(\Omega)$, i.e.
		$$ \binom{\mathrm{card}(\Omega)}{k} = \binom nk $$
		ce qui est un cœfficient binomial.
	\end{props}
	On peut aussi écrire le cœfficient binomial $\mathcal{C}^k_n = \binom nk$. $\mathcal{C}$ signifie \textit{combinaison}~! Ici, l'\textit{ordre ne compte pas}.

	Il existe aussi $\mathcal{A}^k_n = \frac{n!}{(n-k)!}$, ce qui est le nombre de choix possible où l'\textit{ordre compte}. $\mathcal{A}$ pour \textit{arrangement}~!

	\begin{exemple}
		Une personne qui veut aller à un endroit doit forcément faire une combinaison parmis $\{D,D,D,B,B\}$. (Ceci représente tous les chemins les plus rapides pour y aller~: on doit forcément prendre 3 fois droite et deux fois gauche.) Donc, il y a $\binom 52 = 10$ possibilités~: il suffit de choisir quand on descend (donc 2 possibilités) pour avoir tous les cas possibles~! (On aurait aussi pu choisir quand on va à droite, mais les calculs sont plus méchants.)
	\end{exemple}

	\begin{props}
		On a~:
		$$ \mathcal{C}^k_n = \mathcal{C}^{k-1}_{n-1} + \mathcal{C}^k_{n-1} $$
		i.e.
		$$ \binom nk = \binom{k-1}{n-1}+\binom k{n-1} $$
	\end{props}

	Maintenant, voici un énoncé très pratique~: le \textit{binôme de Newton}. Le prof l'a démontré en cours, mais j'avais la flemme d'écrire la démo. Voir la démonstration de l'année dernière.

	\begin{thm}[Binôme de Newton]
		On a~:
		$$ (x+y)^n = \sum_{k=0}^{n} \binom nk x^ky^{n-k} $$
	\end{thm}

	\begin{corol}
		On a cette magnifique relation pas très utile~:
		$$ \sum^n_{k=0}\binom nk = 2^n $$	
	\end{corol}
	\begin{proof}
		D'après le binôme de Newton, on a~:
		$$ (1+1)^n = \sum_{k=0}^{n} \binom nk 1^k1^{n-k} $$
		i.e.
		$$ 2^n = \sum_{k=0}^{n} \binom nk $$
	\end{proof}
	\begin{exemple}
	Dans un groupe de 20 personnes, il y a $2^{20}$ sous-groupes possibles. Il y a $20$ tailles de sous-groupes possibles~: de 1 à 20. Le nombre de sous-groupe contenant personne est $\binom{20}0$, celui contenant une personne est $\binom{20}1$, \ldots, celui contenant 20 personnes est $\binom{20}{20}$. Ainsi, le nombre de sous-groupe est la somme de tout cela et le résultat est donné par la formule juste au dessus.
	\end{exemple}
\end{document}
