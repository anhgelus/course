%%=====================================================================================
%%
%%       Filename:  cours.tex
%%
%%    Description:  
%%
%%        Version:  1.0
%%        Created:  03/06/2024
%%       Revision:  none
%%
%%         Author:  YOUR NAME (), 
%%   Organization:  
%%      Copyright:  Copyright (c) 2024, YOUR NAME
%%
%%          Notes:  
%%
%%=====================================================================================
\documentclass[a4paper, titlepage]{article}

\usepackage[utf8]{inputenc}
\usepackage[T1]{fontenc}
\usepackage{textcomp}
\usepackage[french]{babel}
\usepackage{amsmath, amssymb}
\usepackage{amsthm}
\usepackage[svgnames]{xcolor}
\usepackage{thmtools}
\usepackage{lipsum}
\usepackage{framed}
\usepackage{parskip}
\usepackage{titlesec}

\renewcommand{\familydefault}{\sfdefault}

% figure support
\usepackage{import}
\usepackage{xifthen}
\pdfminorversion=7
\usepackage{pdfpages}
\usepackage{transparent}
\newcommand{\incfig}[1]{%
	\def\svgwidth{\columnwidth}
	\import{./figures/}{#1.pdf_tex}
}

\pdfsuppresswarningpagegroup=1

\colorlet{defn-color}{DarkBlue}
\colorlet{props-color}{Blue}
\colorlet{warn-color}{Red}
\colorlet{exemple-color}{Green}
\colorlet{corol-color}{Orange}
\newenvironment{defn-leftbar}{%
  \def\FrameCommand{{\color{defn-color}\vrule width 3pt} \hspace{10pt}}%
  \MakeFramed {\advance\hsize-\width \FrameRestore}}%
 {\endMakeFramed}
\newenvironment{warn-leftbar}{%
  \def\FrameCommand{{\color{warn-color}\vrule width 3pt} \hspace{10pt}}%
  \MakeFramed {\advance\hsize-\width \FrameRestore}}%
 {\endMakeFramed}
\newenvironment{exemple-leftbar}{%
  \def\FrameCommand{{\color{exemple-color}\vrule width 3pt} \hspace{10pt}}%
  \MakeFramed {\advance\hsize-\width \FrameRestore}}%
 {\endMakeFramed}
\newenvironment{props-leftbar}{%
  \def\FrameCommand{{\color{props-color}\vrule width 3pt} \hspace{10pt}}%
  \MakeFramed {\advance\hsize-\width \FrameRestore}}%
 {\endMakeFramed}
\newenvironment{corol-leftbar}{%
  \def\FrameCommand{{\color{corol-color}\vrule width 3pt} \hspace{10pt}}%
  \MakeFramed {\advance\hsize-\width \FrameRestore}}%
 {\endMakeFramed}

\def \freespace {1em}
\declaretheoremstyle[headfont=\sffamily\bfseries,%
 notefont=\sffamily\bfseries,%
 notebraces={}{},%
 headpunct=,%
 bodyfont=\sffamily,%
 headformat=\color{defn-color}Définition~\NUMBER\hfill\NOTE\smallskip\linebreak,%
 preheadhook=\vspace{\freespace}\begin{defn-leftbar},%
 postfoothook=\end{defn-leftbar},%
]{better-defn}
\declaretheoremstyle[headfont=\sffamily\bfseries,%
 notefont=\sffamily\bfseries,%
 notebraces={}{},%
 headpunct=,%
 bodyfont=\sffamily,%
 headformat=\color{warn-color}Attention\hfill\NOTE\smallskip\linebreak,%
 preheadhook=\vspace{\freespace}\begin{warn-leftbar},%
 postfoothook=\end{warn-leftbar},%
]{better-warn}
\declaretheoremstyle[headfont=\sffamily\bfseries,%
 notefont=\sffamily\bfseries,%
notebraces={}{},%
headpunct=,%
 bodyfont=\sffamily,%
 headformat=\color{exemple-color}Exemple~\NUMBER\hfill\NOTE\smallskip\linebreak,%
 preheadhook=\vspace{\freespace}\begin{exemple-leftbar},%
 postfoothook=\end{exemple-leftbar},%
]{better-exemple}
\declaretheoremstyle[headfont=\sffamily\bfseries,%
 notefont=\sffamily\bfseries,%
 notebraces={}{},%
 headpunct=,%
 bodyfont=\sffamily,%
 headformat=\color{props-color}Proposition~\NUMBER\hfill\NOTE\smallskip\linebreak,%
 preheadhook=\vspace{\freespace}\begin{props-leftbar},%
 postfoothook=\end{props-leftbar},%
]{better-props}
\declaretheoremstyle[headfont=\sffamily\bfseries,%
 notefont=\sffamily\bfseries,%
 notebraces={}{},%
 headpunct=,%
 bodyfont=\sffamily,%
 headformat=\color{props-color}Théorème~\NUMBER\hfill\NOTE\smallskip\linebreak,%
 preheadhook=\vspace{\freespace}\begin{props-leftbar},%
 postfoothook=\end{props-leftbar},%
]{better-thm}
\declaretheoremstyle[headfont=\sffamily\bfseries,%
 notefont=\sffamily\bfseries,%
 notebraces={}{},%
 headpunct=,%
 bodyfont=\sffamily,%
 headformat=\color{corol-color}Corollaire~\NUMBER\hfill\NOTE\smallskip\linebreak,%
 preheadhook=\vspace{\freespace}\begin{corol-leftbar},%
 postfoothook=\end{corol-leftbar},%
]{better-corol}

\declaretheorem[style=better-defn]{defn}
\declaretheorem[style=better-warn]{warn}
\declaretheorem[style=better-exemple]{exemple}
\declaretheorem[style=better-corol]{corol}
\declaretheorem[style=better-props, numberwithin=defn]{props}
\declaretheorem[style=better-thm, sibling=props]{thm}
\newtheorem*{lemme}{Lemme}%[subsection]
%\newtheorem{props}{Propriétés}[defn]

\newenvironment{system}%
{\left\lbrace\begin{align}}%
{\end{align}\right.}

\newenvironment{AQT}{{\fontfamily{qbk}\selectfont AQT}}

\usepackage{LobsterTwo}
\titleformat{\section}{\newpage\LobsterTwo \huge\bfseries}{\thesection.}{1em}{}
\titleformat{\subsection}{\vspace{2em}\LobsterTwo \Large\bfseries}{\thesubsection.}{1em}{}
\titleformat{\subsubsection}{\vspace{1em}\LobsterTwo \large\bfseries}{\thesubsubsection.}{1em}{}

\newenvironment{lititle}%
{\vspace{7mm}\LobsterTwo \large}%
{\\}

\renewenvironment{proof}{$\square$ \footnotesize\textit{Démonstration.}}{\begin{flushright}$\blacksquare$\end{flushright}}

\title{Familles et bases}
\author{William Hergès\thanks{Sorbonne Université - Faculté des Sciences, Faculté des Lettres}}

\begin{document}
	\maketitle
	\tableofcontents
	\newpage
	\section{Familles}
	\begin{defn}
		Soit $(v_1,\ldots,v_p)$ une famille de vecteurs dans $\mathbb{R}^p$.

		La famille est dite liée s'il existe $(\lambda_1,\ldots,\lambda_p)\in\mathbb{R}^p$ non tous nuls tel que :
		$$ \sum_{i=1}^{p} \lambda_iv_i = 0 $$
	\end{defn}
	\begin{defn}
		Si une famille n'est pas liée, alors elle est libre.
	\end{defn}
	Le vecteur nul est toujours dans une famille liée !

	\begin{thm}
		La famille $A=(v_1,\ldots,v_q)$ est libre si et seulement si $\mathrm{rg}(A)=q$.
	\end{thm}

	\begin{defn}
		Une famille $A=(v_1,\ldots,v_q)$ dans $E$, un ev de $\mathbb{K}$, est génératrice si et seulement si :
		$$ \forall b\in E,\quad \exists(\lambda_1,\ldots,\lambda_q)\in\mathbb{K}^q,\quad \sum_{i=1}^{q} \lambda_iv_i = b $$
	\end{defn}
	Avec une famille génératrice, on peut générer tout l'espace.
	\section{Base}
	\begin{defn}
		Une base de $E$ est une famille libre et génératrice.

		On note $\mathrm{dim}(E)$ le cardinal d'une base de $E$.
	\end{defn}
	$\mathrm{dim}(E)$ est unique.

	\begin{thm}
		Soit $A$ une famille de vecteurs de cardinal $q$.

		Si le rang de $A$ vaut $\mathrm{dim}(E)$, alors $A$ est génératrice (i.e. $\forall b\in E,\exists X,\quad AX=b$).

		(Rappel) Si le rang de $A$ vaut $q$, alors $A$ est libre (i.e. $\exists!X,\quad AX=0$).
	\end{thm}
	Une matrice $A$ carrée de $\mathcal{M}_n$ est une base si et seulement si son rang vaut $n$.

	On remarque donc que $A$ est une base s'il existe des opérations élémentaires permettant d'écrire une multiplication des opérations élémentaires successives par $A$ égal à $I_n$. Cela montre que $A$ est inversible et que $A^{-1}$ est la multiplication des opérations élémentaires successives.

	Pour trouver l'inverse, on fait le pivot de Gauss sur la matrice et sur la matrice identité correspondante.

	\begin{exemple}[Technique pour trouver l'inverse]
		On a :
		$$ \begin{pmatrix} 1&3&0&|&1&0&0\\ 2&1&0&|&0&1&0\\0&0&1&|&0&0&1 \end{pmatrix} \rightarrow \begin{pmatrix} 1&3&0&|&1&0&0\\ 0&-5&0&|&-2&1&0\\0&0&1&|&0&0&1 \end{pmatrix} $$
		$$\rightarrow \begin{pmatrix} 1&3&0&|&1&0&0\\ 0&1&0&|&2/5&-1/5&0\\0&0&1&|&0&0&1 \end{pmatrix}\rightarrow  \begin{pmatrix} 1&0&0&|&-1/5&3/5&0\\ 0&1&0&|&2/5&-1/5&0\\0&0&1&|&0&0&1 \end{pmatrix}$$
		Ainsi, on a que la matrice $\small\begin{pmatrix}-1/5&3/5&0\\2/5&-1/5&0\\0&0&1 \end{pmatrix}$ est l'inverse de la première matrice. 

		NLDR: les étapes sont foireuses mais on a la marche à suivre
	\end{exemple}
\end{document}

