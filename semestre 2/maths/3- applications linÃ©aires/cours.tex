%%=====================================================================================
%%
%%       Filename:  cours.tex
%%
%%    Description:  
%%
%%        Version:  1.0
%%        Created:  03/06/2024
%%       Revision:  none
%%
%%         Author:  YOUR NAME (), 
%%   Organization:  
%%      Copyright:  Copyright (c) 2024, YOUR NAME
%%
%%          Notes:  
%%
%%=====================================================================================
\documentclass[a4paper, titlepage]{article}

\usepackage[utf8]{inputenc}
\usepackage[T1]{fontenc}
\usepackage{textcomp}
\usepackage[french]{babel}
\usepackage{amsmath, amssymb}
\usepackage{amsthm}
\usepackage[svgnames]{xcolor}
\usepackage{thmtools}
\usepackage{lipsum}
\usepackage{framed}
\usepackage{parskip}
\usepackage{titlesec}
\usepackage{hyperref}

\renewcommand{\familydefault}{\sfdefault}

% figure support
\usepackage{import}
\usepackage{xifthen}
\pdfminorversion=7
\usepackage{pdfpages}
\usepackage{transparent}
\newcommand{\incfig}[1]{%
	\def\svgwidth{\columnwidth}
	\import{./figures/}{#1.pdf_tex}
}

\pdfsuppresswarningpagegroup=1

\colorlet{defn-color}{DarkBlue}
\colorlet{props-color}{Blue}
\colorlet{warn-color}{Red}
\colorlet{exemple-color}{Green}
\colorlet{corol-color}{Orange}
\newenvironment{defn-leftbar}{%
  \def\FrameCommand{{\color{defn-color}\vrule width 3pt} \hspace{10pt}}%
  \MakeFramed {\advance\hsize-\width \FrameRestore}}%
 {\endMakeFramed}
\newenvironment{warn-leftbar}{%
  \def\FrameCommand{{\color{warn-color}\vrule width 3pt} \hspace{10pt}}%
  \MakeFramed {\advance\hsize-\width \FrameRestore}}%
 {\endMakeFramed}
\newenvironment{exemple-leftbar}{%
  \def\FrameCommand{{\color{exemple-color}\vrule width 3pt} \hspace{10pt}}%
  \MakeFramed {\advance\hsize-\width \FrameRestore}}%
 {\endMakeFramed}
\newenvironment{props-leftbar}{%
  \def\FrameCommand{{\color{props-color}\vrule width 3pt} \hspace{10pt}}%
  \MakeFramed {\advance\hsize-\width \FrameRestore}}%
 {\endMakeFramed}
\newenvironment{corol-leftbar}{%
  \def\FrameCommand{{\color{corol-color}\vrule width 3pt} \hspace{10pt}}%
  \MakeFramed {\advance\hsize-\width \FrameRestore}}%
 {\endMakeFramed}

\def \freespace {1em}
\declaretheoremstyle[headfont=\sffamily\bfseries,%
 notefont=\sffamily\bfseries,%
 notebraces={}{},%
 headpunct=,%
 bodyfont=\sffamily,%
 headformat=\color{defn-color}Définition~\NUMBER\hfill\NOTE\smallskip\linebreak,%
 preheadhook=\vspace{\freespace}\begin{defn-leftbar},%
 postfoothook=\end{defn-leftbar},%
]{better-defn}
\declaretheoremstyle[headfont=\sffamily\bfseries,%
 notefont=\sffamily\bfseries,%
 notebraces={}{},%
 headpunct=,%
 bodyfont=\sffamily,%
 headformat=\color{warn-color}Attention\hfill\NOTE\smallskip\linebreak,%
 preheadhook=\vspace{\freespace}\begin{warn-leftbar},%
 postfoothook=\end{warn-leftbar},%
]{better-warn}
\declaretheoremstyle[headfont=\sffamily\bfseries,%
 notefont=\sffamily\bfseries,%
notebraces={}{},%
headpunct=,%
 bodyfont=\sffamily,%
 headformat=\color{exemple-color}Exemple~\NUMBER\hfill\NOTE\smallskip\linebreak,%
 preheadhook=\vspace{\freespace}\begin{exemple-leftbar},%
 postfoothook=\end{exemple-leftbar},%
]{better-exemple}
\declaretheoremstyle[headfont=\sffamily\bfseries,%
 notefont=\sffamily\bfseries,%
 notebraces={}{},%
 headpunct=,%
 bodyfont=\sffamily,%
 headformat=\color{props-color}Proposition~\NUMBER\hfill\NOTE\smallskip\linebreak,%
 preheadhook=\vspace{\freespace}\begin{props-leftbar},%
 postfoothook=\end{props-leftbar},%
]{better-props}
\declaretheoremstyle[headfont=\sffamily\bfseries,%
 notefont=\sffamily\bfseries,%
 notebraces={}{},%
 headpunct=,%
 bodyfont=\sffamily,%
 headformat=\color{props-color}Théorème~\NUMBER\hfill\NOTE\smallskip\linebreak,%
 preheadhook=\vspace{\freespace}\begin{props-leftbar},%
 postfoothook=\end{props-leftbar},%
]{better-thm}
\declaretheoremstyle[headfont=\sffamily\bfseries,%
 notefont=\sffamily\bfseries,%
 notebraces={}{},%
 headpunct=,%
 bodyfont=\sffamily,%
 headformat=\color{corol-color}Corollaire~\NUMBER\hfill\NOTE\smallskip\linebreak,%
 preheadhook=\vspace{\freespace}\begin{corol-leftbar},%
 postfoothook=\end{corol-leftbar},%
]{better-corol}

\declaretheorem[style=better-defn]{defn}
\declaretheorem[style=better-warn]{warn}
\declaretheorem[style=better-exemple]{exemple}
\declaretheorem[style=better-corol]{corol}
\declaretheorem[style=better-props, numberwithin=defn]{props}
\declaretheorem[style=better-thm, sibling=props]{thm}
\newtheorem*{lemme}{Lemme}%[subsection]
%\newtheorem{props}{Propriétés}[defn]

\newenvironment{system}%
{\left\lbrace\begin{align}}%
{\end{align}\right.}

\newenvironment{AQT}{{\fontfamily{qbk}\selectfont AQT}}

\usepackage{LobsterTwo}
\titleformat{\section}{\newpage\LobsterTwo \huge\bfseries}{\thesection.}{1em}{}
\titleformat{\subsection}{\vspace{2em}\LobsterTwo \Large\bfseries}{\thesubsection.}{1em}{}
\titleformat{\subsubsection}{\vspace{1em}\LobsterTwo \large\bfseries}{\thesubsubsection.}{1em}{}

\newenvironment{lititle}%
{\vspace{7mm}\LobsterTwo \large}%
{\\}

\renewenvironment{proof}{$\square$ \footnotesize\textit{Démonstration.}}{\begin{flushright}$\blacksquare$\end{flushright}}

\title{Applications linéaires et sous espace vectoriel}
\author{William Hergès\thanks{Sorbonne Université - Faculté des Sciences, Faculté des Lettres}}

\begin{document}
	\maketitle
	\tableofcontents
	\newpage
	\section{Définition}
	\begin{defn}
		Une application $f$ est dite linéaire de $E$ dans $F$ (deux sev) si et seulement si :
		$$ \forall (a,b)\in E^2,\forall (x,y)\in E^2,\quad f(ax+by) = af(x)+bf(y) $$
	\end{defn}
	\begin{thm}
		Toute application linéaire est représentable par une matrice.
	\end{thm}
	\begin{exemple}
		Représentation d'une application linéaire de $\mathbb{R}^3\to \mathbb{R}^2$ :
		$$ \begin{pmatrix} x\\y\\z \end{pmatrix} \longmapsto \begin{pmatrix} a_{1,1}&a_{1,2}&a_{1,3}\\a_{2,1}&a_{2,2}&a_{2,3} \end{pmatrix} \begin{pmatrix} x\\y\\z \end{pmatrix}  = \begin{pmatrix} a_{1,1}x+a_{1,2}y+a_{1,3}z\\a_{2,1}x+a_{2,2}y+a_{2,3}z \end{pmatrix} $$
	\end{exemple}
	\begin{defn}
		L'image de $A$ une matrice représentant l'application linéaire $f$ de $E$ dans $F$ est notée $\mathrm{Im}A$ et :
		$$ \mathrm{Im}A =\{AX|X\in E\}$$
	\end{defn}
	L'image est l'ensemble des éléments atteints par l'application linéaire représentée par $A$.
	\begin{defn}
		Le noyau de $A$ une matrice représentant l'application linéaire $f$ de $E$ dans $F$ est noté $\mathrm{Ker}A$ et :
		$$ \mathrm{Ker}A=\{X| AX = 0, X\in E\} $$
	\end{defn}
	Le noyau est l'ensemble des éléments donnant 0 par $f$.
	\begin{defn}
		La dimension d'un espace vectoriel est le nombre de vecteur d'une base (sauf si la base vaut $\{0\}$, dans ce cas là sa dimension vaut 0). On note la dimension de $E$ $\mathrm{dim}E$.

		D'une manière formelle, soit $f$ une base de $E$, on a :
		$$ \mathrm{dim}(E)=\mathrm{card}(f) $$
		(où $\mathrm{card}$ est le cardinal de $f$)\\
		sauf si $f=\{0\}$, où dans ce cas $\mathrm{dim}(E)=0$. 
	\end{defn}
	\begin{thm}
		La dimension de l'image de l'application linéaire $f$ représentée par les matrices $AX$ est égal au rang de $A$, i.e.
		$$ \mathrm{dim}~\mathrm{Im}A = \mathrm{rg}A $$
	\end{thm}
	\begin{thm}[Théorème du rang]
		Soit $f$ une application linéaire de $E$ dans $F$.

		On a que :
		$$ \mathrm{dim}E = \mathrm{dim}~\mathrm{Im}A+\mathrm{dim}~\mathrm{Ker}A $$
	\end{thm}
	\begin{thm}
		Les vecteurs colonnes au dessus de la matrice $A$ se trouvant au dessus des pivots constituent une base de l'image.
	\end{thm}
	\begin{exemple}
		On a :
		$$ \begin{pmatrix} 1&1&0\\1&1&0 \end{pmatrix} $$
		Après le pivot de Gauss, on obtient :
		$$ \begin{pmatrix} \fbox{1}&1&0\\0&0&0 \end{pmatrix} $$
		Donc, une base de l'image est $\begin{pmatrix} 1\\1 \end{pmatrix} $

		Comme $\begin{pmatrix} 1&0&0\\0&1&0 \end{pmatrix}$ est déjà échelonné, on a que $\begin{pmatrix} 1&0\\0&1 \end{pmatrix}$ est une base de l'image.
	\end{exemple}
	\section{Sous espace vectoriel}
	\begin{defn}
		Un sous espace vectoriel $V$ est un espace vectoriel si et seulement si :
		\begin{itemize}
			\item $V\neq \varnothing$
			\item pour tout $v_1,v_2\in V$, on a $v_1+v_2$ est bien dans $V$
			\item pour tout $v$ dans $V$ et pour tout $\lambda$ dans $\mathbb{K}$, on a $\lambda v\in V$
		\end{itemize}
	\end{defn}
	\begin{props}
		L'image et le noyau d'une application linéaire sont des sous-espaces vectoriels.
	\end{props}
	\section{Déterminer une base du noyau}
	On a une base de l'image et on a $A$, la matrice représentant l'application linéaire à l'origine.

	On sait que la base du noyau possède $\mathrm{dim}(E)-\mathrm{dim}~\mathrm{Im}(A)$ (théorème du rang).
	
	Pour chaque colonne sans pivot, on détermine un vecteur de la base du noyau (voir \href{}{ce gif})
\end{document}
